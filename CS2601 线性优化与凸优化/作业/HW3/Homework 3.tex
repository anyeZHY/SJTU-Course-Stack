%! Tex program = xelatex
\documentclass{article}

\usepackage{amsmath,amssymb}
\usepackage{ntheorem}
\usepackage[letterpaper,top=2cm,bottom=2cm,left=3cm,right=3cm,marginparwidth=1.75cm]{geometry}%table package
\usepackage{multirow,booktabs}
\usepackage{makecell}
\usepackage{listings}
\usepackage{graphicx}
\usepackage{enumerate}
\usepackage{color}
\usepackage{physics}

\def\RR{\mathbb{R}}
\def\bw{\boldsymbol{\omega}}
\def\bx{\boldsymbol{x}}
\def\by{\boldsymbol{y}}
\def\bd{\boldsymbol{d}}
\def\bf{\boldsymbol{f}}
\def\vvert{\vert\vert}
\def\bA{\boldsymbol{A}}
\def\bB{\boldsymbol{B}}
\def\bC{\boldsymbol{C}}
\def\bO{\boldsymbol{O}}
\def\convex#1#2{\theta#1+\bar{\theta}#2}

\def\Esolve{\textcolor{blue}{Solve: }}
\def\Eproof{\textcolor{blue}{Proof: }}
\def\Eprco{\textcolor{blue}{Proof by contradiction: }}

\newtheorem{lemma}{Lemma}
\newtheorem{proof}{Proof}
\newtheorem*{theorem}{Theorem}


\begin{document}
\title{Homework 3}
\author{Zhen}
\maketitle

\section*{Problem 1}
Suppose f is a convex function and $S \subset$ dom $f$ is a convex set. Let M be the set of global minima of f over S,
$$M=\qty{\bx^*\in S:f(\bx^*)\le f(\bx),\forall\bx\in S}$$
Show that $M$ is a convex set.
\\
\Eproof

Apparently, $M=\qty{\bx:f(\bx)=C}$ where $C$ is a const and $\forall \bx\in C,f(\bx)\ge C$.

$\forall \bx,\by\in M$ and $\theta\in(0,1)$,we have
$$C\le f(\convex{\bx}{\by})\le\convex{f(\bx)}{f(\by)}=C,$$
which means that $f(\convex{\bx}{\by})=C$.

Thus $\convex{\bx}{\by}\in M$ and $M$ is convex.

\section*{Problem 2}
Let $f$ be convex. If $f(\convex{\bx}{\by})=\convex{f(\bx)}{f(\by)}$ for some $\bx,\by$ and $\theta=\theta_0\in(0,1)$, then it holds for the same $\bx,\by$ and any $\theta\in[0,1]$.
\\
\Eprco

Without loss of generality, assume that
$f(\theta_1\bx+\bar{\theta_1}\by)<\theta_1f(\bx)+\bar{\theta_1}f(\by)$ where $\theta_1\in(\theta_0,1)$.

Then we have

$$
\begin{aligned}
	\frac{\theta_0}{\theta_1}
	f\qty(\theta_1\bx+\bar{\theta_1}\by)
	+
	\qty(1-\frac{\theta_0}{\theta_1})
	f(\by)
	&<
	\frac{\theta_0}{\theta_1}
	\qty(\theta_1f(\bx)+\bar{\theta_1}f(\by))
	+
	\qty(1-\frac{\theta_0}{\theta_1})
	f(\by)
	\\
	&=
	\theta_0 f(\bx)+(1-\theta_0)f(\by)
	\\
	&=
	f\qty(\theta_0\bx+\bar{\theta_0}\by)
	\\
	&=
	f\qty(
	\frac{\theta_0}{\theta_1}\qty(\theta_1\bx+\bar{\theta_1}\by)+\qty(1-\frac{\theta_0}{\theta_1})\by
)
\end{aligned}
$$

Let $\theta_2=\dfrac{\theta_0}{\theta_1}$, $\bx_1=\qty(\theta_1\bx+\bar{\theta_1}\by)$ and $\bx_2=\by$. Thus we get

$$
\theta_2f(\bx_1)+\bar{\theta_2}f(\bx_2)<f\qty(\theta_2\bx_1+\bar{\theta_2}\bx_2)
$$
which deduce a contradiction.

Therefore, $\forall\theta\in(0,1):\convex{f(\bx)}{f(\by)}\le f(\convex{\bx}{\by})\le\convex{f(\bx)}{f(\by)}$.

Then $f(\convex{\bx}{\by})=\convex{f(\bx)}{f(\by)}$, $\forall \theta\in(0,1)$.


\newpage

\section*{Problem 3}
Determine if the following functions are convex, concave, or neither.

\begin{enumerate}[(a)]
	\item $f(\bx)=f(x_1,x_2,x_3)=x_1^2+x_1x_3+x_2^2+x_2x_3+\frac12x_3^2$ on $\RR^3$
		
		\Esolve Convex.

		Assume $f(\bx)=\bx^T\bA\bx$. By the fact that
		$$
		f(\bx)=\qty(x_1+\frac12x_3)^2+\qty(x_2+\frac12x_3)^2
		$$
		we have $\nabla^2f=2\bA\succeq\bO$. Therefore $f$ is convex.

	\item $f(\bx)=f(x_1,x_2)=(x_1x_2)^{-1}$ on $\RR_{++}^2=\qty{(x_1,x_2):x_1>0,x_2>0}$

		\Esolve Convex.

		Considering that:
		$\displaystyle{
			\pdv[2]{f}{x_1}=\dfrac{2}{x_1^3x_2},
			\pdv[2]{f}{x_2}=\dfrac{2}{x_1x_2^3},
			\pdv{f}{x_1}{x_2}=\dfrac{1}{x_1^2x_2^2}
		}$\\

		Then the eigenvalue $\lambda_1,\lambda_2$ of $\nabla^2f$ satisfies that: 
		$$
		(\lambda-\dfrac{2}{x_1^3x_2})(\lambda-\dfrac{2}{x_1x_2^3})-\dfrac{1}{x_1^4x_2^4}=0,\ i=1,2
		$$

		Let $f(\lambda)$ denote \textbf{LHS} of the above equation. Then $f(\lambda)=0$ is a parabola with an axis of symmetry $\lambda=\dfrac{1}{x_1^3x_2}+\dfrac{1}{x_1x_2^3}>0$.\\

		And considering that $f(0)>0$ and $f(\lambda)=0$ is of 2 unique solutions, we have $\lambda_1,\lambda_2>0$. Thus $\nabla^2f\succeq\bO$ and $f$ is convex.
	\item
		$f(x_1,x_2)=x_1x_2^2$ on $\RR_{++}^2$

		\Esolve Neither.
		$$\nabla^2f=\mqty(0 & 2x_2 \\ 2x_2 & 2x_1)$$
		Then the eigenvalue of $\nabla^2f$ $\lambda_{\pm}=x_1\pm\sqrt{x_1^2+4x_2^2}$. Thus $f$ is neither convex nor concave ($\lambda_+>0$ while $\lambda_-<0$).

	\item $f(x_1,x_2)=x_1x_2^{-1/2}$ on $\RR_{++}^2$

		\Esolve Neither.

		Considering that:
		$\displaystyle{\pdv[2]{f}{x_1}=0,\ \pdv{f}{x_1}{x_2}=-\dfrac{1}{2x_2^{3/2}},\ \pdv[2]{f}{x_2}=\dfrac{3x_1}{4x_2^{5/2}}}$. \\

		Then the eigenvalue of $\nabla^2f$ 
		$$
		\lambda_{\pm}=\frac12\qty(\dfrac{3x_1}{4x_2^{5/2}}
			\pm
		\sqrt{\dfrac{9x_1^2}{16x_2^5}+\dfrac{1}{x_2^3}})
		$$
		Thus $f$ is neither convex nor concave ($\lambda_+>0$ while $\lambda_-<0$).



	\item $f(x_1,x_2)=x_1^\alpha x_2^{1-\alpha}$ on $\RR_{++}^2$

		\Esolve Concave.

		Considering that 
		$$
		\nabla^2f
		=
		-\alpha(1-\alpha)x_1^{\alpha-2}x_2^{-1-\alpha}
		\mqty[x_2^2 & x_1x_2 \\ x_1x_2 & x_2^2]
		\triangleq
		C(x_1,x_2,\alpha)\bA
		$$
		By the fact that 
		$C(x_1,x_2,\alpha)<0\land \bA\succeq\bO$, 
		$\nabla^2f\preceq\bO$ and $f$ is concave.
\end{enumerate}

\newpage

\section*{Problem 4}

Suppose $f_i:\RR\to\RR, i=1,2$, are strictly convex functions. Show that $f:\RR^2\to\RR$ defined by $f(x_1,x_2) = f_1(x_1)+f_2(x_2)$ is strictly convex over $\RR^2$, and in particular $f(x_1,x_2) = x_1^2 +x_2^4$ is strictly convex.

\Eproof

$\forall\bx=(x_1,x_2),\by=(y_1,y_2)\in\RR^2$ and $\forall\theta\in(0,1)$, 
$$
\begin{aligned}
	f(\convex{\bx}{\by})
	&=f(\convex{x_1}{y_1},\convex{x_2}{y_2})
	\\
	&=
	f_1(\convex{x_1}{y_1})+f_2(\convex{x_2}{y_2})
	\\
	&\le
	\convex{f_1(x_1)}{f_1(y_1)}+\convex{f_2(x_2)}{f_2(y_2)}
	\\
	&=
	\convex{f(\bx)}{f(\by)}
\end{aligned}
$$
Therefore $f$ is strictly convex.

Trivially, $f_1(x_1)=x_1^2$ and $f_2(x_2)=x_2^4$ are strictly conevx. Then $f(x_1,x_2)=f_1(x_1)+f_2(x_2)$ is strictly convex.

\section*{Problem 5}
Let $f:C\subset\RR^n\to\RR$ be a differentiable function defined on a nonempty open convex set $C$. Show that $f$ is convex if and only if
$$
\langle\nabla f(\bx)-\nabla f(\by),\bx-\by\rangle\ge0, \forall \bx,\by\in C
$$

\Eproof
$$
\begin{aligned}
	&f \mbox{ is convex} 
	\\ \iff
	&
	\forall\mbox{direction } \bd,\forall \bx\in C, g(t)=f(\bx+t\bd) \mbox{ is convex on } \qty{t:\bx+t\bd\in C}
	\\ \iff
	&
	\forall\mbox{direction } \bd,\forall \bx\in C, g'(t)=\nabla f(\bx+t\bd)^T\bd \mbox{ is increasing on } \qty{t:\bx+t\bd\in C}
	\\ \iff
	&
	\forall\mbox{direction } \bd,\forall \bx\in C,\forall h,s\in\qty{t:\bx+t\bd\in C}, \qty[g'(h)-g'(s)](h-s)\ge0
	\\ \iff
	&
	\forall\mbox{direction } \bd,\forall \bx\in C,\forall h,s\in\qty{t:\bx+t\bd\in C}, \qty[\nabla f(\bx+h\bd)^T\bd-f(\bx+s\bd)^T\bd](h-s)\ge0
	\\ \iff
	&
	\forall\mbox{direction } \bd,\forall \bx\in C,\forall h,s\in\qty{t:\bx+t\bd\in C}, \qty[\nabla f(\bx+h\bd)^T-f(\bx+s\bd)^T]\qty[(\bx+h\bd)-(\bx+s\bd)]\ge0
	\\ \iff
	&
	\forall \bx,\by\in C,\qty[\nabla f(\bx)^T-\nabla f(\by)^T]\qty(\bx-\by)\ge0 \mbox{ (Note that: }\exists s,\bd,\ \bx=\bx+0\bd, \by=\bx+t\bd\mbox{)}
	\\ \iff
	&
	\forall \bx,\by\in C,\langle\nabla f(\bx)-\nabla f(\by),\bx-\by\rangle\ge0
\end{aligned}
$$
\end{document}




