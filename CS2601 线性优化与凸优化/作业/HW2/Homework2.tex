%! Tex program = xelatex
\documentclass{article}

\usepackage{amsmath,amssymb}
\usepackage[letterpaper,top=2cm,bottom=2cm,left=3cm,right=3cm,marginparwidth=1.75cm]{geometry}%table package
\usepackage{multirow,booktabs}
\usepackage{makecell}
\usepackage{listings}
\usepackage{graphicx}
\usepackage{enumerate}

\def\RR{\mathbb{R}}
\def\bw{\boldsymbol{\omega}}
\def\bb{\boldsymbol{b}}
\def\bd{\boldsymbol{d}}
\def\bx{\boldsymbol{x}}
\def\by{\boldsymbol{y}}
\def\bz{\boldsymbol{z}}

\def\btheta{\boldsymbol{\theta}}

\def\bA{\boldsymbol{A}}
\def\bB{\boldsymbol{B}}
\def\bC{\boldsymbol{C}}
\def\bX{\boldsymbol{X}}
\def\bY{\boldsymbol{Y}}
\def\bZ{\boldsymbol{Z}}

\def\vvert{\vert\vert}
\def\dd{\mathrm{d}}

\def\convex#1#2{\theta#1+(1-\theta)#2}

\newtheorem{lemma}{Lemma}
\newtheorem{proof}{Proof}
\newtheorem{theorem}{Theorem}

\begin{document}
\title{Homework 2}
\author{Zhen Haoyu}
\maketitle


\section*{Problem 1}

Let
$$
f(\bx)=\bA\bx+\bb.
$$
If $C$ is convex, then $\forall \bx,\by\in f^{-1}(C)$ (i.e., $f(\bx),f(\by)\in C$), $\forall \theta \in (0,1)$, considering that:
$$
\begin{aligned}
	f\big(\theta \bx+(1-\theta)\by\big)
	&=\theta \bA\bx+\theta \bb+(1-\theta)\bA\by+(1-\theta)\bb
	\\
	&=\theta f(\bx)+(1-\theta)f(\by)
	\\
	&\in C
\end{aligned}
$$
Note that: the last line of the equation uses the property of convex set. Then $(\theta \bx+(1-\theta)\by\in f^{-1}(C)$.

\section*{Problem 2}

Trivially, $C$ is a nonempty set.

$\forall \bx,\by \in C$, $\exists \bx_1,\by_1\in C_1 \land \bx_2,\by_2\in C_2$, $s.t.$ $\bx_1-\bx_2=\bx \land \by_1-\by_2=\by$.

Then $\forall\theta in (0,1)$, we have:
$$
\theta \bx +(1-\theta)\by=\theta\bx_1+(1-\theta)\by_1-\theta\bx_2-(1-\theta)\by_2
$$

By $C_1,C_2$ is convex, $\exists \bz_1\in C_1 \land \bz_2\in C_2$, $\theta \bx +(1-\theta)\by=\bz_1-\bz_2\in C$, which means that $C$ is convex.

Trivially, $\boldsymbol{0}\in C$. (If not, then $\exists \bx \in C_1,C_2$, which means $C_1\cap C_2\neq\varnothing$)

\section*{Problem 3}

\begin{enumerate}[(a)]
	\item 
		$\forall \bx_0,\by_0 \in int\ C$, $\exists\ \varepsilon>0$,
		$$\big\{\bx|\ (\bx-\bx_0)^2\le\varepsilon^2\big\}\subset C\land\big\{\by|\ (\by-\by_0)^2\le\varepsilon^2\big\}\subset C.$$

		For the sake of simplicity, let 
		$X$ denotes $\big\{\bx|\ (\bx-\bx_0)^2\le\varepsilon^2\big\}$, 
		$Y$ denotes $\big\{\by|\ (\by-\by_0)^2\le\varepsilon^2\big\}$
		and 
		$\bz_0$ denotes $\theta \bx_0+(1-\theta)\by_0$.

		Then $\forall
		\bz\in\big\{\bx|(\bx-\bz_0)^2\le\varepsilon^2\big\}$,
		$
		\exists
		\bz-\bz_0+\bx_0\in X 
		\land
		\bz-\bz_0+\by_0\in Y
		$, we have
		$$\bz=\theta(\bz-\bz_0+\bx_0)+(1-\theta)(\bz-\bz_0+\by_0).$$

		Thus, $\bz\in C$, which means that $\theta \bx_0+(1-\theta)\by_0$ is a interior point of $C$.
		
		Therefore, $\theta \bx_0+(1-\theta)\by_0\in int\ C$ and $int\ C$ is convex.

		\newpage
	\item
		$\forall \bx_0,\by_0 \in \bar{C}$, we have 2 infinite sequence $\{\bx_n\},\{\by_n\}$, such that
		$$
		\lim_{n\to\infty}\bx_n=\bx_0
		\land
		\lim_{n\to\infty}\by_n=\by_0,
		$$
		and
		$$
		\bx_n,\by_n\in C\ (\forall n).
		$$
		
		Thus we have: $\forall \theta\in(0,1)$, 
		$$
		\lim_{n\to\infty}(\convex{\bx_n}{\by_n})=\convex{\bx_0}{\by_0}
		$$
		where $\forall n$, $\convex{\bx_n}{\by_n}\in C$. Thus  $\convex{\bx_0}{\by_0}\in \bar{C}$, which means that $\bar{C}$ is convex.
\end{enumerate}

\section*{Problem 4}

\begin{enumerate}[(a)]
	\item 
		Let 
		$
		\btheta_{im}
		\equiv
		(\theta_{i1},\theta_{i2},\dots,\theta_{im})^T
		$, $
		\bX_{m}
		\equiv
		(\bx_{i1},\bx_{i2},\dots,\bx_{im})^T
		$ and $
		\bY_{n}
		\equiv
		(\by_{i1},\by_{i2},\dots,\by_{in})^T
		$.

		$\forall \bx,\by\in C$, 
		$\exists \btheta_{1m},\btheta_{2n},\bX_{m},\bY_{n}$,
		such that 
		$$
		\begin{cases}
			&
			\forall k\in\{1,2,\dots,m\}:
			\theta_{1k}\ge0 \land \bx_{1k}\in S
			\\
			&
			\bx=\btheta_{1m}^T\bX_{m}
			\\
			&
			\forall l\in\{1,2,\dots,n\}:
			\theta_{2k}\ge0 \land \by_{2k}\in S
			\\
			&
			\by=\btheta_{2n}^T\bY_{n}
			\\
			&
			\sum_{k=1}^m\theta_{1k}=\sum_{l=1}^n\theta_{2l}=1
		\end{cases}		
		$$

		Then $\forall\theta\in(0,1)$:
		$$
		\convex{\bx}{\by}=
		\begin{bmatrix}
			\theta\btheta_{1m}^T & (1-\theta)\btheta_{2n}^T
		\end{bmatrix}
		\begin{bmatrix}
			\bX_m
			\\
			\bY_n
		\end{bmatrix}
		$$

		Let $\btheta_{3,m+n}$ denote 
		$
		\begin{bmatrix}
			\theta\btheta_{1m}^T & (1-\theta)\btheta_{2n}^T
		\end{bmatrix}
		$ and $\bZ$ denote
		$
		\begin{bmatrix}
			\bX_m
			\\
			\bY_n
		\end{bmatrix}
		$.

		Trivially, for every row vector $\bz_i$ in $\bZ$: $\bz_i\in S$ and for every element $t$ in $\btheta_{3,m+n}$: $t\ge0$.

		Also, $$sum(\btheta_{3,m+n})=
		\theta\sum_{k=1}^m\theta_{1k}
		+
		(1-\theta)\sum_{l=1}^n\theta_{2l}=1
		$$
		
		Thus $\convex{\bx}{\by}\in C$ and $C$ is convex.
	\item
		The defination of $conv\ S$ is the smallest convex set containing S.

		$\forall \bx\in C$, by the defination of $C$, we have
		$\bx=\sum_{i=1}^m\theta_i\bx_i$, where 
		$
		\forall i\in\{1,2,\dots,m\}:\theta_i>0
		\land
		\sum_{i=1}^m\theta_i=1
		$

		Then considering that
		$$
		\begin{aligned}
		\bx
		&=
		\theta_1\bx_1+(1-\theta_1)
		\sum_{i=2}^m\frac{\theta_i}{1-\theta_1}\bx_i
		\\
		&=
		\theta_1\bx_1+(1-\theta_1)
		\bigg[
			\frac{\theta_2}{1-\theta_1}\bx_2
			+
			\sum_{i=3}^m\frac{\theta_i}
			{1-\frac{\theta_2}{1-\theta_1}}\bx_i
		\bigg]
		\\
		&
		\dots
		\end{aligned}
		$$

		The above equation demonstrates the decomposition of $\bx$. To prove $\bx\in conv S$, we could to prove $\sum_{i=2}^m\frac{\theta_i}{1-\theta_1}\bx_i\in conv S$ where $\sum_{i=2}^m\frac{\theta_i}{1-\theta_1}=1$. By finite steps we could only to prove $\bx_m\in conv S$.

		Thus we have $\bx\in conv S$. Then $C\subset conv S$.

		Obviously $S\subset C$. By $C$ is convex we have $conv S\subset C$.

		Therefore $C= conv S$.
\end{enumerate}

\newpage
\section*{Problem 5}

The defination of \textbf{Vonoroi region} is 
$$
V=
\{
	\bx\in\RR:
	\vvert\bx-\bx_0\vvert_2\le\vvert\bx-\bx_i\vvert_2,
	i=1,2,\dots,K
\}.
$$
Let 
$
V_i=
\{
	\bx\in\RR:
	\vvert\bx-\bx_0\vvert_2\le\vvert\bx-\bx_i\vvert_2
\}
$. Then $V=\bigcap_{i=1}^KV_i$.

\begin{lemma}
	$\vvert\bx-\bx_0\vvert_2\le\vvert\bx-\bx_i\vvert_2$
	if and only if 
	$\bA_i(\bx-\dfrac{\bx_0+\bx_i}{2})\le0$, 
	where $\bA_i=(\bx_i-\bx_0)^T$.
\end{lemma}

\begin{proof}
	$$
	\begin{aligned}
		&
		\vvert\bx-\bx_0\vvert_2\le\vvert\bx-\bx_i\vvert_2
		\\
		\iff
		&
		(\bx-\bx_0)^T(\bx-\bx_0)\le(\bx-\bx_i)^T(\bx-\bx_i)
		\\
		\iff
		&
		2(\bx_i-\bx_0)^T\bx+\bx_0^T\bx_0-\bx_i^T\bx_i \le 0
		\\
		\iff
		&
		2(\bx_i-\bx_0)^T\bx-(\bx_i-\bx_0)^T(\bx_i+\bx_0)\le0
		\\
		\iff
		&
		(\bx_i-\bx_0)^T(\bx-(\bx_i+\bx_0)/2)\le0
	\end{aligned}
	$$
	Q.E.D.
\end{proof}

Thus, $\forall \bx_i$, $\exists \bA_i\in\RR^{1\times n},b_i\in\RR$, $s.t.\ V_i=\{\bx:\bA_i\bx\le b_i\}$.
Then let 
$\bA=
\begin{bmatrix}
\bA_1\\
\bA_2\\
\vdots\\
\bA_n
\end{bmatrix}$
and 
$\bb=
\begin{bmatrix}
b_1\\
b_2\\
\vdots\\
b_n
\end{bmatrix}$.

Therefore, $V=\bigcap_{i=1}^KV_i=\{\bA\bx\le\bb\}$
\end{document}

