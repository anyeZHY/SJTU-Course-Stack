%! Tex program = xelatex
\documentclass{article}

\usepackage{amsmath,amssymb}
\usepackage{ntheorem}
\usepackage[letterpaper,top=2cm,bottom=2cm,left=3cm,right=3cm,marginparwidth=1.75cm]{geometry}%table package
\usepackage{multirow,booktabs}
\usepackage{makecell}
\usepackage{listings}
\usepackage{graphicx}
\usepackage{enumerate}

\def\RR{\mathbb{R}}
\def\bw{\boldsymbol{\omega}}
\def\bx{\boldsymbol{x}}
\def\bd{\boldsymbol{d}}
\def\vvert{\vert\vert}
\def\dd{\mathrm{d}}
\def\bA{\boldsymbol{A}}
\def\bB{\boldsymbol{B}}
\def\bC{\boldsymbol{C}}

\newtheorem{lemma}{Lemma}
\newtheorem{proof}{Proof}
\newtheorem*{theorem}{Theorem}

\begin{document}
\title{Homework 1}
\author{Zhen Haoyu}
\maketitle

\section*{Problem 1}

$f(x)=2x_1^2 +x_1x_2+x_2^2-3x_1-5x_2$

\begin{enumerate}[(a)]
	\item Because
		$$
		2x_1^2 +x_1x_2 + x_2^2-3x_1-5x_2
		\ge
		\frac{3}{2}x_1^2 + \frac{1}{2}x_2^2-3x_1-5x_2
		$$
		And if $$|x_1|, |x_2|>100,$$
		then $$f(x)\ge
		\frac{1}{2}x_1^2 + \frac{1}{4}x_2^2.$$

		Thus we get: when $\vert\vert x\vert\vert\rightarrow+\infty$, $f(x)\rightarrow+\infty$, which means that $f(x)$ is coercive.
	\item The maximun of $f(x)$ does \textbf{not} exist.
		
		The local minimun $(x_1,x_2)$ should satisfy:

		$$\frac{\partial f}{\partial x_1}=4x_1+x_2-6=0$$
		and 
		$$\frac{\partial f}{\partial x_2}=x_1+2x_2-5=0.$$
		Then we get $(x_1,x_2)=(\dfrac{1}{7},\dfrac{17}{7})$ and $\nabla^2 f$ is positive definite. 

		Thus the minimum is $-\dfrac{44}{7}$.
\end{enumerate}


\newpage
\section*{Problem 2}

$$ f(\bw)=\sum_{i=1}^m \log(1+e^{-y_i\bx_i^T\bw}) $$

\begin{enumerate}[(a)]
	\item
		By $\log(1+e^x)>0$, we have:
		$$\forall \bw,\ f(\bw)>0.$$

		And $$
		\lim_{k\to+\infty}f(k\bw_0)=
		\lim_{k\to+\infty}\sum_{i=1}^m \log(1+e^{-ky_i\bx_i^T\bw_0})=0
		$$

		Thus $f$ does not have a global minimum.
	\item
		\textbf{Proof by step:}
		\begin{enumerate}[(i)]
			\item 
			\begin{equation*}
			\begin{aligned}
				f(\bw)\ & =\ \  
				\sum_{i=1}^m \log(1+e^{-y_i\bx_i^T\bw})
				\\
				& \ge
				\sum_{y_i\bx_i^T\bw<0}
				\log(1+e^{-y_i\bx_i^T\bw})
				\\
				& \ge
				\sum_{y_i\bx_i^T\bw<0}
				-y_i\bx_i^T\bw
				\\
				& \ge
				\max_{y_i\bx_i^T\bw<0}-y_i\bx_i^T\bw
				\\
				& =
				\max_{1 \le i \le m} -y_i\bx_i^T\bw
			\end{aligned}
			\end{equation*}
			
		\item
			Set $S$ is bounded and closed while the function $h(\bw)$ is continuos.

			By Extreme Value Theorem, $h(\bw)$ has a global minimum.

		\item
			$\forall \bw\in S$:
			$$
			\frac{h(\bw)}{\vvert\bw\vvert}
			\ge
			\max_{1\le i\le m} -y_i\bx_i^T\frac{\bw}{\vvert\bw\vvert}
			\ge C
			$$

		Thus
		$$h(\bw)\ge C\vvert\bw\vvert$$
		\item
			By (i), (ii) and (iii), we have:
			$$
			\exists C\in\RR,\ f(\bw)\ge C\vvert\bw\vvert
			$$
			So when $\vvert\bw\vvert+\infty$, $f(\bw)\to+\infty$. Then function $f$ is coercive and continuous.

			Thus, $f$ has a global minimum.
		\end{enumerate}

	\item 
		$$\nabla f=\sum_{i=1}^m 
		\frac{-y_i\bx_i e^{-y_i\bx_i^T\bw}}
		{1+e^{-y_i\bx_i^T\bw}}
		$$


\end{enumerate}

\newpage
\section*{Problem 3}

\begin{enumerate}[(a)]
	\item Given $\bx$, let 
		$$R(\bd)=f(\bx+\bd)-f(\bx)-\nabla f(\bx)^T\bd$$

		Then
		$$
		\nabla R(\bd) = 
		\nabla f(\bx+\bd)-\nabla f(\bx)
		$$
		
		By Lagrange mean value theorem, $\exists \boldsymbol{\xi}\in(\bx,\ \bx+\bd)$:
		$$
		\nabla f(\bx+\bd)-\nabla f(\bx) = 
		\nabla^2 f(\boldsymbol{\xi})\bd
		$$

		\begin{lemma}
			If $\nabla f(\bx)=\bA\bx$ and $\bA$ is symmetric, then $f(x)=\frac{1}{2}\bx^TA\bx+C$, where $\bA=(a_{ij})_{n\times n}$, $\bx=(x_1,x_2,\dots,x_n)^T$ and $C$ is a const.
		\end{lemma}

		\begin{proof}
			$$
			f(x) = 
			\int\sum_{i=1}^n
			\frac{\partial f}{\partial x_i}\dd x_i = 
			\int\sum_{i=1}^n\sum_{j=1}^n
			a_{ij}x_j\dd x_i =
			\sum_{i\neq j}a_{ij}x_i x_j + 
			\sum_{i=1}^n \frac{1}{2}a_{ii}x_i^2+C
			=
			\frac{1}{2}\bx^TA\bx+C
			$$
		\end{proof}

		Because $\nabla^2f(\boldsymbol{\xi})$ is symmetric, we have:
		$$
		\begin{aligned}
		R(\bd)
		&=
		\frac{1}{2}\bd^T\nabla^2 f(\boldsymbol{\xi})\bd+C
		\\[4pt]
		&=
		\frac{1}{2}\bd^T\nabla^2 f(\bx+t\bd)\bd+C,
		\end{aligned}
		$$
		where $t=\dfrac{\boldsymbol{\xi}-\bx}{d}\in (0,\ 1)$.

		By $R(\boldsymbol{0})=0$, we have: $C=0$ and finnaly get:
		$$R(\bd)=\frac{1}{2}\bd^T\nabla^2 f(\bx+t\bd)\bd$$


	\item \textbf{Proof:}
		By the Newton-Leibniz formula, $\forall i\in{1,2,\dots,n}$:
		$$
		\int_0^1 \big[\nabla f_i(\bx+t\bd)\big]\bd\ \dd t
		=
		\int_0^1
		\sum_{j=1}^n f_{ij}(\bx+t\bd)\dd(td_j)
		=
		f_i(\bx+\bd) - 
		f_i(\bx)
		$$
		where $f_i=\dfrac{\partial f}{\partial x_i}$, 
		$f_{ij}=\dfrac{\partial^2 f}{\partial x_i\partial x_j}$ and 
		$\bd=(d_1,d_2,\dots,d_n)^T$.

		Also, we have
		$$
		\int_0^1\nabla^2 f(\bx+t\bd)\bd\ \dd t=
		\begin{pmatrix}
			\displaystyle{
			\int_0^1 \big[(\nabla f_1)(\bx+t\bd)\big]\bd\ \dd t
		} \\[10pt]


			\displaystyle{
			\int_0^1 \big[(\nabla f_2)(\bx+t\bd)\big]\bd\ \dd t
		} \\[3pt]

			\vdots \\[3pt]
			\displaystyle{
			\int_0^1 \big[(\nabla f_n)(\bx+t\bd)\big]\bd\ \dd t}
		\end{pmatrix}.
		$$
		

		Therefore,

		$$
		\int_0^1\nabla^2 f(\bx+t\bd)\bd\ \dd t
		=
		\nabla f(\bx+\bd) - 
		\nabla f(\bx)
		$$
\end{enumerate}

\section*{Problem 4}
\textbf{Solve:}\\

$D_1(\bA)=6$, $D_2(\bA)=26$ and $D_3(\bA)=30$. So $\bA$ is positive definite.\\

$D_1(\bB)=1$, $D_2(\bB)=-2$ and $D_3(\bB)=5$. So $\bB$ is indefinite.\\

$\bC$'s eigenvalues are $3,\ 3,\ 0$. So $\bC$ is positive semidefinite.\\

\end{document}

