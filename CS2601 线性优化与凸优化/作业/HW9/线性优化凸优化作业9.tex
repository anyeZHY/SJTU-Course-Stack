%! Tex program = xelatex
\documentclass{article}

%中文
%\usepackage[UTF8]{ctex}
%数学公式
\usepackage{amsmath,amssymb}
%\usepackage{ntheorem}
\usepackage{amsthm}
%边界
\usepackage[letterpaper,top=2cm,bottom=2cm,left=3cm,right=3cm,marginparwidth=1.75cm]{geometry}%table package
%Table
\usepackage{multirow,booktabs}
\usepackage{makecell}
%字体颜色
\usepackage{color}
\usepackage[dvipsnames]{xcolor}  % 更全的色系
%代码
\usepackage[OT1]{fontenc}
% MATLAB 代码风格
\usepackage[framed,numbered,autolinebreaks,useliterate]{/Users/anye_zhenhaoyu/Desktop/Latex/mcode}
\usepackage{listings}
\usepackage{algorithm}
\usepackage{algorithmic}
%插图
\usepackage{graphicx}
%改变item格式
\usepackage{enumerate}
%物理
\usepackage{physics}
%extra arrows
\usepackage{extarrows}
% caption(居中指令)
%\usepackage[justification=centering]{caption}
\usepackage{caption}
% htpb
\usepackage{stfloats}
% pdf 拼接
\usepackage{pdfpages}
% 超链接url
\usepackage{url}

\def\RR{\mathbb{R}}
\def\ZZ{\mathbb{Z}}
\def\EE{\mathbb{E}}

\def\Trsp#1{#1^{\mathcal{T}}}

\def\bw{\boldsymbol{\omega}}
\def\ba{\boldsymbol{a}}
\def\bb{\boldsymbol{b}}
\def\bc{\boldsymbol{c}}
\def\bd{\boldsymbol{d}}
\def\be{\boldsymbol{e}}
\def\bf{\boldsymbol{f}}
\def\bg{\boldsymbol{g}}
\def\bt{\boldsymbol{t}}
\def\bu{\boldsymbol{u}}
\def\bv{\boldsymbol{v}}
\def\bx{\boldsymbol{x}}
\def\by{\boldsymbol{y}}
\def\bz{\boldsymbol{z}}

\def\bA{\boldsymbol{A}}
\def\bB{\boldsymbol{B}}
\def\bC{\boldsymbol{C}}
\def\bE{\boldsymbol{E}}
\def\bO{\boldsymbol{O}}
\def\bP{\boldsymbol{P}}
\def\bQ{\boldsymbol{Q}}
\def\bX{\boldsymbol{X}}
\def\bY{\boldsymbol{Y}}

\def\Esolve{\textcolor{blue}{Solve: }}
\def\Eproof{\textcolor{blue}{Proof: }}

\def\suminf#1{\sum_{#1=-\infty}^{+\infty}}

\newtheorem{lemma}{Lemma}
\newtheorem{theorem}{Theorem}

%\graphicspath{{figures/}}


\begin{document}
\title{Homework 9}
\author{Zhen}
\maketitle

\section*{Problem 1}
$$
\begin{aligned}
	\min_{x_1,x_2}\ \ 
	&f(x_1,x_2)=\frac{1}{2}x_1^2+x_1x_2+x_2^2-x_1-3x_2
	\\
	\mbox{s.t.   }&x_1+2x_2=1
\end{aligned}
$$

\begin{enumerate}[(a)]
	\item 
		\[
		    \begin{aligned}
				f(x_1,x_2)=&
				\frac{1}{2}x_1^2+x_1x_2+x_2^2-x_1-3x_2
				\\=&
				\frac{1}{2}(1-2x_2)^2+(1-2x_2)x_2+x_2^2-(1-2x_2)-3x_2
				\\=&
				x_2^2-2x_2-\frac{3}{4}
				\\=&
				(x_2-1)^2-\frac{7}{4}
		    \end{aligned}
		\]
		Then we have $x^*_1=-1,x^*_2=1$.
	\item
		Lagrange equations are:
		\[
		    \begin{aligned}
		        &x^*_1+2x^*_2=1
				\\&
				x_1^*+x_2^*-1+\lambda^*=0
				\\&
				x_1^*+2x_2^*-3+2\lambda^*=0
		    \end{aligned}
		\]
		Thus $x_1^*=-1,x_2^*=1,\lambda^*=1$.
\end{enumerate}

\section*{Problem 2}

\[
    \begin{aligned}
		\min_{\bx}\ \ 
		&\frac{1}{2}\bx^T\bQ\bx+\bg^T\bx+c
		\\\mbox{s.t.   }&
		\bA\bx=\bb
    \end{aligned}
\]
\begin{enumerate}[(a)]
	\item
		The Lagrange condition is: ($\boldsymbol{\lambda}\in\RR^k$)
		\[
		    \begin{aligned}
		       &\bA\bx=\bb
			   \\&
			   \bQ\bx+\bg-\bA^T\boldsymbol{\lambda}=\boldsymbol{0}
		    \end{aligned}
		\]
	\item
		\begin{lemma}
			Let $\bQ\succ\bO\in\RR^{n\times n}$, $\bA\in\RR^{k\times n}$ and $\rank\bA=k$, then $\bA\bQ^{-1}\bA^T\succ\bO$.
		\end{lemma}

		\begin{proof}
			$\forall \bx\in\RR^k$:
		    \[
		        \begin{aligned}
					\bx^T\bA\bQ^{-1}\bA^T\bx
					=&
					(\bA^T\bx)^T\bQ^{-1}(\bA^T\bx)>0
		        \end{aligned}
		    \]
			Thus $\bA\bQ^{-1}\bA^T\succ\bO$.
		\end{proof}
		
		By Lagrange condition:
		$\bx+\bQ^{-1}\bg-\bQ^{-1}\bA^T\boldsymbol{\lambda}=0$.
		Then we have 
		$\bb+\bA\bQ^{-1}\bg-\bA\bQ^{-1}\bA^T\boldsymbol{\lambda}$, which means that 
		$\boldsymbol{\lambda}=(\bA\bQ^{-1}\bA^T)^{-1}\qty(\bb+\bA\bQ^{-1}\bg)$.
		
		Thus
		\begin{equation*}
			\bx=\bQ^{-1}\bA^T(\bA\bQ^{-1}\bA^T)^{-1}\qty(\bb+\bA\bQ^{-1}\bg)-\bQ^{-1}\bg
		\end{equation*}
		\newpage
	\item
		Pluging $\bQ=\bE,\bg=-\bx_0,c=\dfrac{1}{2}\bx_0^T\bx$ into (b):
		\[
		    \begin{aligned}
		        \bx&=\bA^T(\bA\bA^T)^{-1}\qty(\bb-\bA\bx_0)+\bx_0
				\\ &\xlongequal{\bx_0=\boldsymbol{0}}
				\bA^T(\bA\bA^T)^{-1}\bb
		    \end{aligned}
		\]
	\item
		\[
		    \begin{aligned}
		        \bx&=\bw(\bw^T\bw)^{-1}\qty(b-\bw^T\bx_0)+\bx_0
		    \end{aligned}
		\]
		\[
		    \begin{aligned}
				d=&\norm{\bx-\bx_0}
				\\=&
				\norm{\bw(\bw^T\bw)^{-1}\qty(b-\bw^T\bx_0)}
				\\=&
				\frac{\norm{\bw\qty(b-\bw^T\bx_0)}}{\norm{\bw^T\bw}}
				\\=&
				\frac{\norm{b-\bw^T\bx_0}}{\norm{\bw}}
		    \end{aligned}
		\]
\end{enumerate}

\section*{Problem 3}
Lagrange condition is :
\[
    \begin{aligned}
        &x_1^2+4x_2^2=1
		\\&
		x_1-8\lambda x_2=0
		\\&
		x_2-2\lambda x_1=0
    \end{aligned}
\]
Then we have $x_1^2=4x_2^2\Rightarrow x_1=\pm\frac{\sqrt{2}}{2},x_2=\pm\frac{\sqrt{2}}{4}$.\\
\\
Thus the minimum of $x_1x_2$ is $-\dfrac{1}{4}$ with 
$
\begin{cases}
    x_1=-\frac{\sqrt{2}}{2}\\
	x_2=\frac{\sqrt{2}}{4}
\end{cases} \mbox{or}\ \ \ 
\begin{cases}
    x_1=\frac{\sqrt{2}}{2}\\
	x_2=-\frac{\sqrt{2}}{4}
\end{cases}
$.


\section*{Problem 4}

\begin{enumerate}[(a)]
	\item
		The Lagrange condition is:
		\[
		    \begin{aligned}
				&\norm{\bx}^2_2=1
				\\&
				\bA\bx+2\lambda'\bx=0
		    \end{aligned}
		\]
		which means that $\bx^*$ is a eigenvector.
		Then ${\bx^*}^T\bA\bx^*={\bx^*}^T\lambda\bx^*=\lambda$.

		Thus $\min_{\bx}\bx^T\bA\bx=\min_i\lambda_i=\lambda_1$.
	\item
		\begin{enumerate}[i)]
			\item 
				Lagrange condition is: ($\exists \lambda_1,\lambda_2\in\RR$)
				\[
				    \begin{aligned}
				        &\norm{\bx}^2_2=1
						\\&
						\bv^T\bx=0
						\\&
						\bA\bx+2\lambda_1\bx+\lambda_2\bv_1=0
				    \end{aligned}
				\]
				Then $\exists c_0=-2\lambda_1,c_1=-\lambda_2$, $\bA\bx^*=c_0\bx^*+c_1\bv_1$.
			\item
				$\bv_1^T\bA\bx^*=c_0\bv^T\bx^*+c_1\bv_1^T\bv_1$ where $\mbox{LHS}=(\bA^T\bv_1)^T\bx^*=\lambda_1\bv_1^T\bx^*=0$ and $\mbox{RHS}=c_1\norm{\bv_1}^2$.
				Then we attain $c_1=0$.
			\item
				\begin{lemma}
					An eigenvector orthogonal to $\bv_1$ must be associated to one of the eigenvalues $\lambda_2,\dots,\lambda_n$.
				\end{lemma}
				\begin{proof}
					Suppose that $\bv_i$ is associated to $\lambda_i$ and $\bv_j$ is associated to $\lambda_j$.
					Then $\bv_i^T\bA\bv_j=\lambda_i\bv_i^T\bv_j=\lambda_j\bv_i^T\bv_j\Rightarrow\bv_i^T\bv_j=0$. Thus $\mbox{span}{(\bv_1,\dots,\bv_n)}=\RR^n$.
				\end{proof}

				Finally we have $\min_{\bv_1^T\bx=0}\bx^T\bA\bx=\min_{i=2,\dots,n}\lambda_i\norm{\bx^*}_2^2=\min_{i=2,\dots,n}\lambda_i=\lambda_2$. Like (a) we have $\bx^*$ is associated to $\lambda_2$.
		\end{enumerate}
\end{enumerate}


\end{document}


