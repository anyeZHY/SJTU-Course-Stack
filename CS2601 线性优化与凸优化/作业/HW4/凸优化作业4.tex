%! Tex program = xelatex
\documentclass{article}

\usepackage{amsmath,amssymb}
\usepackage{ntheorem}
\usepackage[letterpaper,top=2cm,bottom=2cm,left=3cm,right=3cm,marginparwidth=1.75cm]{geometry}%table package
%Table
\usepackage{multirow,booktabs}
\usepackage{makecell}
%代码
\usepackage{listings}
\usepackage{algorithm}
\usepackage{algorithmic}
%插图
\usepackage{graphicx}
%改变item格式
\usepackage{enumerate}
%字体颜色
\usepackage{color}
%物理
\usepackage{physics}
%extra arrows
\usepackage{extarrows}

\def\RR{\mathbb{R}}
\def\ZZ{\mathbb{Z}}
\def\EE{\mathbb{E}}
\def\bw{\boldsymbol{\omega}}
\def\bx{\boldsymbol{x}}
\def\by{\boldsymbol{y}}
\def\bb{\boldsymbol{b}}
\def\bd{\boldsymbol{d}}
\def\vvert{\vert\vert}
\def\bA{\boldsymbol{A}}
\def\bB{\boldsymbol{B}}
\def\bC{\boldsymbol{C}}
\def\bO{\boldsymbol{O}}

\def\Esolve{\textcolor{blue}{Solve: }}
\def\Eproof{\textcolor{blue}{Proof: }}

\newtheorem{lemma}{Lemma}
\newtheorem{proof}{Proof}
\newtheorem*{theorem}{Theorem}


\begin{document}
\title{Homework 4}
\author{Zhen}
\maketitle


\section*{problem 1}

$$
H(x)=\sum_{i=1}^n -x_i\log{x_i}
$$
\Esolve
\begin{enumerate}[(a)]
	\item 
		Apparently, $\log{x}$ is concave. By the assumption that $\norm{\bx}_0=k$ and the first $k$ components of $\bx$ are nonzero, we have
		$$
		\sum_{i=1}^nx_i(-\log{x_i})
		=
		\sum_{i=1}^kx_i\log{\frac{1}{x_i}}
		\le
		\log\qty({\sum_{i=1}^k\frac{1}{x_i}\times x_i})
		=
		\log{k}
		\le
		\log{n}
		$$

	\item
		$\forall i,x_i>0$, we have
		$$
		\pdv{f}{x_i}{x_j}=
		\begin{cases}
			0 &\mbox{ if }i\neq j \\
			-1/{x_i} &\mbox{ if } i = j
		\end{cases}
		$$
		which means that $H(x)$ is strictly concave (every eigenvalues of $\nabla^2 f$ are negative).

		Let $C \stackrel{def}{=} \qty{\bx\in\Delta_{n-1}:\bx>\boldsymbol{0}}$. Then we hold $\forall \bx\in C$, $H(\bx)\le\log n=H([1/n,\dots,1/n]^\mathcal{T})$ and $[1/n,\dots,1/n]^\mathcal{T}$ is the \textbf{unique} maximum.

		Also $\forall \bx\in \Delta_{n-1}/C$, $H(\bx)\le\log\norm{\bx}_2<\log n$.

		Thus $[1/n,\dots,1/n]^\mathcal{T}$ is the \textbf{unique} maximum of $H(\bx)$.
\end{enumerate}

\section*{problem 2}

\begin{enumerate}[(a)]
	\item 
		\Eproof
		$$
		\begin{aligned}
			&
			\frac{f(\mu)-f(s)}{\mu-s}
			\le
			\frac{f(u)-f(\mu)}{u-\mu}
			\\
			\iff&
			uf(\mu)-(u-\mu)f(s)\le(\mu-s)f(u)+sf(\mu)
			\\
			\iff&
			f(\mu)
			\le
			\frac{u-\mu}{u-s}f(s)+\frac{\mu-s}{u-s}f(u)
			\\
			\iff&
			f(\frac{u-\mu}{u-s}s+\frac{\mu-s}{u-s}u)
			\le
			\frac{u-\mu}{u-s}f(s)+\frac{\mu-s}{u-s}f(u)
			\\
			\Longleftarrow\ &
			f \mbox{ is convex.}
		\end{aligned}
		$$
	\item
		$\exists \beta\in \RR$ such that $f(x)\ge f(\mu)+\beta(x-\mu)$.

		\Eproof Let 
		$$
		\beta=\sup_{a<s<\mu}\frac{f(\mu)-f(s)}{\mu-s}
		$$
		Case 1 \textcolor{blue}{$\mu<x<b$:}

		By (a) we have $\forall a<s<\mu:$
		$$
		\frac{f(\mu)-f(s)}{\mu-s}
		<
		\frac{f(x)-f(\mu)}{x-\mu}
		$$
		Thus 
		$
		\beta=\sup_{a<s<\mu}\dfrac{f(\mu)-f(s)}{\mu-s}
		<
		\dfrac{f(x)-f(\mu)}{x-\mu}
		$, which means $\beta(x-\mu)+f(\mu)<f(x)$.

		Case 2 \textcolor{blue}{$a<x<\mu$:}

		Obviously, 
		$$
		\frac{f(\mu)-f(x)}{\mu-x}
		\le
		\sup_{a<s<\mu}\frac{f(\mu)-f(s)}{\mu-s}=\beta
		$$
		Thus  $\beta(x-\mu)+f(\mu)\le f(x)$.

		Case 3 \textcolor{blue}{$x=\mu$:} The proof is trivial.

	\item
		By (b), $\exists \beta\in\RR,\forall x\in(a,b):f(x)\ge f(\mu)+\beta(x-\mu)$. Then $f(X)\ge f(\mu)+\beta(X-\mu)$.

		Finally,
		$$
		\begin{aligned}
			\EE[f(X)]
			&=
			\int_a^b\Pr(X=x)f(X=x)\dd x
			\\&\ge
			\int_a^b\Pr(X=x)\qty[f(\mu)+\beta(x-\mu)]\dd x
			\\&=
			f(\mu)+\beta(\mu-\mu)
			\\&=
			f\qty(\EE[X])
		\end{aligned}
		$$
\end{enumerate}

\section*{problem 3}

$$
S=\qty{
	\bx=(x_1,x_2)\in\RR^2:
	\max\qty{\norm{\bA\bx+\bb}^3,\log(1+e^{3x_1+2x_2})}\le2
}
$$
\Esolve convex.

We have some \textcolor{blue}{true} propositions as follows:
\begin{enumerate}[1.]
	\item If $f:\RR^n\to\RR$ is convex, then $f(\bA\bx+\bb)$ is also convex ($\forall\bA\in\RR^{n\times m},\bx\in\RR^m,\bb\in\RR^n$).
	\item If $f:\RR^n\to\RR,g:\RR^n\to\RR$ are convex, then $h=\max\qty{f,g}$ is convex.
	\item If $f:\RR^n\to\RR$ is convex, then $S=\qty{\bx:f(\bx)\le Const}$ is convex.
\end{enumerate}

First we have: $\norm{\bx}^3$ and 
$\log\qty(1+e^x)$ are convex.

By proposition 1 we have 
$\norm{\bA\bx+\bb}^3$and $\log(1+e^{3x_1+2x_2})$ are convex.

Then by proposition 2, $\max\qty{\norm{\bA\bx+\bb}^3,\log(1+e^{3x_1+2x_2})}$ is convex.

Thus, by proposition 4, $S$ is convex.
\section*{problem 4}

\begin{enumerate}[(a)]
	\item 
		\Esolve
		It is a convex optimization problem.

		The objective function 
		$f(\bx)= 
		\bx^\mathcal{T}
		\mqty[1 & -1 \\ -1 & 1]
		\bx+\mqty[1 & 1]\bx
		$.
		By the fact that $\mqty[1 & -1 \\ -1 & 1]\succeq \bO$, $f(\bx)$ is convex.

		Inequality constraint function $g(\bx) = (x_1-x_2)^2 + 4x_1x_2 + e^{x_1+x_2}=(x_1+x_2)^2+e^{x_1+x_2}$. Also, $g$ is convex because $(x_1+x_2)^2$ and $e^{x_1+x_2}$ are convex.

		Obviously equality constraint function $h(x)=x_1-3x_2$ is a affine function.
	\item
		\Esolve
		It is \textbf{NOT} a convex optimization problem.

		The equality constraint function $h(x)=6x_1^2-7x_2$ is not a affine function.
\end{enumerate}

\end{document}

