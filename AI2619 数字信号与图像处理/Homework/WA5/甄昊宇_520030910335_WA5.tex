% ! Tex program = xelatex
\documentclass[fontset=macnew]{article}
\PassOptionsToPackage{quiet}{fontspec}% (or try silent)
% 中文
\usepackage[UTF8]{ctex}

% For more choices
% %! Tex program = xelatex
% \documentclass{article}
%中文
%\usepackage[UTF8]{ctex}
%数学公式
\usepackage{amsmath,amssymb}
%\usepackage{ntheorem}
% \usepackage[framemethod=TikZ]{mdframed}
\usepackage{amsthm}
%边界
\usepackage[letterpaper,top=2cm,bottom=3cm,left=2cm,right=2cm,marginparwidth=1.75cm]{geometry}%table package
%Table
\usepackage{multirow,booktabs}
\usepackage{makecell}
%字体颜色
\usepackage{color}
% \usepackage[dvipsnames]{xcolor}  % 更全的色系
%代码
\usepackage[OT1]{fontenc}
% MATLAB 代码风格
%\usepackage[framed,numbered,autolinebreaks,useliterate]{/Users/anye_zhenhaoyu/Desktop/Latex/mcode}
\usepackage{listings}
\usepackage{algorithm}
\usepackage{algorithmic}
\usepackage{pythonhighlight} % Python
%插图
\usepackage{graphicx}
%改变item格式
\usepackage{enumerate}
%物理
\usepackage{physics}
%extra arrows
\usepackage{extarrows}
% caption(居中指令)
%\usepackage[justification=centering]{caption}
\usepackage{caption}
% htpb
\usepackage{stfloats}
% pdf 拼接
\usepackage{pdfpages}
% 超链接url
\usepackage{url}
\usepackage{tikz}
\usepackage{pgfplots}
\pgfplotsset{compat=newest}
\usepackage[colorlinks=true, allcolors=red]{hyperref}
\usepackage{setspace}
\usepackage{bbm}

% --------------definations-------------- %
\def\*#1{\boldsymbol{#1}}
\def\+#1{\mathcal{#1}} 
\def\-#1{\mathrm{#1}}
\def\rm#1{\mathrm{#1}}
\def\=#1{\mathbb{#1}}
% Domains
\def\RR{\mathbb{R}}
\def\EE{\mathbb{E}}
\def\CC{\mathbb{C}}
\def\NN{\mathbb{N}}
\def\ZZ{\mathbb{Z}}
% Newcommand
\newcommand{\inner}[2]{\langle #1,#2\rangle} 
\newcommand{\numP}{\#\mathbf{P}} 
\renewcommand{\P}{\mathbf{P}}
\newcommand{\Var}[2][]{\mathbf{Var}_{#1}\left[#2\right]}
\newcommand{\E}[2][]{\mathbf{E}_{#1}\left[#2\right]}
\renewcommand{\emptyset}{\varnothing}
\newcommand{\ol}{\overline}
\newcommand{\argmin}{\mathop{\arg\min}}
\newcommand{\argmax}{\mathop{\arg\max}}
\renewcommand{\abs}[1]{\qty|#1|}
\newcommand{\defeq}{\triangleq} % triangle over =
\def\deq{\xlongequal{def}} % 'def' over =
\def\LHS{\text{LHS}}
\def\RHS{\text{RHS}}
\def\angbr#1{\langle#1\rangle} % <x>
\def\set#1{\qty{#1}}

\def\Esolve{\textcolor{blue}{Solve: }}
\def\Eproof{\textcolor{blue}{Proof: }}
\def\case#1{\textcolor{blue}{Case \uppercase\expandafter{\romannumeral#1}: }}

% \newmdtheoremenv{lemma}{Lemma}
% \newmdtheoremenv{theorem}{\textcolor{red}{Theorem}}
% \newmdtheoremenv{defi}{\textcolor{blue}{Definition}}
\newtheorem{lemma}{Lemma}
\newtheorem{thm}{Theorem}
\newtheorem{defi}{Definition}
\newtheorem{prp}{Proposition}
\newtheorem{remark}{Remark}
\newenvironment{md}{\begin{mdframed}}{\end{mdframed}}

\graphicspath{{figures/}}

% \begin{document}
% \title{<++>}
\author{Haoyu Zhen}
% \maketitle
\setlength{\parindent}{0pt}
\setstretch{1.2}
% \end{document}

\usepackage{fancyhdr}
\pagestyle{fancy}
% \fancypagestyle{mainFancy}{
%     \fancyhf{}
%     \renewcommand\headrulewidth{.5pt}       % 页眉横线
%     \renewcommand\footrulewidth{0pt}
%     \fancyhead[OC]{\fzkai{\leftmark}}       % 页眉章标题
%     \fancyhead[EC]{\fzkai{\@title}}         % 页眉文章题目
% 	\lhead{\fzkai{author}}
%     \fancyhead[OR,EL]{\thepage}                 % 页眉编号
%     \fancyfoot[r]{\thumb} % 将拇指放到没有被使用的页眉或页脚处
% }
\fancyhf{}
\fancyfoot[C]{\thepage}
\fancyhead[R]{\slshape{AI2615 Algorithm Design and Analysis}}
\fancyhead[L]{\slshape{Haoyu Zhen}}

% % theorems
\usepackage{thmtools}
\usepackage{thm-restate}
\usepackage[framemethod=TikZ]{mdframed}
\mdfsetup{skipabove=1em,skipbelow=0em, innertopmargin=12pt, innerbottommargin=8pt}

\theoremstyle{definition}

\declaretheoremstyle[headfont=\bfseries\sffamily, bodyfont=\normalfont,
	mdframed={
		nobreak,
		backgroundcolor=brown!14,
		topline=false,
		rightline=false,
		leftline=true,
		bottomline=false,
		linewidth=2pt,
		linecolor=brown!180,
	}
]{thmbrownbox}

\declaretheoremstyle[headfont=\bfseries\sffamily, bodyfont=\normalfont,
	mdframed={
		nobreak,
		backgroundcolor=Blue!4,
		topline=false,
		rightline=false,
		leftline=true,
		bottomline=false,
		linewidth=2pt,
		linecolor=NavyBlue!120,
	}
]{thmbluebox}

\declaretheoremstyle[headfont=\bfseries\sffamily, bodyfont=\normalfont,
	mdframed={
		nobreak,
		backgroundcolor=Green!5,
		topline=false,
		rightline=false,
		leftline=true,
		bottomline=false,
		linewidth=2pt,
		linecolor=OliveGreen!120,
	}
]{thmgreenbox}

\declaretheoremstyle[headfont=\bfseries\sffamily, bodyfont=\normalfont,
	mdframed={
		nobreak,
		topline=false,
		rightline=false,
		leftline=true,
		bottomline=false,
		linewidth=2pt,
		linecolor=OliveGreen!120,
	}
]{thmgreenline}

\declaretheoremstyle[headfont=\bfseries\sffamily, bodyfont=\normalfont,
	mdframed={
		nobreak,
		topline=false,
		rightline=false,
		leftline=true,
		bottomline=false,
		linewidth=2pt,
		linecolor=NavyBlue!70,
	}
]{thmblueline}

\declaretheorem[numberwithin=section, style=thmbrownbox, name={\color{Brown}Definition}]{defi}
\declaretheorem[numberwithin=section, style=thmgreenbox, name={\color{OliveGreen}Law}]{law}
\declaretheorem[numberwithin=section, style=thmbluebox, name={\color{Blue}Corollary}]{cor}
\declaretheorem[numberwithin=section, style=thmgreenline, name={\color{OliveGreen}Property}]{prt}
\declaretheorem[numberwithin=section, style=thmbluebox, name={\color{Blue}Proposition}]{prp}
\declaretheorem[numberwithin=section, style=thmbluebox, name={\color{Blue}Theorem}]{thm}
\declaretheorem[numberwithin=section, style=thmbluebox, name={\color{Blue}Lemma}]{lemma}
\declaretheorem[numberwithin=section, style=thmbrownbox,  name={\color{Brown}Example}]{eg}
\declaretheorem[numberwithin=section, style=thmgreenline, name={\color{OliveGreen}Remark}]{remark}
\declaretheorem[numbered=no,style=thmblueline, name={\color{NavyBlue!70}Proof},qed=$\square$]{prf}
\numberwithin{equation}{section}


% %! Tex program = xelatex
% \documentclass{article}
%中文
%\usepackage[UTF8]{ctex}
%数学公式
\usepackage{amsmath,amssymb}
%\usepackage{ntheorem}
% \usepackage[framemethod=TikZ]{mdframed}
\usepackage{amsthm}
%边界
\usepackage[letterpaper,top=2cm,bottom=3cm,left=2cm,right=2cm,marginparwidth=1.75cm]{geometry}%table package
%Table
\usepackage{multirow,booktabs}
\usepackage{makecell}
%字体颜色
\usepackage{color}
% \usepackage[dvipsnames]{xcolor}  % 更全的色系
%代码
\usepackage[OT1]{fontenc}
% MATLAB 代码风格
%\usepackage[framed,numbered,autolinebreaks,useliterate]{/Users/anye_zhenhaoyu/Desktop/Latex/mcode}
\usepackage{listings}
\usepackage{algorithm}
\usepackage{algorithmic}
\usepackage{pythonhighlight} % Python
%插图
\usepackage{graphicx}
%改变item格式
\usepackage{enumerate}
%物理
\usepackage{physics}
%extra arrows
\usepackage{extarrows}
% caption(居中指令)
%\usepackage[justification=centering]{caption}
\usepackage{caption}
% htpb
\usepackage{stfloats}
% pdf 拼接
\usepackage{pdfpages}
% 超链接url
\usepackage{url}
\usepackage{tikz}
\usepackage{pgfplots}
\pgfplotsset{compat=newest}
\usepackage[colorlinks=true, allcolors=red]{hyperref}
\usepackage{setspace}
\usepackage{bbm}

% --------------definations-------------- %
\def\*#1{\boldsymbol{#1}}
\def\+#1{\mathcal{#1}} 
\def\-#1{\mathrm{#1}}
\def\rm#1{\mathrm{#1}}
\def\=#1{\mathbb{#1}}
% Domains
\def\RR{\mathbb{R}}
\def\EE{\mathbb{E}}
\def\CC{\mathbb{C}}
\def\NN{\mathbb{N}}
\def\ZZ{\mathbb{Z}}
% Newcommand
\newcommand{\inner}[2]{\langle #1,#2\rangle} 
\newcommand{\numP}{\#\mathbf{P}} 
\renewcommand{\P}{\mathbf{P}}
\newcommand{\Var}[2][]{\mathbf{Var}_{#1}\left[#2\right]}
\newcommand{\E}[2][]{\mathbf{E}_{#1}\left[#2\right]}
\renewcommand{\emptyset}{\varnothing}
\newcommand{\ol}{\overline}
\newcommand{\argmin}{\mathop{\arg\min}}
\newcommand{\argmax}{\mathop{\arg\max}}
\renewcommand{\abs}[1]{\qty|#1|}
\newcommand{\defeq}{\triangleq} % triangle over =
\def\deq{\xlongequal{def}} % 'def' over =
\def\LHS{\text{LHS}}
\def\RHS{\text{RHS}}
\def\angbr#1{\langle#1\rangle} % <x>
\def\set#1{\qty{#1}}

\def\Esolve{\textcolor{blue}{Solve: }}
\def\Eproof{\textcolor{blue}{Proof: }}
\def\case#1{\textcolor{blue}{Case \uppercase\expandafter{\romannumeral#1}: }}

% \newmdtheoremenv{lemma}{Lemma}
% \newmdtheoremenv{theorem}{\textcolor{red}{Theorem}}
% \newmdtheoremenv{defi}{\textcolor{blue}{Definition}}
\newtheorem{lemma}{Lemma}
\newtheorem{thm}{Theorem}
\newtheorem{defi}{Definition}
\newtheorem{prp}{Proposition}
\newtheorem{remark}{Remark}
\newenvironment{md}{\begin{mdframed}}{\end{mdframed}}

\graphicspath{{figures/}}

% \begin{document}
% \title{<++>}
\author{Haoyu Zhen}
% \maketitle
\setlength{\parindent}{0pt}
\setstretch{1.2}
% \end{document}

\usepackage{fancyhdr}
\pagestyle{fancy}
% \fancypagestyle{mainFancy}{
%     \fancyhf{}
%     \renewcommand\headrulewidth{.5pt}       % 页眉横线
%     \renewcommand\footrulewidth{0pt}
%     \fancyhead[OC]{\fzkai{\leftmark}}       % 页眉章标题
%     \fancyhead[EC]{\fzkai{\@title}}         % 页眉文章题目
% 	\lhead{\fzkai{author}}
%     \fancyhead[OR,EL]{\thepage}                 % 页眉编号
%     \fancyfoot[r]{\thumb} % 将拇指放到没有被使用的页眉或页脚处
% }
\fancyhf{}
\fancyfoot[C]{\thepage}
\fancyhead[R]{\slshape{AI2615 Algorithm Design and Analysis}}
\fancyhead[L]{\slshape{Haoyu Zhen}}

% % theorems
\usepackage{thmtools}
\usepackage{thm-restate}
\usepackage[framemethod=TikZ]{mdframed}
\mdfsetup{skipabove=1em,skipbelow=0em, innertopmargin=12pt, innerbottommargin=8pt}

\theoremstyle{definition}

\declaretheoremstyle[headfont=\bfseries\sffamily, bodyfont=\normalfont,
	mdframed={
		nobreak,
		backgroundcolor=brown!14,
		topline=false,
		rightline=false,
		leftline=true,
		bottomline=false,
		linewidth=2pt,
		linecolor=brown!180,
	}
]{thmbrownbox}

\declaretheoremstyle[headfont=\bfseries\sffamily, bodyfont=\normalfont,
	mdframed={
		nobreak,
		backgroundcolor=Blue!4,
		topline=false,
		rightline=false,
		leftline=true,
		bottomline=false,
		linewidth=2pt,
		linecolor=NavyBlue!120,
	}
]{thmbluebox}

\declaretheoremstyle[headfont=\bfseries\sffamily, bodyfont=\normalfont,
	mdframed={
		nobreak,
		backgroundcolor=Green!5,
		topline=false,
		rightline=false,
		leftline=true,
		bottomline=false,
		linewidth=2pt,
		linecolor=OliveGreen!120,
	}
]{thmgreenbox}

\declaretheoremstyle[headfont=\bfseries\sffamily, bodyfont=\normalfont,
	mdframed={
		nobreak,
		topline=false,
		rightline=false,
		leftline=true,
		bottomline=false,
		linewidth=2pt,
		linecolor=OliveGreen!120,
	}
]{thmgreenline}

\declaretheoremstyle[headfont=\bfseries\sffamily, bodyfont=\normalfont,
	mdframed={
		nobreak,
		topline=false,
		rightline=false,
		leftline=true,
		bottomline=false,
		linewidth=2pt,
		linecolor=NavyBlue!70,
	}
]{thmblueline}

\declaretheorem[numberwithin=section, style=thmbrownbox, name={\color{Brown}Definition}]{defi}
\declaretheorem[numberwithin=section, style=thmgreenbox, name={\color{OliveGreen}Law}]{law}
\declaretheorem[numberwithin=section, style=thmbluebox, name={\color{Blue}Corollary}]{cor}
\declaretheorem[numberwithin=section, style=thmgreenline, name={\color{OliveGreen}Property}]{prt}
\declaretheorem[numberwithin=section, style=thmbluebox, name={\color{Blue}Proposition}]{prp}
\declaretheorem[numberwithin=section, style=thmbluebox, name={\color{Blue}Theorem}]{thm}
\declaretheorem[numberwithin=section, style=thmbluebox, name={\color{Blue}Lemma}]{lemma}
\declaretheorem[numberwithin=section, style=thmbrownbox,  name={\color{Brown}Example}]{eg}
\declaretheorem[numberwithin=section, style=thmgreenline, name={\color{OliveGreen}Remark}]{remark}
\declaretheorem[numbered=no,style=thmblueline, name={\color{NavyBlue!70}Proof},qed=$\square$]{prf}
\numberwithin{equation}{section}


% On my MAC's Desktop
%! Tex program = xelatex
% \documentclass{article}
%中文
%\usepackage[UTF8]{ctex}
%数学公式
\usepackage{amsmath,amssymb}
%\usepackage{ntheorem}
% \usepackage[framemethod=TikZ]{mdframed}
\usepackage{amsthm}
%边界
\usepackage[letterpaper,top=2cm,bottom=3cm,left=2cm,right=2cm,marginparwidth=1.75cm]{geometry}%table package
%Table
\usepackage{multirow,booktabs}
\usepackage{makecell}
%字体颜色
\usepackage{color}
% \usepackage[dvipsnames]{xcolor}  % 更全的色系
%代码
\usepackage[OT1]{fontenc}
% MATLAB 代码风格
%\usepackage[framed,numbered,autolinebreaks,useliterate]{/Users/anye_zhenhaoyu/Desktop/Latex/mcode}
\usepackage{listings}
\usepackage{algorithm}
\usepackage{algorithmic}
\usepackage{pythonhighlight} % Python
%插图
\usepackage{graphicx}
%改变item格式
\usepackage{enumerate}
%物理
\usepackage{physics}
%extra arrows
\usepackage{extarrows}
% caption(居中指令)
%\usepackage[justification=centering]{caption}
\usepackage{caption}
% htpb
\usepackage{stfloats}
% pdf 拼接
\usepackage{pdfpages}
% 超链接url
\usepackage{url}
\usepackage{tikz}
\usepackage{pgfplots}
\pgfplotsset{compat=newest}
\usepackage[colorlinks=true, allcolors=red]{hyperref}
\usepackage{setspace}
\usepackage{bbm}

% --------------definations-------------- %
\def\*#1{\boldsymbol{#1}}
\def\+#1{\mathcal{#1}} 
\def\-#1{\mathrm{#1}}
\def\rm#1{\mathrm{#1}}
\def\=#1{\mathbb{#1}}
% Domains
\def\RR{\mathbb{R}}
\def\EE{\mathbb{E}}
\def\CC{\mathbb{C}}
\def\NN{\mathbb{N}}
\def\ZZ{\mathbb{Z}}
% Newcommand
\newcommand{\inner}[2]{\langle #1,#2\rangle} 
\newcommand{\numP}{\#\mathbf{P}} 
\renewcommand{\P}{\mathbf{P}}
\newcommand{\Var}[2][]{\mathbf{Var}_{#1}\left[#2\right]}
\newcommand{\E}[2][]{\mathbf{E}_{#1}\left[#2\right]}
\renewcommand{\emptyset}{\varnothing}
\newcommand{\ol}{\overline}
\newcommand{\argmin}{\mathop{\arg\min}}
\newcommand{\argmax}{\mathop{\arg\max}}
\renewcommand{\abs}[1]{\qty|#1|}
\newcommand{\defeq}{\triangleq} % triangle over =
\def\deq{\xlongequal{def}} % 'def' over =
\def\LHS{\text{LHS}}
\def\RHS{\text{RHS}}
\def\angbr#1{\langle#1\rangle} % <x>
\def\set#1{\qty{#1}}

\def\Esolve{\textcolor{blue}{Solve: }}
\def\Eproof{\textcolor{blue}{Proof: }}
\def\case#1{\textcolor{blue}{Case \uppercase\expandafter{\romannumeral#1}: }}

% \newmdtheoremenv{lemma}{Lemma}
% \newmdtheoremenv{theorem}{\textcolor{red}{Theorem}}
% \newmdtheoremenv{defi}{\textcolor{blue}{Definition}}
\newtheorem{lemma}{Lemma}
\newtheorem{thm}{Theorem}
\newtheorem{defi}{Definition}
\newtheorem{prp}{Proposition}
\newtheorem{remark}{Remark}
\newenvironment{md}{\begin{mdframed}}{\end{mdframed}}

\graphicspath{{figures/}}

% \begin{document}
% \title{<++>}
\author{Haoyu Zhen}
% \maketitle
\setlength{\parindent}{0pt}
\setstretch{1.2}
% \end{document}

\usepackage{fancyhdr}
\pagestyle{fancy}
% \fancypagestyle{mainFancy}{
%     \fancyhf{}
%     \renewcommand\headrulewidth{.5pt}       % 页眉横线
%     \renewcommand\footrulewidth{0pt}
%     \fancyhead[OC]{\fzkai{\leftmark}}       % 页眉章标题
%     \fancyhead[EC]{\fzkai{\@title}}         % 页眉文章题目
% 	\lhead{\fzkai{author}}
%     \fancyhead[OR,EL]{\thepage}                 % 页眉编号
%     \fancyfoot[r]{\thumb} % 将拇指放到没有被使用的页眉或页脚处
% }
\fancyhf{}
\fancyfoot[C]{\thepage}
\fancyhead[R]{\slshape{AI2615 Algorithm Design and Analysis}}
\fancyhead[L]{\slshape{Haoyu Zhen}}


\graphicspath{{figures/}}
\setlength{\headheight}{14pt}
\fancyhead[C]{\large\textbf{Written Assignment 5}}
\begin{document}
% \tableofcontents
% \title{}
% \maketitle
\section*{Question 1}
依定义我们有:
\[
    p_z\dd z = p_r\dd r \qor \int_0^z p_z\dd z = \int_0^r p_r\dd r
\] 其中 $p_r=2-2r$ 且 $p_z=4z$ 如果 $z<0.5$ 否则 $p_z=4-4z$。
因此
\[
    \begin{cases}
		-r^2+2r=2z^2 & \text{if }z<0.5  \\
		(r-1)^2=2(z-1)^2 & \text{otherwise} \\
    \end{cases}
\] 于是有
\[
    \begin{cases}
		z=\sqrt{\dfrac{-r^2+2r}{2}} & \text{if }r<1-\dfrac{\sqrt{2}}{2}\\[9pt]
		z=\dfrac{\sqrt{2}}{2}(r-1)+1 & \text{if }1-\dfrac{\sqrt{2}}{2}\le r\le 1 \\
    \end{cases}
.\] 

\section*{Question 2}
\subsection*{(a \& b)}
根据定义,我们可以得到:
\[
    \begin{aligned}
		&\sum_x\sum_y\sum_s\sum_t w(s,t)f(x+s,y+t)
		\\=&
		\sum_s\sum_t\sum_x\sum_y w(s,t)f(x+s,y+t)
		\\=&
		\sum_s\sum_t w(s,t) \sum_x\sum_y f(x+s,y+t)
		\\=&
		\sum_s\sum_t w(s,t) \sum_x\sum_y f(x,y)
		\\=&
		\sum_s\sum_t w(s,t)\times \+I
		\ \ \ \ \ = 0
    \end{aligned}
\] 其中第四行等式成立的原因是我们对图片进行了补零操作。同时由于零点对称性,我们可以直接用$\sum_s$或 $\sum_t$代替$\sum_{s=\pm a}^{\mp a}$ 或$\sum_{t=\pm a}^{\mp a}$。故该题两小问结果皆为0。

\section*{Question 3}
Proof of $f(x,y)*h(x,y)\iff F(u,v)H(u,v)$:
\[
	\begin{aligned}
		&\int_\+X\int_\+Y\int_\+S\int_\+T
		f(s,t)h(x-s,y-t)\exp\big[-j2\pi(ux+vy)\big]
		\dd{t}\dd{s}\dd{y}\dd{x}
		\\[6pt]=&
		\int_\+S\int_\+T
		f(s,t)
		\int_\+X\int_\+Y
		h(x-s,y-t)\exp\big[-j2\pi(ux+vy)\big]
		\dd{y}\dd{x}\dd{t}\dd{s}
		\\[6pt]=&
		\int_\+S\int_\+T
		f(s,t)H(u,v)\exp\big[-j2\pi(us+vt)\big]
		\dd{t}\dd{s}
		\\[6pt]=&
		F(u,v)H(u,v)
	\end{aligned}
\] 其中第三行应用了“时域平移性质”。

Proof of $f(x,y)*h(x,y)\iff F(u,v)H(u,v)$:
\[
	\begin{aligned}
		&\int_\+U\int_\+V\int_\+S\int_\+T
		\frac{1}{4\pi^2}
		F(s,t)H(u-s,v-t)\exp\big[j2\pi(ux+vy)\big]
		\dd{t}\dd{s}\dd{v}\dd{u}
		\\[6pt]=&
		\int_\+S\int_\+T
		\frac{1}{4\pi^2}
		F(s,t)
		\int_\+U\int_\+V
		H(u-s,v-t)\exp\big[j2\pi(ux+vy)\big]
		\dd{v}\dd{u}\dd{t}\dd{s}
		\\[6pt]=&
		\int_\+S\int_\+T
		\frac{1}{2\pi}
		F(s,t)h(x,y)\exp\big[j2\pi(xs+yt)\big]
		\dd{t}\dd{s}
		\\[6pt]=&
		f(x,y)h(x,y)
	\end{aligned}
\]类似的,我使用了频移性质。

\section*{Question 4}
\subsection*{(a)}
结果为:
\[
	\mqty{
	0 & \flatfrac{1}{4} & 0 & 0 & 0 & 0 & 0 \\
	\flatfrac{1}{4} & \flatfrac{1}{4} & \flatfrac{1}{2} & 0 & 0 & 0 & 0 \\
	\flatfrac{1}{4} & \flatfrac{1}{2} & \flatfrac{1}{2} & \flatfrac{1}{2} & 0 & 0 & 0 \\
	0 & \flatfrac{1}{2} & \flatfrac{1}{2} & \flatfrac{1}{2} & \flatfrac{1}{2} & \flatfrac{1}{4} & 0 \\
	0 & 0 & \flatfrac{3}{4} & \flatfrac{3}{4} & \flatfrac{1}{2} & \flatfrac{1}{4} & \flatfrac{1}{4}\\
	0 & \flatfrac{1}{2} & \flatfrac{1}{2} & \flatfrac{1}{2} & \flatfrac{1}{2} & \flatfrac{1}{4} & 0 \\
	\flatfrac{1}{4} & \flatfrac{1}{2} & \flatfrac{1}{2} & \flatfrac{1}{2} & 0 & 0 & 0 \\
	\flatfrac{1}{4} & \flatfrac{1}{4} & \flatfrac{1}{2} & 0 & 0 & 0 & 0 \\
	0 & \flatfrac{1}{4} & 0 & 0 & 0 & 0 & 0 \\
}
\] 
这里我设置了padding=2以保留全部信息。

\subsection*{(b)}
容易得到:
\[
	h=\mqty{
		0 & \flatfrac{1}{4} & 0 \\
		\flatfrac{1}{4} & 0 & \flatfrac{1}{4}\\
		0 & \flatfrac{1}{4} & 0 
	}
	\qand
	H(u,v)=\frac{1}{2}\cos(2\pi\frac{u}{7})+\frac{1}{2}\cos(2\pi\frac{v}{9})
\]
\subsection*{(c)}
显然$H$是一个“低通滤波器”,因为原点处幅值响应为1;此外,中频、高频部分幅值响应小于一,起抑制效果。

\end{document}

