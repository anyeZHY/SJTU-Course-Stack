% ! Tex program = xelatex
\documentclass{article}
\newcommand*{\circled}[1]{\lower.7ex\hbox{\tikz\draw (0pt, 0pt)circle (.5em) node {\makebox[1em][c]{\small #1}};}}
% 中文
% \usepackage[UTF8]{ctex}

% For more choices
% %! Tex program = xelatex
% \documentclass{article}
%中文
%\usepackage[UTF8]{ctex}
%数学公式
\usepackage{amsmath,amssymb}
%\usepackage{ntheorem}
% \usepackage[framemethod=TikZ]{mdframed}
\usepackage{amsthm}
%边界
\usepackage[letterpaper,top=2cm,bottom=3cm,left=2cm,right=2cm,marginparwidth=1.75cm]{geometry}%table package
%Table
\usepackage{multirow,booktabs}
\usepackage{makecell}
%字体颜色
\usepackage{color}
% \usepackage[dvipsnames]{xcolor}  % 更全的色系
%代码
\usepackage[OT1]{fontenc}
% MATLAB 代码风格
%\usepackage[framed,numbered,autolinebreaks,useliterate]{/Users/anye_zhenhaoyu/Desktop/Latex/mcode}
\usepackage{listings}
\usepackage{algorithm}
\usepackage{algorithmic}
\usepackage{pythonhighlight} % Python
%插图
\usepackage{graphicx}
%改变item格式
\usepackage{enumerate}
%物理
\usepackage{physics}
%extra arrows
\usepackage{extarrows}
% caption(居中指令)
%\usepackage[justification=centering]{caption}
\usepackage{caption}
% htpb
\usepackage{stfloats}
% pdf 拼接
\usepackage{pdfpages}
% 超链接url
\usepackage{url}
\usepackage{tikz}
\usepackage{pgfplots}
\pgfplotsset{compat=newest}
\usepackage[colorlinks=true, allcolors=red]{hyperref}
\usepackage{setspace}
\usepackage{bbm}

% --------------definations-------------- %
\def\*#1{\boldsymbol{#1}}
\def\+#1{\mathcal{#1}} 
\def\-#1{\mathrm{#1}}
\def\rm#1{\mathrm{#1}}
\def\=#1{\mathbb{#1}}
% Domains
\def\RR{\mathbb{R}}
\def\EE{\mathbb{E}}
\def\CC{\mathbb{C}}
\def\NN{\mathbb{N}}
\def\ZZ{\mathbb{Z}}
% Newcommand
\newcommand{\inner}[2]{\langle #1,#2\rangle} 
\newcommand{\numP}{\#\mathbf{P}} 
\renewcommand{\P}{\mathbf{P}}
\newcommand{\Var}[2][]{\mathbf{Var}_{#1}\left[#2\right]}
\newcommand{\E}[2][]{\mathbf{E}_{#1}\left[#2\right]}
\renewcommand{\emptyset}{\varnothing}
\newcommand{\ol}{\overline}
\newcommand{\argmin}{\mathop{\arg\min}}
\newcommand{\argmax}{\mathop{\arg\max}}
\renewcommand{\abs}[1]{\qty|#1|}
\newcommand{\defeq}{\triangleq} % triangle over =
\def\deq{\xlongequal{def}} % 'def' over =
\def\LHS{\text{LHS}}
\def\RHS{\text{RHS}}
\def\angbr#1{\langle#1\rangle} % <x>
\def\set#1{\qty{#1}}

\def\Esolve{\textcolor{blue}{Solve: }}
\def\Eproof{\textcolor{blue}{Proof: }}
\def\case#1{\textcolor{blue}{Case \uppercase\expandafter{\romannumeral#1}: }}

% \newmdtheoremenv{lemma}{Lemma}
% \newmdtheoremenv{theorem}{\textcolor{red}{Theorem}}
% \newmdtheoremenv{defi}{\textcolor{blue}{Definition}}
\newtheorem{lemma}{Lemma}
\newtheorem{thm}{Theorem}
\newtheorem{defi}{Definition}
\newtheorem{prp}{Proposition}
\newtheorem{remark}{Remark}
\newenvironment{md}{\begin{mdframed}}{\end{mdframed}}

\graphicspath{{figures/}}

% \begin{document}
% \title{<++>}
\author{Haoyu Zhen}
% \maketitle
\setlength{\parindent}{0pt}
\setstretch{1.2}
% \end{document}

\usepackage{fancyhdr}
\pagestyle{fancy}
% \fancypagestyle{mainFancy}{
%     \fancyhf{}
%     \renewcommand\headrulewidth{.5pt}       % 页眉横线
%     \renewcommand\footrulewidth{0pt}
%     \fancyhead[OC]{\fzkai{\leftmark}}       % 页眉章标题
%     \fancyhead[EC]{\fzkai{\@title}}         % 页眉文章题目
% 	\lhead{\fzkai{author}}
%     \fancyhead[OR,EL]{\thepage}                 % 页眉编号
%     \fancyfoot[r]{\thumb} % 将拇指放到没有被使用的页眉或页脚处
% }
\fancyhf{}
\fancyfoot[C]{\thepage}
\fancyhead[R]{\slshape{AI2615 Algorithm Design and Analysis}}
\fancyhead[L]{\slshape{Haoyu Zhen}}

% % theorems
\usepackage{thmtools}
\usepackage{thm-restate}
\usepackage[framemethod=TikZ]{mdframed}
\mdfsetup{skipabove=1em,skipbelow=0em, innertopmargin=12pt, innerbottommargin=8pt}

\theoremstyle{definition}

\declaretheoremstyle[headfont=\bfseries\sffamily, bodyfont=\normalfont,
	mdframed={
		nobreak,
		backgroundcolor=brown!14,
		topline=false,
		rightline=false,
		leftline=true,
		bottomline=false,
		linewidth=2pt,
		linecolor=brown!180,
	}
]{thmbrownbox}

\declaretheoremstyle[headfont=\bfseries\sffamily, bodyfont=\normalfont,
	mdframed={
		nobreak,
		backgroundcolor=Blue!4,
		topline=false,
		rightline=false,
		leftline=true,
		bottomline=false,
		linewidth=2pt,
		linecolor=NavyBlue!120,
	}
]{thmbluebox}

\declaretheoremstyle[headfont=\bfseries\sffamily, bodyfont=\normalfont,
	mdframed={
		nobreak,
		backgroundcolor=Green!5,
		topline=false,
		rightline=false,
		leftline=true,
		bottomline=false,
		linewidth=2pt,
		linecolor=OliveGreen!120,
	}
]{thmgreenbox}

\declaretheoremstyle[headfont=\bfseries\sffamily, bodyfont=\normalfont,
	mdframed={
		nobreak,
		topline=false,
		rightline=false,
		leftline=true,
		bottomline=false,
		linewidth=2pt,
		linecolor=OliveGreen!120,
	}
]{thmgreenline}

\declaretheoremstyle[headfont=\bfseries\sffamily, bodyfont=\normalfont,
	mdframed={
		nobreak,
		topline=false,
		rightline=false,
		leftline=true,
		bottomline=false,
		linewidth=2pt,
		linecolor=NavyBlue!70,
	}
]{thmblueline}

\declaretheorem[numberwithin=section, style=thmbrownbox, name={\color{Brown}Definition}]{defi}
\declaretheorem[numberwithin=section, style=thmgreenbox, name={\color{OliveGreen}Law}]{law}
\declaretheorem[numberwithin=section, style=thmbluebox, name={\color{Blue}Corollary}]{cor}
\declaretheorem[numberwithin=section, style=thmgreenline, name={\color{OliveGreen}Property}]{prt}
\declaretheorem[numberwithin=section, style=thmbluebox, name={\color{Blue}Proposition}]{prp}
\declaretheorem[numberwithin=section, style=thmbluebox, name={\color{Blue}Theorem}]{thm}
\declaretheorem[numberwithin=section, style=thmbluebox, name={\color{Blue}Lemma}]{lemma}
\declaretheorem[numberwithin=section, style=thmbrownbox,  name={\color{Brown}Example}]{eg}
\declaretheorem[numberwithin=section, style=thmgreenline, name={\color{OliveGreen}Remark}]{remark}
\declaretheorem[numbered=no,style=thmblueline, name={\color{NavyBlue!70}Proof},qed=$\square$]{prf}
\numberwithin{equation}{section}


% %! Tex program = xelatex
% \documentclass{article}
%中文
%\usepackage[UTF8]{ctex}
%数学公式
\usepackage{amsmath,amssymb}
%\usepackage{ntheorem}
% \usepackage[framemethod=TikZ]{mdframed}
\usepackage{amsthm}
%边界
\usepackage[letterpaper,top=2cm,bottom=3cm,left=2cm,right=2cm,marginparwidth=1.75cm]{geometry}%table package
%Table
\usepackage{multirow,booktabs}
\usepackage{makecell}
%字体颜色
\usepackage{color}
% \usepackage[dvipsnames]{xcolor}  % 更全的色系
%代码
\usepackage[OT1]{fontenc}
% MATLAB 代码风格
%\usepackage[framed,numbered,autolinebreaks,useliterate]{/Users/anye_zhenhaoyu/Desktop/Latex/mcode}
\usepackage{listings}
\usepackage{algorithm}
\usepackage{algorithmic}
\usepackage{pythonhighlight} % Python
%插图
\usepackage{graphicx}
%改变item格式
\usepackage{enumerate}
%物理
\usepackage{physics}
%extra arrows
\usepackage{extarrows}
% caption(居中指令)
%\usepackage[justification=centering]{caption}
\usepackage{caption}
% htpb
\usepackage{stfloats}
% pdf 拼接
\usepackage{pdfpages}
% 超链接url
\usepackage{url}
\usepackage{tikz}
\usepackage{pgfplots}
\pgfplotsset{compat=newest}
\usepackage[colorlinks=true, allcolors=red]{hyperref}
\usepackage{setspace}
\usepackage{bbm}

% --------------definations-------------- %
\def\*#1{\boldsymbol{#1}}
\def\+#1{\mathcal{#1}} 
\def\-#1{\mathrm{#1}}
\def\rm#1{\mathrm{#1}}
\def\=#1{\mathbb{#1}}
% Domains
\def\RR{\mathbb{R}}
\def\EE{\mathbb{E}}
\def\CC{\mathbb{C}}
\def\NN{\mathbb{N}}
\def\ZZ{\mathbb{Z}}
% Newcommand
\newcommand{\inner}[2]{\langle #1,#2\rangle} 
\newcommand{\numP}{\#\mathbf{P}} 
\renewcommand{\P}{\mathbf{P}}
\newcommand{\Var}[2][]{\mathbf{Var}_{#1}\left[#2\right]}
\newcommand{\E}[2][]{\mathbf{E}_{#1}\left[#2\right]}
\renewcommand{\emptyset}{\varnothing}
\newcommand{\ol}{\overline}
\newcommand{\argmin}{\mathop{\arg\min}}
\newcommand{\argmax}{\mathop{\arg\max}}
\renewcommand{\abs}[1]{\qty|#1|}
\newcommand{\defeq}{\triangleq} % triangle over =
\def\deq{\xlongequal{def}} % 'def' over =
\def\LHS{\text{LHS}}
\def\RHS{\text{RHS}}
\def\angbr#1{\langle#1\rangle} % <x>
\def\set#1{\qty{#1}}

\def\Esolve{\textcolor{blue}{Solve: }}
\def\Eproof{\textcolor{blue}{Proof: }}
\def\case#1{\textcolor{blue}{Case \uppercase\expandafter{\romannumeral#1}: }}

% \newmdtheoremenv{lemma}{Lemma}
% \newmdtheoremenv{theorem}{\textcolor{red}{Theorem}}
% \newmdtheoremenv{defi}{\textcolor{blue}{Definition}}
\newtheorem{lemma}{Lemma}
\newtheorem{thm}{Theorem}
\newtheorem{defi}{Definition}
\newtheorem{prp}{Proposition}
\newtheorem{remark}{Remark}
\newenvironment{md}{\begin{mdframed}}{\end{mdframed}}

\graphicspath{{figures/}}

% \begin{document}
% \title{<++>}
\author{Haoyu Zhen}
% \maketitle
\setlength{\parindent}{0pt}
\setstretch{1.2}
% \end{document}

\usepackage{fancyhdr}
\pagestyle{fancy}
% \fancypagestyle{mainFancy}{
%     \fancyhf{}
%     \renewcommand\headrulewidth{.5pt}       % 页眉横线
%     \renewcommand\footrulewidth{0pt}
%     \fancyhead[OC]{\fzkai{\leftmark}}       % 页眉章标题
%     \fancyhead[EC]{\fzkai{\@title}}         % 页眉文章题目
% 	\lhead{\fzkai{author}}
%     \fancyhead[OR,EL]{\thepage}                 % 页眉编号
%     \fancyfoot[r]{\thumb} % 将拇指放到没有被使用的页眉或页脚处
% }
\fancyhf{}
\fancyfoot[C]{\thepage}
\fancyhead[R]{\slshape{AI2615 Algorithm Design and Analysis}}
\fancyhead[L]{\slshape{Haoyu Zhen}}

% % theorems
\usepackage{thmtools}
\usepackage{thm-restate}
\usepackage[framemethod=TikZ]{mdframed}
\mdfsetup{skipabove=1em,skipbelow=0em, innertopmargin=12pt, innerbottommargin=8pt}

\theoremstyle{definition}

\declaretheoremstyle[headfont=\bfseries\sffamily, bodyfont=\normalfont,
	mdframed={
		nobreak,
		backgroundcolor=brown!14,
		topline=false,
		rightline=false,
		leftline=true,
		bottomline=false,
		linewidth=2pt,
		linecolor=brown!180,
	}
]{thmbrownbox}

\declaretheoremstyle[headfont=\bfseries\sffamily, bodyfont=\normalfont,
	mdframed={
		nobreak,
		backgroundcolor=Blue!4,
		topline=false,
		rightline=false,
		leftline=true,
		bottomline=false,
		linewidth=2pt,
		linecolor=NavyBlue!120,
	}
]{thmbluebox}

\declaretheoremstyle[headfont=\bfseries\sffamily, bodyfont=\normalfont,
	mdframed={
		nobreak,
		backgroundcolor=Green!5,
		topline=false,
		rightline=false,
		leftline=true,
		bottomline=false,
		linewidth=2pt,
		linecolor=OliveGreen!120,
	}
]{thmgreenbox}

\declaretheoremstyle[headfont=\bfseries\sffamily, bodyfont=\normalfont,
	mdframed={
		nobreak,
		topline=false,
		rightline=false,
		leftline=true,
		bottomline=false,
		linewidth=2pt,
		linecolor=OliveGreen!120,
	}
]{thmgreenline}

\declaretheoremstyle[headfont=\bfseries\sffamily, bodyfont=\normalfont,
	mdframed={
		nobreak,
		topline=false,
		rightline=false,
		leftline=true,
		bottomline=false,
		linewidth=2pt,
		linecolor=NavyBlue!70,
	}
]{thmblueline}

\declaretheorem[numberwithin=section, style=thmbrownbox, name={\color{Brown}Definition}]{defi}
\declaretheorem[numberwithin=section, style=thmgreenbox, name={\color{OliveGreen}Law}]{law}
\declaretheorem[numberwithin=section, style=thmbluebox, name={\color{Blue}Corollary}]{cor}
\declaretheorem[numberwithin=section, style=thmgreenline, name={\color{OliveGreen}Property}]{prt}
\declaretheorem[numberwithin=section, style=thmbluebox, name={\color{Blue}Proposition}]{prp}
\declaretheorem[numberwithin=section, style=thmbluebox, name={\color{Blue}Theorem}]{thm}
\declaretheorem[numberwithin=section, style=thmbluebox, name={\color{Blue}Lemma}]{lemma}
\declaretheorem[numberwithin=section, style=thmbrownbox,  name={\color{Brown}Example}]{eg}
\declaretheorem[numberwithin=section, style=thmgreenline, name={\color{OliveGreen}Remark}]{remark}
\declaretheorem[numbered=no,style=thmblueline, name={\color{NavyBlue!70}Proof},qed=$\square$]{prf}
\numberwithin{equation}{section}


% On my MAC's Desktop
%! Tex program = xelatex
% \documentclass{article}
%中文
%\usepackage[UTF8]{ctex}
%数学公式
\usepackage{amsmath,amssymb}
%\usepackage{ntheorem}
% \usepackage[framemethod=TikZ]{mdframed}
\usepackage{amsthm}
%边界
\usepackage[letterpaper,top=2cm,bottom=3cm,left=2cm,right=2cm,marginparwidth=1.75cm]{geometry}%table package
%Table
\usepackage{multirow,booktabs}
\usepackage{makecell}
%字体颜色
\usepackage{color}
% \usepackage[dvipsnames]{xcolor}  % 更全的色系
%代码
\usepackage[OT1]{fontenc}
% MATLAB 代码风格
%\usepackage[framed,numbered,autolinebreaks,useliterate]{/Users/anye_zhenhaoyu/Desktop/Latex/mcode}
\usepackage{listings}
\usepackage{algorithm}
\usepackage{algorithmic}
\usepackage{pythonhighlight} % Python
%插图
\usepackage{graphicx}
%改变item格式
\usepackage{enumerate}
%物理
\usepackage{physics}
%extra arrows
\usepackage{extarrows}
% caption(居中指令)
%\usepackage[justification=centering]{caption}
\usepackage{caption}
% htpb
\usepackage{stfloats}
% pdf 拼接
\usepackage{pdfpages}
% 超链接url
\usepackage{url}
\usepackage{tikz}
\usepackage{pgfplots}
\pgfplotsset{compat=newest}
\usepackage[colorlinks=true, allcolors=red]{hyperref}
\usepackage{setspace}
\usepackage{bbm}

% --------------definations-------------- %
\def\*#1{\boldsymbol{#1}}
\def\+#1{\mathcal{#1}} 
\def\-#1{\mathrm{#1}}
\def\rm#1{\mathrm{#1}}
\def\=#1{\mathbb{#1}}
% Domains
\def\RR{\mathbb{R}}
\def\EE{\mathbb{E}}
\def\CC{\mathbb{C}}
\def\NN{\mathbb{N}}
\def\ZZ{\mathbb{Z}}
% Newcommand
\newcommand{\inner}[2]{\langle #1,#2\rangle} 
\newcommand{\numP}{\#\mathbf{P}} 
\renewcommand{\P}{\mathbf{P}}
\newcommand{\Var}[2][]{\mathbf{Var}_{#1}\left[#2\right]}
\newcommand{\E}[2][]{\mathbf{E}_{#1}\left[#2\right]}
\renewcommand{\emptyset}{\varnothing}
\newcommand{\ol}{\overline}
\newcommand{\argmin}{\mathop{\arg\min}}
\newcommand{\argmax}{\mathop{\arg\max}}
\renewcommand{\abs}[1]{\qty|#1|}
\newcommand{\defeq}{\triangleq} % triangle over =
\def\deq{\xlongequal{def}} % 'def' over =
\def\LHS{\text{LHS}}
\def\RHS{\text{RHS}}
\def\angbr#1{\langle#1\rangle} % <x>
\def\set#1{\qty{#1}}

\def\Esolve{\textcolor{blue}{Solve: }}
\def\Eproof{\textcolor{blue}{Proof: }}
\def\case#1{\textcolor{blue}{Case \uppercase\expandafter{\romannumeral#1}: }}

% \newmdtheoremenv{lemma}{Lemma}
% \newmdtheoremenv{theorem}{\textcolor{red}{Theorem}}
% \newmdtheoremenv{defi}{\textcolor{blue}{Definition}}
\newtheorem{lemma}{Lemma}
\newtheorem{thm}{Theorem}
\newtheorem{defi}{Definition}
\newtheorem{prp}{Proposition}
\newtheorem{remark}{Remark}
\newenvironment{md}{\begin{mdframed}}{\end{mdframed}}

\graphicspath{{figures/}}

% \begin{document}
% \title{<++>}
\author{Haoyu Zhen}
% \maketitle
\setlength{\parindent}{0pt}
\setstretch{1.2}
% \end{document}

\usepackage{fancyhdr}
\pagestyle{fancy}
% \fancypagestyle{mainFancy}{
%     \fancyhf{}
%     \renewcommand\headrulewidth{.5pt}       % 页眉横线
%     \renewcommand\footrulewidth{0pt}
%     \fancyhead[OC]{\fzkai{\leftmark}}       % 页眉章标题
%     \fancyhead[EC]{\fzkai{\@title}}         % 页眉文章题目
% 	\lhead{\fzkai{author}}
%     \fancyhead[OR,EL]{\thepage}                 % 页眉编号
%     \fancyfoot[r]{\thumb} % 将拇指放到没有被使用的页眉或页脚处
% }
\fancyhf{}
\fancyfoot[C]{\thepage}
\fancyhead[R]{\slshape{AI2615 Algorithm Design and Analysis}}
\fancyhead[L]{\slshape{Haoyu Zhen}}


\graphicspath{{figures/}}
\setlength{\headheight}{14pt}
\fancyhead[C]{\large\textbf{Written Assignment 3}}
\begin{document}
% \tableofcontents
% \title{}
% \maketitle

\section*{Question 1}
$A[n]=\{a_0,\,a_1,\,\cdots,\,a_m\}$ denotes $A[0]=a_0,\,A[1]=a_1,\,\cdots,\,A[m]=a_{m}$.
\[
	x[n]=\{1,\,0,\,-1,\,0\}\text{ and  }h[n]=\{1,\,2,\,4,\,8\}
	.\]
\textbf{(a).}
\[
	X[k]=\sum_nx[n]e^{\flatfrac{-j\pi kn}{2}}=\{0,\,2,\,0,\,2\}
	.\]
\\
\textbf{(b).}
\[
	\begin{aligned}
		H[k]
		=\sum_nh[n]e^{\flatfrac{-j\pi kn}{2}}
		=\{15,\,-3+6j,\,-5,\,-3-6j\}
	\end{aligned}
	.\]
\\
\textbf{(c).} By defination,
\[
	x[n]\circled{4}h[n]
	=
	\{-3,\,-6,\,3,\,6\}
	.\]
\\
\textbf{(d).}
$Y[n]=X[n]\cdot H[n]=\{0,\,-6+12j,\,0,\,-6-12j\}$. Thus
\[
	y[n]
	=
	\frac{1}{N}\qty{\mathrm{DFT}\qty(Y^*[n])}^*
	=
	\frac{1}{4}\{-12,\,-24,\,12,\,24\}
	=
	\{-3,\,-6,\,3,\,6\}
	.\]
\section*{Question 2}
\textbf{(a).}
\case{1}
$\forall r\in\ZZ$, $m-k\ne r(N+1)$.
\[
	\begin{aligned}
		\LHS
		 & =
		\sum_{n=\flatfrac{N}{2}}^{\flatfrac{N}{2}}\exp[-\frac{2j\pi n(m-k)}{N+1}]
		\\[6pt]&=
		\flatfrac
		{\qty{\exp[\frac{j\pi(m-k)N}{N+1}]-\exp[\frac{j\pi(m-k)(N+2)}{N+1}]}}
		{\qty{1-\exp[-\frac{2j\pi(m-k)}{N+1}]}}
		\\[6pt]&=0
	\end{aligned}
	.\]
\case{2}
$\exists r\in\ZZ$, $m-k=r(N+1)$.
\[
	\LHS
	=
	\sum_{n=\flatfrac{N}{2}}^{\flatfrac{N}{2}}1
	=N+1
	.\]
Thus $\LHS=\RHS$.
\newpage
\textbf{(b).}
I use Dirac Braket $\braket{\cdot}{\cdot}$ to represent inner product.

Then \[\vec{F}=\frac{1}{N+1}\sum_{k=\flatfrac{-N}{2}}^{\flatfrac{N}{2}}\braket{f}{\hat{e}_k}\hat{e}_k.\]
By the orthogonality property that $\braket{\hat{e}_k}{\hat{e}_m}=0$ if $m\ne k$ otherwise $\braket{\hat{e}_k}{\hat{e}_m}=N+1$:
\[
	f_m
	=
	\braket{F}{\hat{e}_m}
	=
	\frac{1}{N+1}
	\sum_{k=\flatfrac{-N}{2}}^{\flatfrac{N}{2}}
	\braket{f}{\hat{e}_k}
	\braket{\hat{e}_k}{\hat{e}_m}
	=
	\braket{f}{\hat{e}_m}
	,\] which means
\[
	f_m=\sum_{k=\flatfrac{-N}{2}}^{\flatfrac{N}{2}}F_kW_{N+1}^{mk}
	.\]
\textbf{(c).}
\[
	\lim_{N\to\infty} x_{\pm\flatfrac{N}{2}}=\lim_{N\to\infty}\pm\frac{NA}{2(N+1)}=\pm \frac{A}{2}
	.\]
\section*{Question 3}
By defination of DFT:
\[
	F[k]=\mathrm{DFT}(\bar{f}_n)[k]=
	\begin{cases}
		1                                                         & k=0 \\
		(-2-\flatfrac{\sqrt{2}}{2})+(2+\flatfrac{3\sqrt{2}}{2})j  & k=1 \\
		j                                                         & k=2 \\
		(-2+\flatfrac{\sqrt{2}}{2})+(-2+\flatfrac{3\sqrt{2}}{2})j & k=3 \\
		-1                                                        & k=4 \\
		(-2+\flatfrac{\sqrt{2}}{2})+(2-\flatfrac{3\sqrt{2}}{2})j  & k=5 \\
		-j                                                        & k=6 \\
		(-2-\flatfrac{\sqrt{2}}{2})-(2+\flatfrac{3\sqrt{2}}{2})j  & k=7 \\
	\end{cases}
	,\] which is not odd and imaginary. This is because $f(x)$ is not well-defined when implementing periodic extension. I.e.,  $f(-0.5)\ne f(-0.5+1)=f(0.5)$. To ameliorate it, the input sequence should be:
\[
	\bar{f}_n=\{0,\,-1,\,-1,\,-1,\,0,\,1,\,1,\,1\}
	.\]
Then
\[
	F=
	\begin{cases}
		0               & k=0,2,4,6 \\
		(2+2\sqrt{2})j  & k=1       \\
		(-2+2\sqrt{2})j & k=3       \\
		(2-2\sqrt{2})j  & k=5       \\
		(-2-2\sqrt{2})j & k=7       \\
	\end{cases}
,\] which is odd and imaginary.

\section*{Question 4}
\textbf{(a).}
Trivially, $n=50+10-1=59$.
\\\\
\textbf{(b).}
Let $y_1=x[n]\circled{50}h[n]$, $y_2=x[n]*h[n]$ for the first 5 points and $y=x[n]*h[n]$.
\\\case{1}
Obviously, $y[n]=y_2[n]=5$ when $0\le n\le 4$.
\\\case{2}
Also, $y[n]=y_1[n]=10$ when $9\le n\le 49$.
\\\case{3}
If $50\le n\le 54$, then $y[n]=y_1[n]-y_2[n-50]=5$.
\\
For other $n$, $y[n]$ could not be determined.
\end{document}
