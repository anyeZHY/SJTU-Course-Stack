%! Tex program = xelatex
\documentclass{article}
%中文
\usepackage[UTF8]{ctex}
%数学公式
\usepackage{amsmath,amssymb}
%\usepackage{ntheorem}
\usepackage{mdframed} %公式框1
% e.g., \newmdtheoremenv{theorem}{Theorem}
\usepackage{amsthm}
%边界
\usepackage[letterpaper,top=2cm,bottom=2cm,left=2.5cm,right=2.5cm,marginparwidth=1.75cm]{geometry}%table package
%Table
\usepackage{multirow,booktabs}
\usepackage{makecell}
%字体颜色
\usepackage{color}
\usepackage[dvipsnames]{xcolor}  % 更全的色系
%代码
\usepackage[OT1]{fontenc}
% MATLAB 代码风格
\usepackage[framed,numbered,autolinebreaks,useliterate]{/Users/anye_zhenhaoyu/Desktop/Latex/mcode}
\usepackage{listings}
\lstset{frame=none,aboveskip=5pt,belowskip=4pt}
\usepackage{algorithm}
\usepackage{algorithmic}
\usepackage{pythonhighlight} % Python
%插图
\usepackage{graphicx}
%改变item格式
\usepackage{enumerate}
%物理
\usepackage{physics}
%extra arrows
\usepackage{extarrows}
% caption(居中指令)
%\usepackage[justification=centering]{caption}
\usepackage{caption}
% htpb
\usepackage{stfloats}
% pdf 拼接
\usepackage{pdfpages}
% 超链接url
\usepackage{url}
\usepackage{multicol}
\setlength\columnsep{1.5cm}
\columnseprule=1pt
\usepackage[colorlinks=true, allcolors=blue]{hyperref}

% --------------definations-------------- %
\def\*#1{\boldsymbol{#1}}
\def\+#1{\mathcal{#1}} 
\def\-#1{\mathrm{#1}}
% Domains
\def\RR{\mathbb{R}}
\def\CC{\mathbb{C}}
\def\NN{\mathbb{N}}
\def\ZZ{\mathbb{Z}}
% Newcommand
\newcommand{\inner}[2]{\langle #1,#2\rangle} 
\renewcommand{\mid}{\;\middle\vert\;} 
\newcommand{\cmid}{\,:\,}
\newcommand{\numP}{\#\mathbf{P}} 
\renewcommand{\P}{\mathbf{P}}
\newcommand{\defeq}{\triangleq}
\newcommand{\Var}[2][]{\mathbf{Var}_{#1}\left[#2\right]}
\newcommand{\E}[2][]{\mathbf{E}_{#1}\left[#2\right]}
\renewcommand{\emptyset}{\varnothing}
\newcommand{\ol}{\overline}
\newcommand{\argmin}{\mathop{\arg\min}}
\newcommand{\argmax}{\mathop{\arg\max}}

\def\Esolve{\textcolor{blue}{Solve: }}
\def\Eproof{\textcolor{blue}{Proof: }}
\def\case#1{\textcolor{blue}{Case \uppercase\expandafter{\romannumeral#1}: }}

\newmdtheoremenv{lemma}{Lemma}
\newmdtheoremenv{theorem}{\textcolor{red}{Theorem}}
\newmdtheoremenv{defi}{\textcolor{blue}{Definition}}

\newenvironment{md}{\begin{mdframed}}{\end{mdframed}}

\graphicspath{{../}}

\begin{document}
\title{编程作业1简明报告}
\author{Haoyu Zhen}
\maketitle
\begin{multicols}{2}
\section{添加模糊及噪音}
\subsection{模糊化}
利用卷积运算即可制造模糊,
\begin{lstlisting}
% Blurred image
PSF = ones([5,5])*0.04;
% blurred = conv2(origin,PSF);
% size = 516;
blurred = conv2(origin,PSF,'same');
size = 512;
\end{lstlisting}
效果如图(\ref{blur})所示。
值得说明的是,我传入了`same'这个参数,以使得前后图片大小一致,但这会导致边缘信息的丢失。因此在恢复图像时,图片边缘会出现“方格”现象,如图(\ref{wiener})所示,若不传入`same'参数,后续的逆滤波会取得更好的效果。
\subsection{噪音化}
使用awgn函数可直接为图片添加噪音,对应代码如下:
\begin{lstlisting}
% Blurred and noised image
bn30 = awgn(blurred,30,'measured');
bn20 = awgn(blurred,20,'measured');
bn10 = awgn(blurred,10,'measured');
\end{lstlisting}
效果如图(\ref{noise})所示。

\section{图像复原}
\subsection{直接逆滤波}
在编程上可以用两种方式实现,基于定义,利用傅立叶变换进行除法;或者利用deconvwnr函数,并将nsr设为0,(默认值)。
我选择了后者作为实现方式,前者在源代码中以注释形式给出:
\begin{lstlisting}
% Implement filtering directly
% F_SPF = fft2(PSF,size,size);
% r1 = ifft2(fft2(bn10)./F_SPF);
% r2 = ifft2(fft2(bn20)./F_SPF);
% r3 = ifft2(fft2(bn30)./F_SPF);
% For simplicity, we could use deconvwnr(bn, PSF)
df1 = deconvwnr(bn10, PSF);
df2 = deconvwnr(bn20, PSF);
df3 = deconvwnr(bn30, PSF);
\end{lstlisting}
效果如图(\ref{direct})所示。



\subsection{Wiener 滤波}
使用MATLAB中deconvwnr函数可一定程度的去除噪音与模糊,代码如下:
\begin{lstlisting}
% Implement wiener filtering
wnr1 = deconvwnr(bn10, PSF, 0.1);
wnr2 = deconvwnr(bn20, PSF, 0.01);
wnr3 = deconvwnr(bn30, PSF, 0.001);
\end{lstlisting}
效果如图(\ref{wiener})所示。
\subsection{伪-逆滤波}
“Image Restoration.pdf”第29页给出了pseudo-inverse filtering方法,实现如下,不过源代码中被我所注释并附在于最后。
\begin{lstlisting}
% implement pseudo-inverse filtering
% Here \epsilon = 0.5
F_PSF = fft2(PSF,size,size);
H = (abs(F_PSF)>0.5)./F_PSF;
F_bn30 = fft2(bn30);
F_bn20 = fft2(bn20);
F_bn10 = fft2(bn10);
rr1 = ifft2(F_bn10.*H);
rr2 = ifft2(F_bn20.*H);
rr3 = ifft2(F_bn30.*H);
\end{lstlisting}
\end{multicols}

\newpage
\section*{\centering\textcolor{blue}{Appendix}}
\begin{figure}[H]
	\centering
	\begin{minipage}[b]{0.46\linewidth}
		\includegraphics[width=\linewidth,bb=0 0 20cm 20cm]{baboon.bmp}
		\caption{Original image}
	\end{minipage}
	\begin{minipage}[b]{0.46\linewidth}
		\includegraphics[width=\linewidth,bb=0 0 20cm 20cm]{blurred.bmp}
		\caption{blurred image}
		\label{blur}
	\end{minipage}
\end{figure}
\begin{figure}[H]
	\centering
	\begin{minipage}[b]{0.32\linewidth}
		\includegraphics[width=\linewidth,bb=0 0 20cm 20cm]{bn3.bmp}
		\caption*{30dB}
	\end{minipage}
	\begin{minipage}[b]{0.32\linewidth}
		\includegraphics[width=\linewidth,bb=0 0 20cm 20cm]{bn2.bmp}
		\caption*{20dB}
	\end{minipage}
	\begin{minipage}[b]{0.32\linewidth}
		\includegraphics[width=\linewidth,bb=0 0 20cm 20cm]{bn1.bmp}
		\caption*{10dB}
	\end{minipage}
	\caption{Blurred and noised image}
	\label{noise}
\end{figure}
\begin{figure}[H]
	\centering
	\begin{minipage}[b]{0.32\linewidth}
		\includegraphics[width=\linewidth,bb=0 0 20cm 20cm]{df3.bmp}
		\caption*{Restoration-30dB}
	\end{minipage}
	\begin{minipage}[b]{0.32\linewidth}
		\includegraphics[width=\linewidth,bb=0 0 20cm 20cm]{df2.bmp}
		\caption*{Restoration-20dB}
	\end{minipage}
	\begin{minipage}[b]{0.32\linewidth}
		\includegraphics[width=\linewidth,bb=0 0 20cm 20cm]{df1.bmp}
		\caption*{Restoration-10dB}
	\end{minipage}
	\caption{Naive inverse filtering}
	\label{direct}
\end{figure}
\begin{figure}[H]
	\centering
	\begin{minipage}[b]{0.32\linewidth}
		\includegraphics[width=\linewidth,bb=0 0 20cm 20cm]{wnr3.bmp}
		\caption*{Restoration-30dB}
	\end{minipage}
	\begin{minipage}[b]{0.32\linewidth}
		\includegraphics[width=\linewidth,bb=0 0 20cm 20cm]{wnr2.bmp}
		\caption*{Restoration-20dB}
	\end{minipage}
	\begin{minipage}[b]{0.32\linewidth}
		\includegraphics[width=\linewidth,bb=0 0 20cm 20cm]{wnr1.bmp}
		\caption*{Restoration-10dB}
	\end{minipage}
	\caption{Wiener filtering}
	\label{wiener}
\end{figure}

\end{document}


