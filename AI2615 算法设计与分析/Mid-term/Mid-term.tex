% ! Tex program = xelatex
\documentclass{article}
% \PassOptionsToPackage{quiet}{fontspec}% (or try silent)
% 中文
% \usepackage[UTF8]{ctex}

% On my MAC's Desktop
%! Tex program = xelatex
% \documentclass{article}
%中文
%\usepackage[UTF8]{ctex}
%数学公式
\usepackage{amsmath,amssymb}
%\usepackage{ntheorem}
% \usepackage[framemethod=TikZ]{mdframed}
\usepackage{amsthm}
%边界
\usepackage[letterpaper,top=3cm,bottom=3cm,left=2cm,right=2cm,marginparwidth=1.75cm]{geometry}%table package
%Table
\usepackage{multirow,booktabs}
\usepackage{makecell}
%字体颜色
\usepackage{color}
\usepackage[dvipsnames]{xcolor}  % 更全的色系
%代码
\usepackage[OT1]{fontenc}
% MATLAB 代码风格
% \usepackage[framed,numbered,autolinebreaks,useliterate]{/Users/anye_zhenhaoyu/Desktop/Latex/mcode}
\usepackage{listings}
\usepackage{algorithm}
\usepackage{algorithmic}
\usepackage{pythonhighlight} % Python
%插图
\usepackage{graphicx}
%改变item格式
\usepackage{enumerate}
%物理
\usepackage{physics}
%extra arrows
\usepackage{extarrows}
% caption(居中指令)
\usepackage[justification=centering]{caption}
% \usepackage{caption}
% htpb
\usepackage{stfloats}
% pdf 拼接
\usepackage{pdfpages}
% 超链接url
\usepackage{url}
% \usepackage{tikz}
\usepackage{pgfplots}
\pgfplotsset{compat=newest}
\usepackage[colorlinks=true, allcolors=blue]{hyperref}
\usepackage{setspace}

% --------------definations-------------- %
\def\*#1{\boldsymbol{#1}}
\def\+#1{\mathcal{#1}} 
\def\-#1{\bar{#1}}
% Domains
\def\RR{\mathbb{R}}
\def\CC{\mathbb{C}}
\def\NN{\mathbb{N}}
\def\ZZ{\mathbb{Z}}
% Newcommand
\newcommand{\inner}[2]{\langle #1,#2\rangle} 
\newcommand{\numP}{\#\mathbf{P}} 
\renewcommand{\P}{\mathbf{P}}
\newcommand{\Var}[2][]{\mathbf{Var}_{#1}\left[#2\right]}
\newcommand{\E}[2][]{\mathbf{E}_{#1}\left[#2\right]}
\renewcommand{\emptyset}{\varnothing}
\newcommand{\ol}{\overline}
\newcommand{\argmin}{\mathop{\arg\min}}
\newcommand{\argmax}{\mathop{\arg\max}}
\renewcommand{\abs}[1]{\qty|#1|}
\newcommand{\defeq}{\triangleq} % triangle over =
\def\deq{\xlongequal{def}} % 'def' over =
\def\LHS{\text{LHS}}
\def\RHS{\text{RHS}}
\def\angbr#1{\langle#1\rangle} % <x>

\def\Esolve{\textcolor{blue}{Solve: }}
\def\Eproof{\textcolor{blue}{Proof: }}
\def\case#1{\textcolor{blue}{Case \uppercase\expandafter{\romannumeral#1}: }}

\newtheorem{lemma}{Lemma}
\newtheorem{thm}{Theorem}
\newtheorem{defi}{Definition}
\newtheorem{prp}{Proposition}
\newenvironment{md}{\begin{mdframed}}{\end{mdframed}}

\date{\today}
\usepackage{fancyhdr}
\pagestyle{fancy}
\fancyhead[L]{\slshape{Haoyu Zhen}}
\fancyhead[R]{{\today}}

% \begin{document}
% \title{<++>}
\author{Haoyu Zhen}
% \maketitle
\setlength{\parindent}{0pt}
\setstretch{1.2}
% \end{document}



\graphicspath{{figures/}}

\begin{document}
% \tableofcontents
\title{\vspace{-0.7em}Edmonds’ Blossom Algorithm (Mid-Exam of AI2615)}
\maketitle

\section*{Problem 1}
``$\Longrightarrow$'': trivial.

\vspace{0.5em}
If a $M$-augmenting path $p=v_1v_2\cdots v_n$ exists, then 
\[
	\big(M-\{e|e\in p\land e\in M\}\big)
	\cup
	\{e|e\in p\land e\notin M\}
\]
is a matching larger than $M$.

\vspace{1em}
``$\Longleftarrow$'':

\vspace{0.5em}
Proof by contradiction. If $M$ is not a maximum matching and let $M'$ be a maximum one.
Then we analyse the property of $M^\dagger\defeq(M\cup M')-(M\cap M')$. For simplicity, we could count $\+M$ as a graph (indeed, it is a set of edges).
\begin{enumerate}
	\item 
		$\forall e\in M^\dagger$, $e\in M$ XOR $e\in M'$.
	\item 
		The paths in $M^\dagger$ are alternating. I.e., for all vertex $v$ in $M^\dagger$, $v$ could be incident to \textbf{at most one} edge from $M$ and \textbf{at most one} edge from $M'$. 
	\item
		By 2., $\forall$ connected component $C$ in $M^\dagger$,$C$ is a path.
	\item
		There exists a connected component $\+C$ in $M^\dagger$, such that 
		\[
			\big|\{e|e\in \+C\land e\in M'\}\big|
			>
			\big|\{e|e\in \+C\land e\in M\}\big|
		.\]
		(Beacuse $\abs{M'}>\abs{M}$.)
	\item
		The both ends of $\+C$ (in 4.) is in $M'$.
\end{enumerate}
Finally, we construct a augmenting path $\+C$ for $M$, which leads to a contradition.

\section*{Problem 2}
Contract the cycle $v_1v_2\cdots v_{2k+1}v_1$ to $u$.

\vspace{0.5em}
``$\Longrightarrow$'':

\vspace{0.5em}
Proof by contradiction.
Suppose that there exists a matching $M'$ of $G'$ such that $\abs{M'}>\abs{M-C}$.

Then there exists $C_{\-{new}}$ in $G$ such that $C_{\-{new}}$ meet no other edge in $M'$, because ``$2k+1=2\times k +1$''. LHS: the number of vertices in $C_{\-{new}}$. RHS: 1 represents $u$ and $2\times k$ represents the matched vertices by $C_{\-{new}}$.

Then $M'\cup C_{\-{new}}$ is a larger matching of $G$. Contradiction!

\vspace{1em}
``$\Longleftarrow$'':

\vspace{0.5em}
Similarly, assume that there exists a matching $M'$ for $G$ such that $\abs{M'}>\abs{M}$.

Then in $M'$, there exists a vertex $v$ who is not matched by $C$. Let this $v$ be $u$ and contract the cycle $C$.
We get a larger matching of $G'$. Contradiction.

\section*{Problem 3}
M-alternating forests (MAF) exist by the fact that empty set is an MAF.
Now we find a maximal one greedily:
\begin{algorithm}[htbp]
	\caption{Find an maximal forest $\defeq\+F(G,M)$}
	\label{maf}
	\begin{algorithmic}[1]
		\renewcommand{\algorithmicrequire}{\textbf{Input:}}
		\renewcommand{\algorithmicensure}{\textbf{Output:}}
		\renewcommand{\algorithmiccomment}[1]{\hfill\textit{\textcolor{blue}{\##1}}}
		\REQUIRE A graph $G$ whose matching is $M$
		\STATE $F_{\max}\gets \emptyset$
		\STATE Select a root $r$ in $G$. $T\gets (\{r\},\emptyset)$.
		\WHILE{$\exists$ a alternating path $p$ begins and ends at outer vertices.}
		\STATE Insert $p$ into $T$
		\STATE Update $O_T$ and $I_T$
		\COMMENT{$O_T$ represents the outer vertices in $T$ ($I_T$: inner)}
		\STATE Delete all edges (not in $T$) which are incident with the inner vertex
		\ENDWHILE
		\STATE Generate a new graph $H$ and a matching $M_{\mathrm{new}}$ by deleting the tree $T$ from $G$
		\STATE Merge $T$ and $\+F(H,M_{\mathrm{new}})$ $\longrightarrow$ $F_{\max}$
		\COMMENT{Recursion here}
		\RETURN $F_{\max}$
	\end{algorithmic} 
\end{algorithm}

\textbf{Correctness:} we just need to prove that $T$ is ``maximal''.
\begin{itemize}
	\item 
		The root of $T$ is the end points of a  maximal augmenting path $\implies$ T will not be the sub-tree of other maximal alternating tree.
	\item 
		There is no alternating path could be added into $T$.
\end{itemize}

\textbf{Complexity:} (We bound the complexity \textbf{LOOSELY})
\begin{itemize}
	\item 
		Find the root: $\+O(nm)$.
	\item
		Delete and insert (with the idea of charging): $\+O(m)$.
	\item
		Find the alternating path (line 3; we use DFS here): $\+O(nm)$.
	\item
		Update $O_T$ and  $I_T$:  $\+O(n)$.
\end{itemize}

\section*{Problem 4}
The augmenting path should begin at a root $r_1$ and end at another one ($r_2$).

First I introduce my intuition: as figure 4 shown (in \texttt{edmonds.pdf}), we need to go \textbf{down} and then go \textbf{up} along several vertices. The conversion from down to up needs 2 consecutive outer vertices.

\textit{Proof}.
\begin{itemize}
	\item Define going up and down formally. 

		Going down: if we are at a inner vertex, then we will choose an edge in $M$. For outer vertex, choose an dege whiche are not in $M$. Otherwise we are going up.
	\item
		Now we consider some possible cases.
	\item
		Inner $\to$ inner, ``$\to$'' is a edge in $M$. Not exists. Because this leads to a cernario that a vertex is matched twice.
	\item
		Inner $\to$ outer OR outer $\to$ inner. This does not change the upward and downward trend.
	\item
		Outer $\to$ outer, here ``$\to$'' is a edge not in $M$. Thus, at the second outer vertex, we need to choose an edge in $M$. We go up now!
\end{itemize}
$\hfill\square$

\section*{Problem 5}
Quite trivial!

\vspace{1em}If $u$ and $v$ belong to distinct components of $F$:

\vspace{0.3em}
We construct the $M$-augmenting path directly: $r_1\to u-v \to r_2$ where $u(v)$ belongs to the tree with root $r_1(r_2)$.

\vspace{1em}If $u$ and $v$ belong to the same component of $F$:

\vspace{0.3em}
Obviously that the path in tree from $u$ to $v$ is even (by alternating property). Thus we have an odd cycle.

\section*{Problem 6}
My idea/intuition: I want to design a algorithm by which we could eradicate the augmenting path greedily and recursively. Why to contract the blossoms? Because we wan to find the path easily.
\begin{algorithm}[H]
	\caption{Find the augmenting path $\defeq \+P(G,M)$}
	\begin{algorithmic}[1]
		\renewcommand{\algorithmicrequire}{\textbf{Input:}}
		\renewcommand{\algorithmicensure}{\textbf{Output:}}
		\renewcommand{\algorithmiccomment}[1]{\hfill\textit{\textcolor{blue}{\##1}}}
		\label{path}
		\REQUIRE A matching $M$ of the graph $G$
		\ENSURE Path $p$
		\STATE Construct the forest $F(G,M)$
		\STATE Find an edge whose endpoints are 2 outer vertices $u$ and $v$. If not, \textbf{return} $\emptyset$
		\IF{$u$ and  $v$ belong to the same component of $F$}
		\RETURN $r_1\to u-v \to r_2$
		\ELSE
		\STATE We could find an odd cycle $C$ with root $r$
		\STATE $v\gets \min_{u\in V(C)}d(u,r)$
		\COMMENT{Find vertex whose distance to $r$ is minimum}
		\STATE $M'\gets (M-p)\cup\{e|e\notin M\land e\in p\}$ where $p$ is the path from  $r$ to $v$
		\STATE Now we could contract the cycle ``safely'' $\to\ G'$
		\COMMENT{Satisfy the requirement in P2.}
		\STATE $P'\gets \+P(G',M')$
		\STATE ``Interpret'' $P'$: \\
		If $P'$ contains the vertex contracted from a cycle $C$, find an even path in $C$ to connect $P'$ to $P$
		\RETURN $P$
		\ENDIF
	\end{algorithmic} 
\end{algorithm}
Finally, we design an agorithm to find a maximum path. Forany given $G$,

\begin{algorithm}[H]
	\caption{Find the maximum matching}
	\label{matching}
	\begin{algorithmic}[1]
		\renewcommand{\algorithmicrequire}{\textbf{Input:}}
		\renewcommand{\algorithmicensure}{\textbf{Output:}}
		\renewcommand{\algorithmiccomment}[1]{\hfill\textit{\textcolor{blue}{\##1}}}
		\REQUIRE 
		\ENSURE 
		\STATE $M\gets \{e\}$ (we pick $e$ randomly)
		\WHILE{$p\gets \+P(G,M)$ is not an emptyset}
		\STATE Update $M$: eradicate this augmenting path. (Use the method in P1.)
		\ENDWHILE
		\RETURN $M$
	\end{algorithmic}
\end{algorithm}
\textbf{Soundness}:
\begin{enumerate}
	\item 
		Algo \ref{matching} holds naturally by P1. Also, $M$ is strictly increasing and bounded by $\abs{E}$ (Algo \ref{matching} will terminate).
	\item
		Now we analyse Algo \ref{path}.
		\begin{itemize}
			\item
				Line 7-8: $v$ represents the vertex in cycle who meets other edges. So we update $M$ without changing its cardinality.
			\item
				Line 9: $\+P(G',M')$ exists iff $\+P(G,M)$ exists
			\item
				Line 10-11: convert the augmenting path  $P'$ in  $G'$ to a path $G$.
		\end{itemize}
\end{enumerate}

\vspace{1em}
\textbf{Complexity}:
\begin{itemize}
	\item 
		While-loop in Algo \ref{matching} is bounded by $\abs{E}$.
	\item
		Algo \ref{path} is dominated by $\+F(G,M)$.
\end{itemize}
Thus the complexity is $\+O(n^2m)$.

\section*{Reference}
\begin{enumerate}
	\item \url{https://www14.in.tum.de/lehre/2015WS/ea/split/sec-Augmenting-Paths-for-Matchings-single.pdf} (for problem 1).
	\item \url{https://en.wikipedia.org/wiki/Blossom_algorithm} (for problem 6).
\end{enumerate}

\end{document}
