% ! Tex program = xelatex
\documentclass{article}
% \PassOptionsToPackage{quiet}{fontspec}% (or try silent)
% 中文
% \usepackage[UTF8]{ctex}

% For more choices
% %! Tex program = xelatex
% \documentclass{article}
%中文
%\usepackage[UTF8]{ctex}
%数学公式
\usepackage{amsmath,amssymb}
%\usepackage{ntheorem}
% \usepackage[framemethod=TikZ]{mdframed}
\usepackage{amsthm}
%边界
\usepackage[letterpaper,top=2cm,bottom=3cm,left=2cm,right=2cm,marginparwidth=1.75cm]{geometry}%table package
%Table
\usepackage{multirow,booktabs}
\usepackage{makecell}
%字体颜色
\usepackage{color}
% \usepackage[dvipsnames]{xcolor}  % 更全的色系
%代码
\usepackage[OT1]{fontenc}
% MATLAB 代码风格
%\usepackage[framed,numbered,autolinebreaks,useliterate]{/Users/anye_zhenhaoyu/Desktop/Latex/mcode}
\usepackage{listings}
\usepackage{algorithm}
\usepackage{algorithmic}
\usepackage{pythonhighlight} % Python
%插图
\usepackage{graphicx}
%改变item格式
\usepackage{enumerate}
%物理
\usepackage{physics}
%extra arrows
\usepackage{extarrows}
% caption(居中指令)
%\usepackage[justification=centering]{caption}
\usepackage{caption}
% htpb
\usepackage{stfloats}
% pdf 拼接
\usepackage{pdfpages}
% 超链接url
\usepackage{url}
\usepackage{tikz}
\usepackage{pgfplots}
\pgfplotsset{compat=newest}
\usepackage[colorlinks=true, allcolors=red]{hyperref}
\usepackage{setspace}
\usepackage{bbm}

% --------------definations-------------- %
\def\*#1{\boldsymbol{#1}}
\def\+#1{\mathcal{#1}} 
\def\-#1{\mathrm{#1}}
\def\rm#1{\mathrm{#1}}
\def\=#1{\mathbb{#1}}
% Domains
\def\RR{\mathbb{R}}
\def\EE{\mathbb{E}}
\def\CC{\mathbb{C}}
\def\NN{\mathbb{N}}
\def\ZZ{\mathbb{Z}}
% Newcommand
\newcommand{\inner}[2]{\langle #1,#2\rangle} 
\newcommand{\numP}{\#\mathbf{P}} 
\renewcommand{\P}{\mathbf{P}}
\newcommand{\Var}[2][]{\mathbf{Var}_{#1}\left[#2\right]}
\newcommand{\E}[2][]{\mathbf{E}_{#1}\left[#2\right]}
\renewcommand{\emptyset}{\varnothing}
\newcommand{\ol}{\overline}
\newcommand{\argmin}{\mathop{\arg\min}}
\newcommand{\argmax}{\mathop{\arg\max}}
\renewcommand{\abs}[1]{\qty|#1|}
\newcommand{\defeq}{\triangleq} % triangle over =
\def\deq{\xlongequal{def}} % 'def' over =
\def\LHS{\text{LHS}}
\def\RHS{\text{RHS}}
\def\angbr#1{\langle#1\rangle} % <x>
\def\set#1{\qty{#1}}

\def\Esolve{\textcolor{blue}{Solve: }}
\def\Eproof{\textcolor{blue}{Proof: }}
\def\case#1{\textcolor{blue}{Case \uppercase\expandafter{\romannumeral#1}: }}

% \newmdtheoremenv{lemma}{Lemma}
% \newmdtheoremenv{theorem}{\textcolor{red}{Theorem}}
% \newmdtheoremenv{defi}{\textcolor{blue}{Definition}}
\newtheorem{lemma}{Lemma}
\newtheorem{thm}{Theorem}
\newtheorem{defi}{Definition}
\newtheorem{prp}{Proposition}
\newtheorem{remark}{Remark}
\newenvironment{md}{\begin{mdframed}}{\end{mdframed}}

\graphicspath{{figures/}}

% \begin{document}
% \title{<++>}
\author{Haoyu Zhen}
% \maketitle
\setlength{\parindent}{0pt}
\setstretch{1.2}
% \end{document}

\usepackage{fancyhdr}
\pagestyle{fancy}
% \fancypagestyle{mainFancy}{
%     \fancyhf{}
%     \renewcommand\headrulewidth{.5pt}       % 页眉横线
%     \renewcommand\footrulewidth{0pt}
%     \fancyhead[OC]{\fzkai{\leftmark}}       % 页眉章标题
%     \fancyhead[EC]{\fzkai{\@title}}         % 页眉文章题目
% 	\lhead{\fzkai{author}}
%     \fancyhead[OR,EL]{\thepage}                 % 页眉编号
%     \fancyfoot[r]{\thumb} % 将拇指放到没有被使用的页眉或页脚处
% }
\fancyhf{}
\fancyfoot[C]{\thepage}
\fancyhead[R]{\slshape{AI2615 Algorithm Design and Analysis}}
\fancyhead[L]{\slshape{Haoyu Zhen}}

% % theorems
\usepackage{thmtools}
\usepackage{thm-restate}
\usepackage[framemethod=TikZ]{mdframed}
\mdfsetup{skipabove=1em,skipbelow=0em, innertopmargin=12pt, innerbottommargin=8pt}

\theoremstyle{definition}

\declaretheoremstyle[headfont=\bfseries\sffamily, bodyfont=\normalfont,
	mdframed={
		nobreak,
		backgroundcolor=brown!14,
		topline=false,
		rightline=false,
		leftline=true,
		bottomline=false,
		linewidth=2pt,
		linecolor=brown!180,
	}
]{thmbrownbox}

\declaretheoremstyle[headfont=\bfseries\sffamily, bodyfont=\normalfont,
	mdframed={
		nobreak,
		backgroundcolor=Blue!4,
		topline=false,
		rightline=false,
		leftline=true,
		bottomline=false,
		linewidth=2pt,
		linecolor=NavyBlue!120,
	}
]{thmbluebox}

\declaretheoremstyle[headfont=\bfseries\sffamily, bodyfont=\normalfont,
	mdframed={
		nobreak,
		backgroundcolor=Green!5,
		topline=false,
		rightline=false,
		leftline=true,
		bottomline=false,
		linewidth=2pt,
		linecolor=OliveGreen!120,
	}
]{thmgreenbox}

\declaretheoremstyle[headfont=\bfseries\sffamily, bodyfont=\normalfont,
	mdframed={
		nobreak,
		topline=false,
		rightline=false,
		leftline=true,
		bottomline=false,
		linewidth=2pt,
		linecolor=OliveGreen!120,
	}
]{thmgreenline}

\declaretheoremstyle[headfont=\bfseries\sffamily, bodyfont=\normalfont,
	mdframed={
		nobreak,
		topline=false,
		rightline=false,
		leftline=true,
		bottomline=false,
		linewidth=2pt,
		linecolor=NavyBlue!70,
	}
]{thmblueline}

\declaretheorem[numberwithin=section, style=thmbrownbox, name={\color{Brown}Definition}]{defi}
\declaretheorem[numberwithin=section, style=thmgreenbox, name={\color{OliveGreen}Law}]{law}
\declaretheorem[numberwithin=section, style=thmbluebox, name={\color{Blue}Corollary}]{cor}
\declaretheorem[numberwithin=section, style=thmgreenline, name={\color{OliveGreen}Property}]{prt}
\declaretheorem[numberwithin=section, style=thmbluebox, name={\color{Blue}Proposition}]{prp}
\declaretheorem[numberwithin=section, style=thmbluebox, name={\color{Blue}Theorem}]{thm}
\declaretheorem[numberwithin=section, style=thmbluebox, name={\color{Blue}Lemma}]{lemma}
\declaretheorem[numberwithin=section, style=thmbrownbox,  name={\color{Brown}Example}]{eg}
\declaretheorem[numberwithin=section, style=thmgreenline, name={\color{OliveGreen}Remark}]{remark}
\declaretheorem[numbered=no,style=thmblueline, name={\color{NavyBlue!70}Proof},qed=$\square$]{prf}
\numberwithin{equation}{section}


% %! Tex program = xelatex
% \documentclass{article}
%中文
%\usepackage[UTF8]{ctex}
%数学公式
\usepackage{amsmath,amssymb}
%\usepackage{ntheorem}
% \usepackage[framemethod=TikZ]{mdframed}
\usepackage{amsthm}
%边界
\usepackage[letterpaper,top=2cm,bottom=3cm,left=2cm,right=2cm,marginparwidth=1.75cm]{geometry}%table package
%Table
\usepackage{multirow,booktabs}
\usepackage{makecell}
%字体颜色
\usepackage{color}
% \usepackage[dvipsnames]{xcolor}  % 更全的色系
%代码
\usepackage[OT1]{fontenc}
% MATLAB 代码风格
%\usepackage[framed,numbered,autolinebreaks,useliterate]{/Users/anye_zhenhaoyu/Desktop/Latex/mcode}
\usepackage{listings}
\usepackage{algorithm}
\usepackage{algorithmic}
\usepackage{pythonhighlight} % Python
%插图
\usepackage{graphicx}
%改变item格式
\usepackage{enumerate}
%物理
\usepackage{physics}
%extra arrows
\usepackage{extarrows}
% caption(居中指令)
%\usepackage[justification=centering]{caption}
\usepackage{caption}
% htpb
\usepackage{stfloats}
% pdf 拼接
\usepackage{pdfpages}
% 超链接url
\usepackage{url}
\usepackage{tikz}
\usepackage{pgfplots}
\pgfplotsset{compat=newest}
\usepackage[colorlinks=true, allcolors=red]{hyperref}
\usepackage{setspace}
\usepackage{bbm}

% --------------definations-------------- %
\def\*#1{\boldsymbol{#1}}
\def\+#1{\mathcal{#1}} 
\def\-#1{\mathrm{#1}}
\def\rm#1{\mathrm{#1}}
\def\=#1{\mathbb{#1}}
% Domains
\def\RR{\mathbb{R}}
\def\EE{\mathbb{E}}
\def\CC{\mathbb{C}}
\def\NN{\mathbb{N}}
\def\ZZ{\mathbb{Z}}
% Newcommand
\newcommand{\inner}[2]{\langle #1,#2\rangle} 
\newcommand{\numP}{\#\mathbf{P}} 
\renewcommand{\P}{\mathbf{P}}
\newcommand{\Var}[2][]{\mathbf{Var}_{#1}\left[#2\right]}
\newcommand{\E}[2][]{\mathbf{E}_{#1}\left[#2\right]}
\renewcommand{\emptyset}{\varnothing}
\newcommand{\ol}{\overline}
\newcommand{\argmin}{\mathop{\arg\min}}
\newcommand{\argmax}{\mathop{\arg\max}}
\renewcommand{\abs}[1]{\qty|#1|}
\newcommand{\defeq}{\triangleq} % triangle over =
\def\deq{\xlongequal{def}} % 'def' over =
\def\LHS{\text{LHS}}
\def\RHS{\text{RHS}}
\def\angbr#1{\langle#1\rangle} % <x>
\def\set#1{\qty{#1}}

\def\Esolve{\textcolor{blue}{Solve: }}
\def\Eproof{\textcolor{blue}{Proof: }}
\def\case#1{\textcolor{blue}{Case \uppercase\expandafter{\romannumeral#1}: }}

% \newmdtheoremenv{lemma}{Lemma}
% \newmdtheoremenv{theorem}{\textcolor{red}{Theorem}}
% \newmdtheoremenv{defi}{\textcolor{blue}{Definition}}
\newtheorem{lemma}{Lemma}
\newtheorem{thm}{Theorem}
\newtheorem{defi}{Definition}
\newtheorem{prp}{Proposition}
\newtheorem{remark}{Remark}
\newenvironment{md}{\begin{mdframed}}{\end{mdframed}}

\graphicspath{{figures/}}

% \begin{document}
% \title{<++>}
\author{Haoyu Zhen}
% \maketitle
\setlength{\parindent}{0pt}
\setstretch{1.2}
% \end{document}

\usepackage{fancyhdr}
\pagestyle{fancy}
% \fancypagestyle{mainFancy}{
%     \fancyhf{}
%     \renewcommand\headrulewidth{.5pt}       % 页眉横线
%     \renewcommand\footrulewidth{0pt}
%     \fancyhead[OC]{\fzkai{\leftmark}}       % 页眉章标题
%     \fancyhead[EC]{\fzkai{\@title}}         % 页眉文章题目
% 	\lhead{\fzkai{author}}
%     \fancyhead[OR,EL]{\thepage}                 % 页眉编号
%     \fancyfoot[r]{\thumb} % 将拇指放到没有被使用的页眉或页脚处
% }
\fancyhf{}
\fancyfoot[C]{\thepage}
\fancyhead[R]{\slshape{AI2615 Algorithm Design and Analysis}}
\fancyhead[L]{\slshape{Haoyu Zhen}}

% % theorems
\usepackage{thmtools}
\usepackage{thm-restate}
\usepackage[framemethod=TikZ]{mdframed}
\mdfsetup{skipabove=1em,skipbelow=0em, innertopmargin=12pt, innerbottommargin=8pt}

\theoremstyle{definition}

\declaretheoremstyle[headfont=\bfseries\sffamily, bodyfont=\normalfont,
	mdframed={
		nobreak,
		backgroundcolor=brown!14,
		topline=false,
		rightline=false,
		leftline=true,
		bottomline=false,
		linewidth=2pt,
		linecolor=brown!180,
	}
]{thmbrownbox}

\declaretheoremstyle[headfont=\bfseries\sffamily, bodyfont=\normalfont,
	mdframed={
		nobreak,
		backgroundcolor=Blue!4,
		topline=false,
		rightline=false,
		leftline=true,
		bottomline=false,
		linewidth=2pt,
		linecolor=NavyBlue!120,
	}
]{thmbluebox}

\declaretheoremstyle[headfont=\bfseries\sffamily, bodyfont=\normalfont,
	mdframed={
		nobreak,
		backgroundcolor=Green!5,
		topline=false,
		rightline=false,
		leftline=true,
		bottomline=false,
		linewidth=2pt,
		linecolor=OliveGreen!120,
	}
]{thmgreenbox}

\declaretheoremstyle[headfont=\bfseries\sffamily, bodyfont=\normalfont,
	mdframed={
		nobreak,
		topline=false,
		rightline=false,
		leftline=true,
		bottomline=false,
		linewidth=2pt,
		linecolor=OliveGreen!120,
	}
]{thmgreenline}

\declaretheoremstyle[headfont=\bfseries\sffamily, bodyfont=\normalfont,
	mdframed={
		nobreak,
		topline=false,
		rightline=false,
		leftline=true,
		bottomline=false,
		linewidth=2pt,
		linecolor=NavyBlue!70,
	}
]{thmblueline}

\declaretheorem[numberwithin=section, style=thmbrownbox, name={\color{Brown}Definition}]{defi}
\declaretheorem[numberwithin=section, style=thmgreenbox, name={\color{OliveGreen}Law}]{law}
\declaretheorem[numberwithin=section, style=thmbluebox, name={\color{Blue}Corollary}]{cor}
\declaretheorem[numberwithin=section, style=thmgreenline, name={\color{OliveGreen}Property}]{prt}
\declaretheorem[numberwithin=section, style=thmbluebox, name={\color{Blue}Proposition}]{prp}
\declaretheorem[numberwithin=section, style=thmbluebox, name={\color{Blue}Theorem}]{thm}
\declaretheorem[numberwithin=section, style=thmbluebox, name={\color{Blue}Lemma}]{lemma}
\declaretheorem[numberwithin=section, style=thmbrownbox,  name={\color{Brown}Example}]{eg}
\declaretheorem[numberwithin=section, style=thmgreenline, name={\color{OliveGreen}Remark}]{remark}
\declaretheorem[numbered=no,style=thmblueline, name={\color{NavyBlue!70}Proof},qed=$\square$]{prf}
\numberwithin{equation}{section}


% On my MAC's Desktop
%! Tex program = xelatex
% \documentclass{article}
%中文
%\usepackage[UTF8]{ctex}
%数学公式
\usepackage{amsmath,amssymb}
%\usepackage{ntheorem}
% \usepackage[framemethod=TikZ]{mdframed}
\usepackage{amsthm}
%边界
\usepackage[letterpaper,top=2cm,bottom=3cm,left=2cm,right=2cm,marginparwidth=1.75cm]{geometry}%table package
%Table
\usepackage{multirow,booktabs}
\usepackage{makecell}
%字体颜色
\usepackage{color}
% \usepackage[dvipsnames]{xcolor}  % 更全的色系
%代码
\usepackage[OT1]{fontenc}
% MATLAB 代码风格
%\usepackage[framed,numbered,autolinebreaks,useliterate]{/Users/anye_zhenhaoyu/Desktop/Latex/mcode}
\usepackage{listings}
\usepackage{algorithm}
\usepackage{algorithmic}
\usepackage{pythonhighlight} % Python
%插图
\usepackage{graphicx}
%改变item格式
\usepackage{enumerate}
%物理
\usepackage{physics}
%extra arrows
\usepackage{extarrows}
% caption(居中指令)
%\usepackage[justification=centering]{caption}
\usepackage{caption}
% htpb
\usepackage{stfloats}
% pdf 拼接
\usepackage{pdfpages}
% 超链接url
\usepackage{url}
\usepackage{tikz}
\usepackage{pgfplots}
\pgfplotsset{compat=newest}
\usepackage[colorlinks=true, allcolors=red]{hyperref}
\usepackage{setspace}
\usepackage{bbm}

% --------------definations-------------- %
\def\*#1{\boldsymbol{#1}}
\def\+#1{\mathcal{#1}} 
\def\-#1{\mathrm{#1}}
\def\rm#1{\mathrm{#1}}
\def\=#1{\mathbb{#1}}
% Domains
\def\RR{\mathbb{R}}
\def\EE{\mathbb{E}}
\def\CC{\mathbb{C}}
\def\NN{\mathbb{N}}
\def\ZZ{\mathbb{Z}}
% Newcommand
\newcommand{\inner}[2]{\langle #1,#2\rangle} 
\newcommand{\numP}{\#\mathbf{P}} 
\renewcommand{\P}{\mathbf{P}}
\newcommand{\Var}[2][]{\mathbf{Var}_{#1}\left[#2\right]}
\newcommand{\E}[2][]{\mathbf{E}_{#1}\left[#2\right]}
\renewcommand{\emptyset}{\varnothing}
\newcommand{\ol}{\overline}
\newcommand{\argmin}{\mathop{\arg\min}}
\newcommand{\argmax}{\mathop{\arg\max}}
\renewcommand{\abs}[1]{\qty|#1|}
\newcommand{\defeq}{\triangleq} % triangle over =
\def\deq{\xlongequal{def}} % 'def' over =
\def\LHS{\text{LHS}}
\def\RHS{\text{RHS}}
\def\angbr#1{\langle#1\rangle} % <x>
\def\set#1{\qty{#1}}

\def\Esolve{\textcolor{blue}{Solve: }}
\def\Eproof{\textcolor{blue}{Proof: }}
\def\case#1{\textcolor{blue}{Case \uppercase\expandafter{\romannumeral#1}: }}

% \newmdtheoremenv{lemma}{Lemma}
% \newmdtheoremenv{theorem}{\textcolor{red}{Theorem}}
% \newmdtheoremenv{defi}{\textcolor{blue}{Definition}}
\newtheorem{lemma}{Lemma}
\newtheorem{thm}{Theorem}
\newtheorem{defi}{Definition}
\newtheorem{prp}{Proposition}
\newtheorem{remark}{Remark}
\newenvironment{md}{\begin{mdframed}}{\end{mdframed}}

\graphicspath{{figures/}}

% \begin{document}
% \title{<++>}
\author{Haoyu Zhen}
% \maketitle
\setlength{\parindent}{0pt}
\setstretch{1.2}
% \end{document}

\usepackage{fancyhdr}
\pagestyle{fancy}
% \fancypagestyle{mainFancy}{
%     \fancyhf{}
%     \renewcommand\headrulewidth{.5pt}       % 页眉横线
%     \renewcommand\footrulewidth{0pt}
%     \fancyhead[OC]{\fzkai{\leftmark}}       % 页眉章标题
%     \fancyhead[EC]{\fzkai{\@title}}         % 页眉文章题目
% 	\lhead{\fzkai{author}}
%     \fancyhead[OR,EL]{\thepage}                 % 页眉编号
%     \fancyfoot[r]{\thumb} % 将拇指放到没有被使用的页眉或页脚处
% }
\fancyhf{}
\fancyfoot[C]{\thepage}
\fancyhead[R]{\slshape{AI2615 Algorithm Design and Analysis}}
\fancyhead[L]{\slshape{Haoyu Zhen}}


\graphicspath{{figures/}}

\begin{document}
% \tableofcontents
\title{Homework 4}
\maketitle

\section*{Problem 1}
\subsection*{(a)}
Intuitively, I design the algorithm \ref{11}.
\begin{algorithm}[htbp]
	\caption{Find the maximum $(1,1)$-step subsequence}
	\label{11}
	\begin{algorithmic}[1]
		\renewcommand{\algorithmicrequire}{\textbf{Input:}}
		\renewcommand{\algorithmicensure}{\textbf{Output:}}
		\renewcommand{\algorithmiccomment}[1]{\hfill\textit{\textcolor{blue}{\##1}}}
		\REQUIRE A sequence of integers $a_1,\cdots,a_n$
		\ENSURE The maximu revenue we can get from a $(1,1)$-step subsequence
		\STATE $\+R\gets 0$, $M_{\mathrm{cur}}\gets 0$, $i\gets 1$
		\WHILE{$i\le n$}
		\STATE $M_{\mathrm{cur}}\gets M_{\mathrm{cur}}+a_i$
		\IF{$M_{\mathrm{cur}}<0$}
		\STATE $M_{\mathrm{cur}}\gets 0$
		\ENDIF
		\IF{$M_{\mathrm{cur}}>\+M$}
		\STATE $\+R\gets M_{\mathrm{cur}}$
		\ENDIF
		\STATE $i\gets i+1$
		\ENDWHILE
		\RETURN $\+R$
	\end{algorithmic}
\end{algorithm}
Ostensively, the complexity of the algorithm is $\+O(n)$.

The soundness of Algo. \ref{11}:
\begin{itemize}
	\item 
		The interpretation of $M_{\mathrm{cur}}^i$ at $a_i$: the maximum ravenue which \textbf{ends} at $a_i$. 
	\item
		The propostion above holds by mathematical induction. $M_{\mathrm{cur}}^i=\max(0,M_{\mathrm{cur}}^{i-1}+a_i)$.
	\item
		Thus $\+R=\max_iM_{\mathrm{cur}}^i$ which is what we want.
\end{itemize}

\subsection*{(b)}
Similarly, we have algorithm \ref{lr}.
\begin{algorithm}[htbp]
	\caption{Find the maximum $(L,R)$-step subsequence}
	\label{lr}
	\begin{algorithmic}[1]
		\renewcommand{\algorithmicrequire}{\textbf{Input:}}
		\renewcommand{\algorithmicensure}{\textbf{Output:}}
		\renewcommand{\algorithmiccomment}[1]{\hfill\textit{\textcolor{blue}{\##1}}}
		\REQUIRE A sequence of integers $a_1,\cdots,a_n$
		\ENSURE The maximu revenue we can get from a $(L,R)$-step subsequence
		\STATE $\*M\defeq(M_1,M_2,..,M_n)\gets\*0$
		\STATE $M_i\gets\max(a_i,0)$ for all $i\in[L]$
		\FOR{$i=(L+1,L+2,\cdots,n)$}
		\STATE $M_i=a_i+\max_{j\in\{i-R,\cdots,i-L\}}M_j$ 
		\COMMENT{Complexity $\+O(R-L)=\+O(n)$}
		\STATE $M_i\gets \max(M_i,0)$
		\ENDFOR
		\RETURN $\max_{i\in[n]}M_i$
	\end{algorithmic} 
\end{algorithm}

The correctness of Algo \ref{lr} is almost same with that of Algo \ref{11}. The complexity is obviously $\+O(n^2)$.

\subsection*{(c)}
Design the algorithm \ref{lr:pro} with the auxiliary of the \textit{Priority Queue}.
\begin{algorithm}[htbp]
	\caption{Find the maximum $(L,R)$-step subsequence (\textbf{Pro})}
	\label{lr:pro}
	\begin{algorithmic}[1]
		\renewcommand{\algorithmicrequire}{\textbf{Input:}}
		\renewcommand{\algorithmicensure}{\textbf{Output:}}
		\renewcommand{\algorithmiccomment}[1]{\hfill\textit{\textcolor{blue}{\##1}}}
		\REQUIRE A sequence of integers $a_1,\cdots,a_n$
		\ENSURE The maximu revenue we can get from a $(L,R)$-step subsequence
		\STATE $\*M\defeq(M_1,M_2,..,M_n)\gets\*0$
		\STATE $M_i\gets\max(a_i,0)$ for all $i\in[L]$
		\STATE $\mathrm{PLL}=\mathrm{NULL}$
		\FOR{$i=(L+1,L+2,\cdots,n)$}
		\IF{PLL.front.index$\le i-R$}
		\STATE PLL.PopFront
		\ENDIF
		\WHILE{PLL.back.value$\le M_{i-L}$}
		\STATE PLL.PopBack
		\ENDWHILE
		\STATE PLL.PushBack($M_{i-L}$)
		\STATE $M_i=a_i+\mathrm{PLL.Front}$
		\STATE $M_i\gets \max(M_i,0)$
		\ENDFOR
		\RETURN $\max_{i\in[n]}M_i$
	\end{algorithmic} 
\end{algorithm}

The complexity is $\+O(n)$ because every element in $a$ will be inserted into PLL once.

The soundness of Algo \ref{lr:pro} intrinsicly hold by the fact that PLL.Front at $i$ is  $\max_{j\in\{i-R,\cdots,i-L\}}M_j$ and $M_i$ is the maximum ravenue which \textbf{ends} at $a_i$.

\section*{Problem 2}

First we define the function $dp(i,j)$: the total number of comparisions for words  $a_i,\cdots,a_j$.
The transition equation is: 
\[
	dp(i,j) = \sum^{j}_{k=i}w_i + \max_{i+1\le k\le j-1} dp(i,k-1)+dp(k+1,j)
.\]
Proof: if $k$ is the best BST's root for $i$ to $j$, we obtain $\mathrm{cost}=\sum^{j}_{k=i}w_i + dp(i,k-1)+dp(k+1,j)$.

Then we have the naive algorithm:
\begin{algorithm}[H]
	\caption{Find the best BST for the $n$ words with the minimum number of comparisons}
	\label{bst}
	\begin{algorithmic}[1]
		\renewcommand{\algorithmicrequire}{\textbf{Input:}}
		\renewcommand{\algorithmicensure}{\textbf{Output:}}
		\renewcommand{\algorithmiccomment}[1]{\hfill\textit{\textcolor{blue}{\##1}}}
		\REQUIRE A sequence $\*w=(w_1,w_2,\cdots,w_n)$
		\STATE Initiate two 2D arrays: $\*n$ and $\*{\mathrm{idx}}$
		\STATE $n_{i,i}=w_i$ and  $\mathrm{idx}_{i,i}=i$
		\FOR{$i=n-1,n-2\cdots,1$}
		\FOR{$j=i+1,i+2,\cdots,n$}
		\STATE $i\gets t$ and $j\gets k+t-1$\\[3pt]
		\STATE $\*n(i,j)=\sum^{j}_{k=i}w_i +
				\min_{[i+1\le k\le j-1]} \qty[\*n(i,k-1)+\*n(k+1,j)]$
				\\[3pt]
		\STATE $\*{\mathrm{idx}}\gets\argmin_{[i+1\le k\le j-1]}\*n(i,k-1)+\*n(k+1,j))$
		\ENDFOR
		\ENDFOR
	\end{algorithmic} 
\end{algorithm}
Now we could construct the tree with $\*{\mathrm{idx}}$:
\begin{enumerate}
	\item 
		Let the root be $\*{\mathrm{idx}}[1,n]$.
	\item
		If $k$ is the root of the BST for $a_i,\cdots,a_j$, then $k$'s left child is  $\*{\mathrm{idx}}[i,k-1]$ and its right child is $\*{\mathrm{idx}}[k+1,j]$.
	\item
		Do step 2 \textbf{recusively}.
\end{enumerate}
The complexity is $\+O(n^3)$. The soundness of the algorithm depends on that of the transition equation.

\section*{Problem 3}
For given string $s$: $\bar{s}_{i,j}\defeq s[i:j]$ which is a consecutive subsequencee of $s$. Then we define $f(i,j)$ as the longest palindrome with length $l(i,j)$ that is a subsequence of  $\bar{s}_{i,j}$.
Then
\begin{equation}
	\label{eq:pal}
	l(i,j)=\max\big[l(i+1,j),\,l(i,j-1),\,2\times1(s_i=s_j)+l(i+1,j-1)\big]
\end{equation}
And we update $f(i,j)$ correspondingly. The correctness of EQ \ref{eq:pal} is trivial. Here's the \textit{story}:
\begin{itemize}
	\item If $s_i\ne s_j$, then the lonest palindrome is in $\bar{s}_{i,j-1}$ or in  $\bar{s}_{i+1,j}$.
	\item If $s_i=s_j$, we have a new choice: $\mathrm{append}(s_i,s',s_j)$ where $s'$ is in $\bar{s}_{i+1,j-1}$.
\end{itemize}
Hence we design the algorithm \ref{pal}.
\begin{algorithm}[htbp]
	\caption{find the longest palindrome that is a subsequence of a string}
	\label{pal}
	\begin{algorithmic}[1]
		\renewcommand{\algorithmicrequire}{\textbf{Input:}}
		\renewcommand{\algorithmicensure}{\textbf{Output:}}
		\renewcommand{\algorithmiccomment}[1]{\hfill\textit{\textcolor{blue}{\##1}}}
		\REQUIRE A string $s$
		\STATE Initiate 2 2D arrays: $\*l$ and $\*f$
		\STATE $l_{i,i}=1$, $f_{i,i}=s_i$ and $l_{i,j}=0$, $f(i,j)=\emptyset$ for all  $i>j$
		\FOR{$i=n-1,n-2\cdots,1$}
		\FOR{$j=i+1,i+2,\cdots,n$}
		\STATE Update l
		\ENDFOR
		\ENDFOR
		\RETURN $f(1,n)$
	\end{algorithmic}
\end{algorithm}

The running time is $\+O(n^2)$.

\section*{Problem 4}
For any given tree $G$ with root $r$, denote $f(r)$ as the number of independent sets in $G$.
Then
\[
	f(r)
	=
	\prod_{v\in\{r\text{'s children}\}} f(v)
	+
	\prod_{v\in\{r\text{'s grandchildren}\}} f(v)
.\]
The first term of $\LHS$ represents the number of independent sets under the case that $r$ is not in these sets.
Choosing $r$, we will obtain the latter term.

Now we design the algorithm \ref{cni}.
\begin{algorithm}[htbp]
	\caption{Count the number of independent sets in a tree}
	\label{cni}
	\begin{algorithmic}[1]
		\renewcommand{\algorithmicrequire}{\textbf{Input:}}
		\renewcommand{\algorithmicensure}{\textbf{Output:}}
		\renewcommand{\algorithmiccomment}[1]{\hfill\textit{\textcolor{blue}{\##1}}}
		\REQUIRE A tree $G$ with a arbitrary root $r$
		\STATE Implement BFS from $r$. Assign the depth, parent and children to every vertex. And we could get a sequence $\*v=(v_1,\cdots,v_n)$ where vertices is sorted by their depth.
		\FOR{$u=(v_1,v_2,\cdots,v_n)$}
		\IF{$u$ is a leaf}
		\STATE $f(u)=1$
		\ELSIF{$u$ has no grandchild}
		\STATE $f(u)=\prod_{v\in\{u\text{'s children}\}} f(v)$
		\ELSE
		\STATE Update $f(u)$ by the transition EQ.
		\ENDIF
		\ENDFOR
		\RETURN 
	\end{algorithmic} 
\end{algorithm}

The complexity is $\+O(n)$ because every vertex will be mutiplied at most twice.

\section*{Problem 5}
About a day.
Difficulty 3.
Collaborators: Yinlin Sun.
\end{document}

