% ! Tex program = xelatex
\documentclass{article}

%! Tex program = xelatex
% \documentclass{article}
%中文
%\usepackage[UTF8]{ctex}
%数学公式
\usepackage{amsmath,amssymb}
%\usepackage{ntheorem}
% \usepackage[framemethod=TikZ]{mdframed}
\usepackage{amsthm}
%边界
\usepackage[letterpaper,top=2cm,bottom=3cm,left=2cm,right=2cm,marginparwidth=1.75cm]{geometry}%table package
%Table
\usepackage{multirow,booktabs}
\usepackage{makecell}
%字体颜色
\usepackage{color}
% \usepackage[dvipsnames]{xcolor}  % 更全的色系
%代码
\usepackage[OT1]{fontenc}
% MATLAB 代码风格
%\usepackage[framed,numbered,autolinebreaks,useliterate]{/Users/anye_zhenhaoyu/Desktop/Latex/mcode}
\usepackage{listings}
\usepackage{algorithm}
\usepackage{algorithmic}
\usepackage{pythonhighlight} % Python
%插图
\usepackage{graphicx}
%改变item格式
\usepackage{enumerate}
%物理
\usepackage{physics}
%extra arrows
\usepackage{extarrows}
% caption(居中指令)
%\usepackage[justification=centering]{caption}
\usepackage{caption}
% htpb
\usepackage{stfloats}
% pdf 拼接
\usepackage{pdfpages}
% 超链接url
\usepackage{url}
\usepackage{tikz}
\usepackage{pgfplots}
\pgfplotsset{compat=newest}
\usepackage[colorlinks=true, allcolors=red]{hyperref}
\usepackage{setspace}
\usepackage{bbm}

% --------------definations-------------- %
\def\*#1{\boldsymbol{#1}}
\def\+#1{\mathcal{#1}} 
\def\-#1{\mathrm{#1}}
\def\rm#1{\mathrm{#1}}
\def\=#1{\mathbb{#1}}
% Domains
\def\RR{\mathbb{R}}
\def\EE{\mathbb{E}}
\def\CC{\mathbb{C}}
\def\NN{\mathbb{N}}
\def\ZZ{\mathbb{Z}}
% Newcommand
\newcommand{\inner}[2]{\langle #1,#2\rangle} 
\newcommand{\numP}{\#\mathbf{P}} 
\renewcommand{\P}{\mathbf{P}}
\newcommand{\Var}[2][]{\mathbf{Var}_{#1}\left[#2\right]}
\newcommand{\E}[2][]{\mathbf{E}_{#1}\left[#2\right]}
\renewcommand{\emptyset}{\varnothing}
\newcommand{\ol}{\overline}
\newcommand{\argmin}{\mathop{\arg\min}}
\newcommand{\argmax}{\mathop{\arg\max}}
\renewcommand{\abs}[1]{\qty|#1|}
\newcommand{\defeq}{\triangleq} % triangle over =
\def\deq{\xlongequal{def}} % 'def' over =
\def\LHS{\text{LHS}}
\def\RHS{\text{RHS}}
\def\angbr#1{\langle#1\rangle} % <x>
\def\set#1{\qty{#1}}

\def\Esolve{\textcolor{blue}{Solve: }}
\def\Eproof{\textcolor{blue}{Proof: }}
\def\case#1{\textcolor{blue}{Case \uppercase\expandafter{\romannumeral#1}: }}

% \newmdtheoremenv{lemma}{Lemma}
% \newmdtheoremenv{theorem}{\textcolor{red}{Theorem}}
% \newmdtheoremenv{defi}{\textcolor{blue}{Definition}}
\newtheorem{lemma}{Lemma}
\newtheorem{thm}{Theorem}
\newtheorem{defi}{Definition}
\newtheorem{prp}{Proposition}
\newtheorem{remark}{Remark}
\newenvironment{md}{\begin{mdframed}}{\end{mdframed}}

\graphicspath{{figures/}}

% \begin{document}
% \title{<++>}
\author{Haoyu Zhen}
% \maketitle
\setlength{\parindent}{0pt}
\setstretch{1.2}
% \end{document}

\usepackage{fancyhdr}
\pagestyle{fancy}
% \fancypagestyle{mainFancy}{
%     \fancyhf{}
%     \renewcommand\headrulewidth{.5pt}       % 页眉横线
%     \renewcommand\footrulewidth{0pt}
%     \fancyhead[OC]{\fzkai{\leftmark}}       % 页眉章标题
%     \fancyhead[EC]{\fzkai{\@title}}         % 页眉文章题目
% 	\lhead{\fzkai{author}}
%     \fancyhead[OR,EL]{\thepage}                 % 页眉编号
%     \fancyfoot[r]{\thumb} % 将拇指放到没有被使用的页眉或页脚处
% }
\fancyhf{}
\fancyfoot[C]{\thepage}
\fancyhead[R]{\slshape{AI2615 Algorithm Design and Analysis}}
\fancyhead[L]{\slshape{Haoyu Zhen}}


\graphicspath{{figures/}}

\begin{document}

\title{Homework 6}
\maketitle

\section*{(b)}
Let $f$ whose inputs are 2 graphs $H$ and $G$ denote the subgraph problem.
\begin{itemize}
	\item $f\in\text{NP}$

		Let $\+M$ be a injective mapping from $V_H$ to $V_G$. Then the string $y$ representing $\+M$ is the certificate in a Turning machine $\+A$.
		If for all $u,v\in V_H$ such that $\{u,v\}\in E_H\implies \{\+M(u),\+M(v)\}\in E_G$ and $\{u,v\}\notin E_H\implies\{\+M(u),\+M(v)\}\notin E_G$, then $\+A(x,y)=1$ (note taht $\+A(x,y)=0$ otherwise).
		This could be done in $\+O(\abs{E})$ time.

	\item $\text{Clique}\,\le_k f$

		Obviously, a loop containing $k$ vetices is a subgraph  $G$ \textbf{iff} $G$ has a $k$-clique.
\end{itemize}

Finally, we obtain that $f$ is NP-complete.

\section*{(e)}
Here we face the ``SubsetSum0'' problem. In the lecture we've already proved that SubsetSum+ is NP-complete.

 \begin{itemize}
	 \item Obviously, $\text{SubsetSum0}\,\in\text{NP}$ because addition operation is petty.
	 \item
		 $\text{SubsetSum}+ \le_k \text{SubsetSum}0$ (Indeed, trivial proposition)

		 \texttt{Proof:}
		 Let $S$ denote the set of positive integers. We want to decide whether $\exists A\subseteq S$ such that $\sum_{a\in A}=k$.
		 Now $S^\dagger \gets S\cup\{-k\}$ and feed $S$ into  SubsetSum0 algorithm. We could get a set $B$ such that $\sum_{b\in B}b=0$. Note that $-k\in B$ and we have  $\sum_{a\in B-\{-k\}}a=k$.
\end{itemize}
Thus, ``SubsetSum0'' is NP-complete.

\section*{(f)}
$f$ denotes the decision problem whose inputs are a colored graph $G$ and a number $k$.
\begin{itemize}
	\item $f\in$NP

		For any given $G$ which is colored: we could check whether all vertices will eventually become black after updates in  $\+O(\abs{V}\abs{E})$ time.

	\item $g\defeq k\text{-Vertex-Cover}\,\le_k f$

		\texttt{Proof:} 
		Assume that $(G,k)$ is input of  $g$. I.e., we want to decide whether there exists a vertex cover whose size is  $k$ in $G$.
		Here we colour all vertices in $G$ \textbf{white} and feed it into $f$ with $k$.
		\begin{lemma}
			$u,v$ are white and  $(u,v)\in E$ $\implies$ $u,v$ will be white forever. (This is a stable structure.)
			\label{stable}
		\end{lemma}
		Naturally, lemma \ref{stable} entails the following lemma.
		\begin{lemma}
			If a coloured $G$ will end with a black one, then  $\forall (u,v)\in E$, $u$ is black or  $v$ is black, which means that $G$ is \textbf{covered} by balck vertices.
			\label{black}
		\end{lemma}
		
		Thus, $g(G,k)=f(G_{\mathrm{white}},k)$.
\end{itemize}

Hence we could deduce that $f$ is NP-complete.

\section*{Misc}
5-6 hours. Difficulty 2. No collaborator.
\end{document}
