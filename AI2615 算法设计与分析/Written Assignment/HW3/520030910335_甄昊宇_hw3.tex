% ! Tex program = xelatex
\documentclass{article}
\usepackage{pst-node}
% 中文
% \usepackage[UTF8]{ctex}

% For more choices
% %! Tex program = xelatex
% \documentclass{article}
%中文
%\usepackage[UTF8]{ctex}
%数学公式
\usepackage{amsmath,amssymb}
%\usepackage{ntheorem}
% \usepackage[framemethod=TikZ]{mdframed}
\usepackage{amsthm}
%边界
\usepackage[letterpaper,top=3cm,bottom=3cm,left=2cm,right=2cm,marginparwidth=1.75cm]{geometry}%table package
%Table
\usepackage{multirow,booktabs}
\usepackage{makecell}
%字体颜色
\usepackage{color}
\usepackage[dvipsnames]{xcolor}  % 更全的色系
%代码
\usepackage[OT1]{fontenc}
% MATLAB 代码风格
% \usepackage[framed,numbered,autolinebreaks,useliterate]{/Users/anye_zhenhaoyu/Desktop/Latex/mcode}
\usepackage{listings}
\usepackage{algorithm}
\usepackage{algorithmic}
\usepackage{pythonhighlight} % Python
%插图
\usepackage{graphicx}
%改变item格式
\usepackage{enumerate}
%物理
\usepackage{physics}
%extra arrows
\usepackage{extarrows}
% caption(居中指令)
\usepackage[justification=centering]{caption}
% \usepackage{caption}
% htpb
\usepackage{stfloats}
% pdf 拼接
\usepackage{pdfpages}
% 超链接url
\usepackage{url}
% \usepackage{tikz}
\usepackage{pgfplots}
\pgfplotsset{compat=newest}
\usepackage[colorlinks=true, allcolors=blue]{hyperref}
\usepackage{setspace}

% --------------definations-------------- %
\def\*#1{\boldsymbol{#1}}
\def\+#1{\mathcal{#1}} 
\def\-#1{\bar{#1}}
% Domains
\def\RR{\mathbb{R}}
\def\CC{\mathbb{C}}
\def\NN{\mathbb{N}}
\def\ZZ{\mathbb{Z}}
% Newcommand
\newcommand{\inner}[2]{\langle #1,#2\rangle} 
\newcommand{\numP}{\#\mathbf{P}} 
\renewcommand{\P}{\mathbf{P}}
\newcommand{\Var}[2][]{\mathbf{Var}_{#1}\left[#2\right]}
\newcommand{\E}[2][]{\mathbf{E}_{#1}\left[#2\right]}
\renewcommand{\emptyset}{\varnothing}
\newcommand{\ol}{\overline}
\newcommand{\argmin}{\mathop{\arg\min}}
\newcommand{\argmax}{\mathop{\arg\max}}
\renewcommand{\abs}[1]{\qty|#1|}
\newcommand{\defeq}{\triangleq} % triangle over =
\def\deq{\xlongequal{def}} % 'def' over =
\def\LHS{\text{LHS}}
\def\RHS{\text{RHS}}
\def\angbr#1{\langle#1\rangle} % <x>

\def\Esolve{\textcolor{blue}{Solve: }}
\def\Eproof{\textcolor{blue}{Proof: }}
\def\case#1{\textcolor{blue}{Case \uppercase\expandafter{\romannumeral#1}: }}

\newtheorem{lemma}{Lemma}
\newtheorem{thm}{Theorem}
\newtheorem{defi}{Definition}
\newtheorem{prp}{Proposition}
\newenvironment{md}{\begin{mdframed}}{\end{mdframed}}

\date{\today}
\usepackage{fancyhdr}
\pagestyle{fancy}
\fancyhead[L]{\slshape{Haoyu Zhen}}
\fancyhead[R]{{\today}}

% \begin{document}
% \title{<++>}
\author{Haoyu Zhen}
% \maketitle
\setlength{\parindent}{0pt}
\setstretch{1.2}
% \end{document}


% % theorems
\usepackage{thmtools}
\usepackage{thm-restate}
\usepackage[framemethod=TikZ]{mdframed}
\mdfsetup{skipabove=1em,skipbelow=0em, innertopmargin=12pt, innerbottommargin=8pt}

\theoremstyle{definition}

\declaretheoremstyle[headfont=\bfseries\sffamily, bodyfont=\normalfont,
	mdframed={
		nobreak,
		backgroundcolor=brown!14,
		topline=false,
		rightline=false,
		leftline=true,
		bottomline=false,
		linewidth=2pt,
		linecolor=brown!180,
	}
]{thmbrownbox}

\declaretheoremstyle[headfont=\bfseries\sffamily, bodyfont=\normalfont,
	mdframed={
		nobreak,
		backgroundcolor=Blue!4,
		topline=false,
		rightline=false,
		leftline=true,
		bottomline=false,
		linewidth=2pt,
		linecolor=NavyBlue!120,
	}
]{thmbluebox}

\declaretheoremstyle[headfont=\bfseries\sffamily, bodyfont=\normalfont,
	mdframed={
		nobreak,
		backgroundcolor=Green!5,
		topline=false,
		rightline=false,
		leftline=true,
		bottomline=false,
		linewidth=2pt,
		linecolor=OliveGreen!120,
	}
]{thmgreenbox}

\declaretheoremstyle[headfont=\bfseries\sffamily, bodyfont=\normalfont,
	mdframed={
		nobreak,
		topline=false,
		rightline=false,
		leftline=true,
		bottomline=false,
		linewidth=2pt,
		linecolor=OliveGreen!120,
	}
]{thmgreenline}

\declaretheoremstyle[headfont=\bfseries\sffamily, bodyfont=\normalfont,
	mdframed={
		nobreak,
		topline=false,
		rightline=false,
		leftline=true,
		bottomline=false,
		linewidth=2pt,
		linecolor=NavyBlue!70,
	}
]{thmblueline}

\declaretheorem[numberwithin=section, style=thmbrownbox, name={\color{Brown}Definition}]{defi}
\declaretheorem[numberwithin=section, style=thmgreenbox, name={\color{OliveGreen}Law}]{law}
\declaretheorem[numberwithin=section, style=thmbluebox, name={\color{Blue}Corollary}]{cor}
\declaretheorem[numberwithin=section, style=thmgreenline, name={\color{OliveGreen}Property}]{prt}
\declaretheorem[numberwithin=section, style=thmbluebox, name={\color{Blue}Proposition}]{prp}
\declaretheorem[numberwithin=section, style=thmbluebox, name={\color{Blue}Theorem}]{thm}
\declaretheorem[numberwithin=section, style=thmbluebox, name={\color{Blue}Lemma}]{lemma}
\declaretheorem[numberwithin=section, style=thmbrownbox,  name={\color{Brown}Example}]{eg}
\declaretheorem[numberwithin=section, style=thmgreenline, name={\color{OliveGreen}Remark}]{remark}
\declaretheorem[numbered=no,style=thmblueline, name={\color{NavyBlue!70}Proof},qed=$\square$]{prf}
\numberwithin{equation}{section}


% %! Tex program = xelatex
% \documentclass{article}
%中文
%\usepackage[UTF8]{ctex}
%数学公式
\usepackage{amsmath,amssymb}
%\usepackage{ntheorem}
% \usepackage[framemethod=TikZ]{mdframed}
\usepackage{amsthm}
%边界
\usepackage[letterpaper,top=3cm,bottom=3cm,left=2cm,right=2cm,marginparwidth=1.75cm]{geometry}%table package
%Table
\usepackage{multirow,booktabs}
\usepackage{makecell}
%字体颜色
\usepackage{color}
\usepackage[dvipsnames]{xcolor}  % 更全的色系
%代码
\usepackage[OT1]{fontenc}
% MATLAB 代码风格
% \usepackage[framed,numbered,autolinebreaks,useliterate]{/Users/anye_zhenhaoyu/Desktop/Latex/mcode}
\usepackage{listings}
\usepackage{algorithm}
\usepackage{algorithmic}
\usepackage{pythonhighlight} % Python
%插图
\usepackage{graphicx}
%改变item格式
\usepackage{enumerate}
%物理
\usepackage{physics}
%extra arrows
\usepackage{extarrows}
% caption(居中指令)
\usepackage[justification=centering]{caption}
% \usepackage{caption}
% htpb
\usepackage{stfloats}
% pdf 拼接
\usepackage{pdfpages}
% 超链接url
\usepackage{url}
% \usepackage{tikz}
\usepackage{pgfplots}
\pgfplotsset{compat=newest}
\usepackage[colorlinks=true, allcolors=blue]{hyperref}
\usepackage{setspace}

% --------------definations-------------- %
\def\*#1{\boldsymbol{#1}}
\def\+#1{\mathcal{#1}} 
\def\-#1{\bar{#1}}
% Domains
\def\RR{\mathbb{R}}
\def\CC{\mathbb{C}}
\def\NN{\mathbb{N}}
\def\ZZ{\mathbb{Z}}
% Newcommand
\newcommand{\inner}[2]{\langle #1,#2\rangle} 
\newcommand{\numP}{\#\mathbf{P}} 
\renewcommand{\P}{\mathbf{P}}
\newcommand{\Var}[2][]{\mathbf{Var}_{#1}\left[#2\right]}
\newcommand{\E}[2][]{\mathbf{E}_{#1}\left[#2\right]}
\renewcommand{\emptyset}{\varnothing}
\newcommand{\ol}{\overline}
\newcommand{\argmin}{\mathop{\arg\min}}
\newcommand{\argmax}{\mathop{\arg\max}}
\renewcommand{\abs}[1]{\qty|#1|}
\newcommand{\defeq}{\triangleq} % triangle over =
\def\deq{\xlongequal{def}} % 'def' over =
\def\LHS{\text{LHS}}
\def\RHS{\text{RHS}}
\def\angbr#1{\langle#1\rangle} % <x>

\def\Esolve{\textcolor{blue}{Solve: }}
\def\Eproof{\textcolor{blue}{Proof: }}
\def\case#1{\textcolor{blue}{Case \uppercase\expandafter{\romannumeral#1}: }}

\newtheorem{lemma}{Lemma}
\newtheorem{thm}{Theorem}
\newtheorem{defi}{Definition}
\newtheorem{prp}{Proposition}
\newenvironment{md}{\begin{mdframed}}{\end{mdframed}}

\date{\today}
\usepackage{fancyhdr}
\pagestyle{fancy}
\fancyhead[L]{\slshape{Haoyu Zhen}}
\fancyhead[R]{{\today}}

% \begin{document}
% \title{<++>}
\author{Haoyu Zhen}
% \maketitle
\setlength{\parindent}{0pt}
\setstretch{1.2}
% \end{document}


% % theorems
\usepackage{thmtools}
\usepackage{thm-restate}
\usepackage[framemethod=TikZ]{mdframed}
\mdfsetup{skipabove=1em,skipbelow=0em, innertopmargin=12pt, innerbottommargin=8pt}

\theoremstyle{definition}

\declaretheoremstyle[headfont=\bfseries\sffamily, bodyfont=\normalfont,
	mdframed={
		nobreak,
		backgroundcolor=brown!14,
		topline=false,
		rightline=false,
		leftline=true,
		bottomline=false,
		linewidth=2pt,
		linecolor=brown!180,
	}
]{thmbrownbox}

\declaretheoremstyle[headfont=\bfseries\sffamily, bodyfont=\normalfont,
	mdframed={
		nobreak,
		backgroundcolor=Blue!4,
		topline=false,
		rightline=false,
		leftline=true,
		bottomline=false,
		linewidth=2pt,
		linecolor=NavyBlue!120,
	}
]{thmbluebox}

\declaretheoremstyle[headfont=\bfseries\sffamily, bodyfont=\normalfont,
	mdframed={
		nobreak,
		backgroundcolor=Green!5,
		topline=false,
		rightline=false,
		leftline=true,
		bottomline=false,
		linewidth=2pt,
		linecolor=OliveGreen!120,
	}
]{thmgreenbox}

\declaretheoremstyle[headfont=\bfseries\sffamily, bodyfont=\normalfont,
	mdframed={
		nobreak,
		topline=false,
		rightline=false,
		leftline=true,
		bottomline=false,
		linewidth=2pt,
		linecolor=OliveGreen!120,
	}
]{thmgreenline}

\declaretheoremstyle[headfont=\bfseries\sffamily, bodyfont=\normalfont,
	mdframed={
		nobreak,
		topline=false,
		rightline=false,
		leftline=true,
		bottomline=false,
		linewidth=2pt,
		linecolor=NavyBlue!70,
	}
]{thmblueline}

\declaretheorem[numberwithin=section, style=thmbrownbox, name={\color{Brown}Definition}]{defi}
\declaretheorem[numberwithin=section, style=thmgreenbox, name={\color{OliveGreen}Law}]{law}
\declaretheorem[numberwithin=section, style=thmbluebox, name={\color{Blue}Corollary}]{cor}
\declaretheorem[numberwithin=section, style=thmgreenline, name={\color{OliveGreen}Property}]{prt}
\declaretheorem[numberwithin=section, style=thmbluebox, name={\color{Blue}Proposition}]{prp}
\declaretheorem[numberwithin=section, style=thmbluebox, name={\color{Blue}Theorem}]{thm}
\declaretheorem[numberwithin=section, style=thmbluebox, name={\color{Blue}Lemma}]{lemma}
\declaretheorem[numberwithin=section, style=thmbrownbox,  name={\color{Brown}Example}]{eg}
\declaretheorem[numberwithin=section, style=thmgreenline, name={\color{OliveGreen}Remark}]{remark}
\declaretheorem[numbered=no,style=thmblueline, name={\color{NavyBlue!70}Proof},qed=$\square$]{prf}
\numberwithin{equation}{section}


% On my MAC's Desktop
%! Tex program = xelatex
% \documentclass{article}
%中文
%\usepackage[UTF8]{ctex}
%数学公式
\usepackage{amsmath,amssymb}
%\usepackage{ntheorem}
% \usepackage[framemethod=TikZ]{mdframed}
\usepackage{amsthm}
%边界
\usepackage[letterpaper,top=3cm,bottom=3cm,left=2cm,right=2cm,marginparwidth=1.75cm]{geometry}%table package
%Table
\usepackage{multirow,booktabs}
\usepackage{makecell}
%字体颜色
\usepackage{color}
\usepackage[dvipsnames]{xcolor}  % 更全的色系
%代码
\usepackage[OT1]{fontenc}
% MATLAB 代码风格
% \usepackage[framed,numbered,autolinebreaks,useliterate]{/Users/anye_zhenhaoyu/Desktop/Latex/mcode}
\usepackage{listings}
\usepackage{algorithm}
\usepackage{algorithmic}
\usepackage{pythonhighlight} % Python
%插图
\usepackage{graphicx}
%改变item格式
\usepackage{enumerate}
%物理
\usepackage{physics}
%extra arrows
\usepackage{extarrows}
% caption(居中指令)
\usepackage[justification=centering]{caption}
% \usepackage{caption}
% htpb
\usepackage{stfloats}
% pdf 拼接
\usepackage{pdfpages}
% 超链接url
\usepackage{url}
% \usepackage{tikz}
\usepackage{pgfplots}
\pgfplotsset{compat=newest}
\usepackage[colorlinks=true, allcolors=blue]{hyperref}
\usepackage{setspace}

% --------------definations-------------- %
\def\*#1{\boldsymbol{#1}}
\def\+#1{\mathcal{#1}} 
\def\-#1{\bar{#1}}
% Domains
\def\RR{\mathbb{R}}
\def\CC{\mathbb{C}}
\def\NN{\mathbb{N}}
\def\ZZ{\mathbb{Z}}
% Newcommand
\newcommand{\inner}[2]{\langle #1,#2\rangle} 
\newcommand{\numP}{\#\mathbf{P}} 
\renewcommand{\P}{\mathbf{P}}
\newcommand{\Var}[2][]{\mathbf{Var}_{#1}\left[#2\right]}
\newcommand{\E}[2][]{\mathbf{E}_{#1}\left[#2\right]}
\renewcommand{\emptyset}{\varnothing}
\newcommand{\ol}{\overline}
\newcommand{\argmin}{\mathop{\arg\min}}
\newcommand{\argmax}{\mathop{\arg\max}}
\renewcommand{\abs}[1]{\qty|#1|}
\newcommand{\defeq}{\triangleq} % triangle over =
\def\deq{\xlongequal{def}} % 'def' over =
\def\LHS{\text{LHS}}
\def\RHS{\text{RHS}}
\def\angbr#1{\langle#1\rangle} % <x>

\def\Esolve{\textcolor{blue}{Solve: }}
\def\Eproof{\textcolor{blue}{Proof: }}
\def\case#1{\textcolor{blue}{Case \uppercase\expandafter{\romannumeral#1}: }}

\newtheorem{lemma}{Lemma}
\newtheorem{thm}{Theorem}
\newtheorem{defi}{Definition}
\newtheorem{prp}{Proposition}
\newenvironment{md}{\begin{mdframed}}{\end{mdframed}}

\date{\today}
\usepackage{fancyhdr}
\pagestyle{fancy}
\fancyhead[L]{\slshape{Haoyu Zhen}}
\fancyhead[R]{{\today}}

% \begin{document}
% \title{<++>}
\author{Haoyu Zhen}
% \maketitle
\setlength{\parindent}{0pt}
\setstretch{1.2}
% \end{document}



\graphicspath{{figures/}}

\begin{document}
% \tableofcontents
\title{\vspace{-1.5cm}Homework 3}
\maketitle
\section*{Problem 1}
\textbf{(a).}
Suppose that $\{d_i\}$ is sorted. If $\exists i\in\{1,2,\cdots,n-1\}$, $d_{i+1}-d_i>C$ or  $D-d_n>C$, then it is NOT possible to reach B from A.
If not, it is possible.
\\\\
\textbf{(b).}
Intuitively, I design the Algo.\ref{gch} based on Algo.\ref{go}. While Algo.\ref{gch} may be indecipherable because I use a heap to optimize its complexity. The naive version of the backbone (the ``while'') is available at Algo.\ref{gcn}.

The correctness of Algo.\ref{gcn} entails that of Algo.\ref{gch}. I.e., the use of heap just reduce many operations of line 11 in Algo.\ref{gcn}.
Now I will use mathematical induction to show the soundness of Algo.\ref{gcn}:
\begin{itemize}
	\item Suppose that $\*d$ is ordered and $\+S_k$ is the minimum cost from  $A$ to  $d_k$.
	\item Now the car is at $d_k$ with the corresponding gas $C_{\mathrm{now}}$ left in tank.
	\item Assume that $(d_{k+1},d_{k+2},\cdots,d_l)\le C+d_k$ while $d_{l+1}>C+d_k$. For simplicity, $\+D\defeq\{k+1,\cdots,l\}$
	\item \case{1}
		$\forall i\in\+D$, $p_i>p_k$.
		The driver should fill the tank full and go to station $m\defeq\argmin_{i\in\+D}p_i$.
		This is because in $[d_k+C_{\mathrm{now}},d_k+C]$, he will spend $p_k(C-C_k)<p_kl_k+\sum_{i\in\+D}p_il_i$ where $l_k+\sum_{i\in\+D}l_i=C-C_{\mathrm{now}}$.
	\item \case{2}
		$\exists i\in\+D$, $p_i\le p_k$ while $(p_{k+1},p_{k+2},\cdots,p_{i-1})>p_k$.
		If $C_{\mathrm{now}}>d_i-d_k$, go to $i$ directly beacuse $p_k\sum_tl_t<p_i\sum_tl_t$ (station $i$ could cover any point $k$ could reach). If not, fill the tank just to $i$ (i.e., fill the gas tank to $d_i-d_k$). This holds by 
	\[
		p_k(C_{\mathrm{now}}+d_i-d_k)+p_k\sum_tl_t
		<
		\sum_{t=k}^{i-1}p_tl_t+p_i\sum_tl_t
	\]where $\sum_{t=k}^{i-1}l_t=C_{\mathrm{now}}+d_i-d_k$.
\end{itemize}

The complexity of Algo.\ref{gch}: Every stataion is inserted and poped \textbf{once} in the heap. So \[T(n)=\+O(n\log n).\]For \textit{naive} version, the complexity is $\+O(n^2)$.
\begin{algorithm}[htbp]
	\caption{$\+{GO}(l,p_k,d_k,d_l,\+S,C_{\mathrm{now}})$}
	\label{go}
	\begin{algorithmic}[1]
		\renewcommand{\algorithmicrequire}{\textbf{Input:}}
		\renewcommand{\algorithmicensure}{\textbf{Output:}}
		\renewcommand{\algorithmiccomment}[1]{\hfill\textit{\textcolor{blue}{\##1}}}
		\REQUIRE The car is at station $k$ with gas $C_{\mathrm{now}}$ and cost $\+S$. He want to go to station  $l$ directly.
		\ENSURE Update the value of $\+S,C_{\mathrm{now}}$ at station $l$.
		\STATE $\+S\gets\+S+p_k\max(d_l-d_k-C_{\mathrm{now}},0)$
		\STATE $C_{\mathrm{now}}\gets C_{\mathrm{now}}+\max(d_l-d_k-C_{\mathrm{now}},0)-(d_l-d_k)$
		\RETURN $(l,\+S,C_{\mathrm{now}})$
	\end{algorithmic} 
\end{algorithm}
\begin{algorithm}[htbp]
	\caption{Minimize the gas cost}
	\label{gch}
	\begin{algorithmic}[1]
		\renewcommand{\algorithmicrequire}{\textbf{Input:}}
		\renewcommand{\algorithmicensure}{\textbf{Output:}}
		\renewcommand{\algorithmiccomment}[1]{\hfill\textit{\textcolor{blue}{\##1}}}
		\REQUIRE Distance $D$, capacity  $C$, and 2 sequence $\*d=\{d_i\},\ \*p=\{p_i\}$  where  $i=1,2,\cdots,n$
		\ENSURE Minimal cost: $\+S$
		\STATE $d_{n+1}=D,\ p_{n+1}=+\infty$ and $\*d\gets\*d\cup\{d_{n+1}\},\ \*p\gets\*p\cup\{p_{n+1}\}$
		\STATE Sort $(d_i,p_i)$ w.r.t. $d_i$ such that $d_i<d_{i+1}$ for all $i$.
		\STATE $k\gets 1$, $l\gets 2$ and $C_{\mathrm{now}}\gets 0$
		\STATE Initialize a min-heap $H$ to store $(i,d_i,p_i)$.
		\STATE In $H$, $(i,d_i,p_i)<(j,d_j,p_j)$ if $p_i<p_j$. Also, $\mathrm{top}(H)$ will return the index instead of $d_i$ or $p_i$.
		\STATE Insert $(1,d_1,p_1)$ into $H$
		\WHILE{$k<n+1$}
		\STATE flag$\gets$False
		\WHILE{$d_l-d_k<C$}
		\IF{$p_l\le p_k$}
		\STATE $k,\+S,C_{\mathrm{now}}\gets\+{GO}(l,p_k,d_k,d_l,\+S,C_{\mathrm{now}})$
		\COMMENT{Go to $l$ directly}
		\STATE flag$\gets$True
		\STATE \textbf{Break}
		\ENDIF
		\STATE Insert $(l,d_l,p_l)$ into $H$
		\STATE  $l\gets l+1$
		\ENDWHILE
		\IF{flag}
		\STATE \textbf{Continue}
		\COMMENT{We have changed $k$}
		\ENDIF
		\WHILE{$\mathrm{top}(H)<k$ and $\mathrm{size}(H)>0$}
		\STATE $H$.pop()
		\COMMENT{Pop the stations in front of $k$}
		\ENDWHILE
		\IF{$\mathrm{size}(H)<0$}
		\RETURN IMPOSSIBLE
		\COMMENT{We could not reach any station at $k$}
		\ENDIF
		\STATE $j\gets \mathrm{top}(H)$ and $H$.pop()
		\IF{$p_j\le p_k$}
		\STATE $k,\+S,C_{\mathrm{now}}\gets\+{GO}(j,p_k,d_k,d_j,\+S,C_{\mathrm{now}})$
		\COMMENT{Go to $j$ directly}
		\ELSE
			\IF{$C+d_k>D$}
			\STATE $k,\+S,C_{\mathrm{now}}\gets\+{GO}(n+1,p_k,d_k,d_{n+1},\+S,C_{\mathrm{now}})$
			\COMMENT{Go to $B$ (the destination)}
			\ELSE
				\STATE $\+S\gets\+S+(C-C_{\mathrm{now}})p_k$
				\COMMENT{Fill the tank full}
				\STATE $C_{\mathrm{now}}\gets C-C_{\mathrm{now}}$
				\STATE $k\gets j$
			\ENDIF
		\ENDIF
		\ENDWHILE
	\end{algorithmic} 
\end{algorithm}
\begin{algorithm}[htbp]
	\caption{The backbone of Algo.\ref{gch}}
	\label{gcn}
	\begin{algorithmic}[1]
		\renewcommand{\algorithmiccomment}[1]{\hfill\textit{\textcolor{blue}{\##1}}}
		\WHILE{$k<n+1$}
		\STATE flag$\gets$False
		\STATE $l\gets k$
		\FOR{$d_l-d_k<C$}
		\IF{$p_l<p_k$}
		\STATE $k,\+S,C_{\mathrm{now}}\gets\+{GO}(l,p_k,d_k,d_l,\+S,C_{\mathrm{now}})$
		\COMMENT{Go to $l$ directly}
		\STATE flag$\gets$True
		\STATE \textbf{Break}
		\ENDIF
		\IF{$p_m\ge p_l$}
		\STATE $m=l$
		\COMMENT{Find the minimum of $p_l$ station $k$ could reach}
		\ENDIF
		\STATE $l\gets l+1$
		\ENDFOR
		\IF{flag}
		\STATE \textbf{Continue}
		\ENDIF
		\STATE $k,\+S,C_{\mathrm{now}}\gets\+{GO}(m,p_k,d_k,d_m,\+S,C_{\mathrm{now}})$
		\COMMENT{Go to $m$ directly}
		\ENDWHILE
	\end{algorithmic} 
\end{algorithm}

\section*{Problem 2}
\textbf{(1).}
\begin{algorithm}[H]
	\caption{$\+M(T)$}
	\label{c1}
	\begin{algorithmic}[1]
		\renewcommand{\algorithmicrequire}{\textbf{Input:}}
		\renewcommand{\algorithmicensure}{\textbf{Output:}}
		\renewcommand{\algorithmiccomment}[1]{\hfill\textit{\textcolor{blue}{\##1}}}
		\REQUIRE A given tree $T=(V,E)$ with the root $r$
		\ENSURE A minimum-size subset $\+S$ of vertices that covers all the vertices in $G$ w.r.t $k=1$.
		\IF{$\abs{V}=1$}
		\RETURN $V$
		\ENDIF
		\STATE $\+S\gets \emptyset$
		\STATE Implement BFS from root $r$ during which update the height and parent for each vertex
		\STATE Meanwhile, store the vertex we meet into a stack $s$
		\COMMENT{In $s$, vertex's height is ordered}
		\WHILE{$s$ is not empty}
		\STATE $v\gets s$.top
		\COMMENT{$v$ has the largest height in $s$}
		\STATE s.pop
		\IF{$v$ is not marked}
		\IF{$v$ is the root}
		\STATE  $\+S\gets\+S\cup\{v\}$
		\ENDIF
		\ELSE
		\STATE Mark $v$ and get $v$'s parent $u$
		\STATE $\+S\gets\+S\cup\{u\}$
		\STATE Mark $u$'s neighbor and $u$ itself
		\ENDIF
		\ENDWHILE
		\RETURN $\+S$
	\end{algorithmic} 
\end{algorithm}
\newpage
The correctness of Algo.\ref{c1}:
\begin{itemize}
	\item 
		First I need to interpret what Algo.\ref{c1} do: for given $T$, we find its leaves. Put these leaves' parents into $\+S$ and delete them. And do it repeatly. I use a stack and BFS to ameliorate the complexity.
	\item
		In $T$, if $u$ is a leaf whose parent is $v$, $v$ must in the vertex cover (because $v$ dominate $u$).
	\item
		Then deleting $u$ and $u$'s neighbor, we get a subgraph. It's obvious that $\+S=\+S'\cup\{u\}$ where $\+S'$ is minimum and cover the subgraph
\end{itemize}
The complexity: $\+O(\abs{V})$ because:
\begin{itemize}
	\item 
		$\+O(\rm{BFS})=\abs{V}$. Every vertex is pushed and poped once.
	\item
		Every vertex is marked at most twice (by ``itself/one of its child'' and by his parent)
\end{itemize}
\textbf{(2).} Similarly,
\begin{algorithm}[H]
	\caption{$\+M(T)$}
	\label{ck}
	\begin{algorithmic}[1]
		\renewcommand{\algorithmicrequire}{\textbf{Input:}}
		\renewcommand{\algorithmicensure}{\textbf{Output:}}
		\renewcommand{\algorithmiccomment}[1]{\hfill\textit{\textcolor{blue}{\##1}}}
		\REQUIRE A given tree $T=(V,E)$ with the root $r$
		\ENSURE A minimum-size subset $\+S$ of vertices that covers all the vertices in $G$ w.r.t $k$.
		\IF{$\abs{V}=1$}
		\RETURN $V$
		\ENDIF
		\STATE $\+S\gets \emptyset$
		\STATE Implement BFS from root $r$ during which update the height and parent for each vertex
		\STATE Meanwhile, store the vertex we meet into a stack $s$
		\COMMENT{In $s$, vertex's height is ordered}
		\WHILE{$s$ is not empty}
		\STATE $v\gets s$.top
		\COMMENT{$v$ has the largest height in $s$}
		\STATE s.pop
		\IF{$v$ is not marked}
		\IF{$v$'s height$<k$}
		\STATE  $\+S\gets\+S\cup\{r\}$
		\RETURN $\+S$
		\ELSE
		\STATE Mark $v$ and get $v$'s $k^{\rm{th}}$ ancestor $u$
		\STATE $\+S\gets\+S\cup\{u\}$
		\STATE Mark every vertex $t\in\{t\mid d(u,t)\le k\}$ where $d(a,b)$ is the distance between $a$ and  $b$
		\ENDIF
		\ENDIF
		\ENDWHILE
		\RETURN $\+S$
	\end{algorithmic} 
\end{algorithm}
The soundness of Algo.\ref{ck} is similar as Algo.\ref{c1}.
\begin{itemize}
	\item To generalize, I just simply alter the way of marking vertices.
	\item Also, for every $v'\in T$ could cover the deepest leaf $v$, $v'$ is in $u$'s subtree which contain $v$. And it is obvious that  $u$ dominate  the vertices $v'$ could reach.
\end{itemize}
The complexity: $\+O(f(k)\abs{V})$.
\begin{itemize}
	\item 
		The function $f(k)$'s physical interpretation is $\max_T(\text{average times for which a vertex is marked in $T$})$
	\item 
		The function $f(\cdot)$ may not be explicit but $f(k)<\abs{V}$ and $f(1)=2$.
\end{itemize}
\newpage
Now I will optimize Algo.\ref{ck}. Indeed, line 18 is redundant.
This time I will mark vertex with a specific number $m_v$.
\begin{algorithm}[H]
	\caption{The optimized version of the ``while'' backbone in $\+M(T)$}
	\label{ckp}
	\begin{algorithmic}[1]
		\renewcommand{\algorithmicrequire}{\textbf{Input:}}
		\renewcommand{\algorithmicensure}{\textbf{Output:}}
		\renewcommand{\algorithmiccomment}[1]{\hfill\textit{\textcolor{blue}{\##1}}}
		\WHILE{$s$ is not empty}
		\STATE $v\gets s$.top
		\COMMENT{$v$ has the largest height in $s$}
		\STATE s.pop
		\STATE $u\gets v$, $flag\gets$True
		\WHILE{$d(v,u)\le k$}
		\IF{$u$ is marked and $d(u,v)+m_u\le \Rnode{checkbegin}{k}$}
		\STATE $flag\gets$False, \textbf{break}
		\COMMENT{This means that $v$ has been covered}
		\ENDIF
		\STATE $u\gets u$'s parent (check before: if $u$ is the root, \textbf{break})
		\ENDWHILE
		\IF{$flag$}
		\STATE $\+S\gets\+S\cup\{u\}$
		\STATE Mark every vertex $t\in\{t\mid d(u,t)\le k\land t \text{ is the ancestor of $v$}\}$ and $m_t=\min[d(u,t),m_t]$
		\STATE
		\COMMENT{We initiate $m_t=+\infty$ for every $t\in V$}
		\ENDIF
		\ENDWHILE
		\RETURN $\+S$
	\end{algorithmic} 
\end{algorithm}
Correctness: 
\begin{itemize}
		\item I use a trick to determine whether a vertex has already been covered.
		\item That is: if his ancestor $u$ is marked with a number $m$. This number tells us that $d(u,t)$ where $t$ is in  $\+S$.
		\item If $d(u,t)+d(u,v)\le k$, vertex $v$ could be covered by $t$.
\end{itemize}
The complexity is $\+O(k\abs{V})$ because every vertex will be marked and \Rnode{checkend}{\underline{checked}} for at most  $2k+1$ times.
\ncarc[arrows = ->, linecolor =red, arcangle = 20, nodesep = .5pt]{checkbegin}{checkend}

\section*{Problem 3}
\textbf{(a).}
Let set $A$ and $B$ be 2 different maximal independent sets of $\+I$. 
If $\abs{A}<\abs{B}$,  then $A\subset A\cup\{x\}$ where  $x\in B-A$.
So it holds that $\abs{A}=\abs{B}$
\\\\
\textbf{(b).}
\begin{defi}
	$A$ is a set containing edges.  $\+F_A\defeq (V_A,A)$ where $V_A\defeq\{v|(v,u)\in A \text{ for some }u\}$.
\end{defi}
\begin{lemma}
	For all $A\in\+I$,  $\+F_A$ is a forest.
\end{lemma}
By lemma and defination above,
\begin{itemize}
	\item 
		Trivially, $A$ is hereditary. A forest's subgraph is a forest.
	\item
		Let set $A,B\in\+I$ such that  $\abs{A}<\abs{B}$.
		We have $\abs{V_A}<\abs{V_B}$. There exists $v\in V_B$, $v\notin V_A$. Also, $\exists e=(v,v')\in B-A$. Finnaly, $A\cup\{e\}\in\+I$ because $v$ in $\+F_{A\cup\{e\}}$ is a leaf. The exchange property holds for $\+I$.
\end{itemize}
Apparently, the maximal sets of this matroid is the maximum spanning forest of $G$.
\\\\
\textbf{(c).}
Firstly, $\{x\}\in\+I$ and $\forall w(y)>w(x):\{y\}\notin\+I$. Suppose that $\+S$ is the maximal independent set with maximum weight. By finite steps we could generate a maximal independent set $\+S'$:
\begin{enumerate}
	\item $\+S'=\{x\}$ 
	\item Update  $\+S'$ following exchange property: $\abs{\+S'}<\abs{\+S}$. Then $\+S'\gets \+S'\cup\{x'\}$ where $x'\in\+S-\+S'$.
	\item  Do step 2 repeatly and finnaly we have $\abs{\+S}=\abs{\+S'}$.
\end{enumerate} 
Obviously, $\abs{\+S'-\+S}\le 1$. Thus $w(\+S')\ge w(\+S)$ because $w(x)=\argmax_{y\in\{y|\{y\}\in\+I\}}w(y)$. Ultimately, $\+S'$ is what we want.
\\\\
\textbf{(d).}
Proof by mathematical induction. The abbreviation ``MISMW'' means the maximal independent set with maximum weight.
\begin{itemize}
	\item 
		The first element the algorithm added to $\+S$ is $x_1$. By (c), $\exists \+S'\in\+I$ such that $\{x_1\}\subset\+S'$.
	\item Suppose that the algorithm has added $\{x_1,x_2,\cdots,x_k\}$ into $\+S$. And $\exists \+S_k\in\+I$ such that $\+S\subset\+S_k$.
	\item  Now the algorithm will add  $x_{k+1}$ into $\+S$. Some property we could conclude:  $\forall y$ such that $w(x_{k+1})<w(y)$ and $y \ne (x_1,\cdots,x_k)$, there's no MISMW containing $\{x_1,x_2,\cdots,x_k,y\}$.
		Then we could do step.2 in (c) repeatly with $\+S_k$. And finally we will get $\+S_{k+1}$ which is a MISMW and contains  $\{x_1,x_2,\cdots,x_{k+1}\}$. (Nota bene: this is true by $\abs{\+S_{k+1}-\+S_k} \le 1$ and the property mentioned above.)
\end{itemize}
\textbf{(e).}
Let $\+I=\qty{u\subset U|\,\text{vectors in $u$ is linearly independent}}$. Obviously $M=(U,\+I)$ is a matroid: $M$ is naturally hereditary. Also if $\abs{A}<\abs{B}$, then $\exists \vec{v}\in B$ such tht $A\cup\{\vec{v}\}$ is linearly independent by the fact that $\dim(\-{span}(A))<\dim(\-{span}(B))$.

So we just run the algorithm given in problem set with a tuning: check whether $S\cup\{x\}$ is linearly independent. Here's the implementation of the check:
\\
\begin{algorithm}[htbp]
	\caption{$\+F(\*x,C)$: minimize the number of machines approximately}
	\label{li}
	\begin{algorithmic}[1]
		\renewcommand{\algorithmicrequire}{\textbf{Input:}}
		\renewcommand{\algorithmicensure}{\textbf{Output:}}
		\renewcommand{\algorithmiccomment}[1]{\hfill\textit{\textcolor{blue}{\##1}}}
		\REQUIRE A linearly independent set $S=\{\*s_1,\*s_2,\cdots,\*s_k\}$ and a vector  $\*x$
		\ENSURE True or False: $S\cup\{x\}$ is linearly independent
		\STATE  $m\gets 1$
		\WHILE{$\*x\ne \*0$ and $m\le k$}
		\STATE $\*x\gets\*x-(\*s_m\cdot\*x)\*s_m$
		\COMMENT{Part of Gram-–Schmidt Process}
		\STATE $m\gets m+1$
		\ENDWHILE
		\RETURN $\*x\ne \*0$
	\end{algorithmic}
\end{algorithm}
The complexity is $\+O(n^2)$.

\newpage
\section*{Problem 4}
\textbf{(a).}
\begin{algorithm}[htbp]
	\caption{$\+F(\*x,C)$: minimize the number of machines approximately}
	\label{nbin}
	\begin{algorithmic}[1]
		\renewcommand{\algorithmicrequire}{\textbf{Input:}}
		\renewcommand{\algorithmicensure}{\textbf{Output:}}
		\renewcommand{\algorithmiccomment}[1]{\hfill\textit{\textcolor{blue}{\##1}}}
		\REQUIRE $n$ jobs with sizes $x_i<C$, the capacity of machines  $C$
		\ENSURE  approximately minimum number of machine: $m$
		\STATE $m\gets 1$, $l\gets 1$ and $\+M_1,\+M_2,\cdots,\+M_n=0$
		\STATE Sort $x_i$ such that  $x_i>x_j$ for all  $i<j$
		\WHILE{$l\le n$}
		\STATE $j\gets \argmin_{0<i<m}\+M_i$
		\COMMENT{Using a heap to put $\+M$}
		\IF{$\+M_j+x_l\le C$}
		\STATE $\+M_j\gets \+M_j+x_l$
		\ELSE
			\STATE $\+M_m\gets x_l$
			\STATE $m\gets m+1$
		\ENDIF
		\STATE $l\gets l+1$
		\ENDWHILE
		\RETURN $\sum_{i=1}^n1[\+M_i\ne0]$
	\end{algorithmic} 
\end{algorithm}
Analysis of Algo.\ref{nbin}:\\[6pt]
Now we want to bound ALG$\defeq m$. For simplicity, let $\min\defeq\min_ix_i$.
If $\min>\flatfrac{C}{2}$, then obviously ALG=OPT. If not:
\[
	(\-{ALG}-1)(C-\min)+\min<\sum_ix_i\le C\cdot\-{OPT}
,\] which means tht
\begin{equation}
	\-{ALG}<\frac{C\cdot\-{OPT}-\min}{C-\min}+1
	\implies
	\-{ALG}\le 2\-{OPT}-1.
	\label{opt}
\end{equation}
The complexity is $\+O(n\log n)$.
\\[5pt]
Reference: \href{https://www.cs.jhu.edu/~mdinitz/classes/ApproxAlgorithms/Spring2015/notes/lec8.pdf}{Approximation Algorithms, Johns Hopkins University}.
\\\\
\textbf{(b).} For simplicity and without loss of generality, let $C=1$.
 
My intuition:
\begin{itemize}
	\item If $\min\le \flatfrac{1}{3}$. From Eq.\ref{opt} we conclude $\rm{ALG}<\flatfrac{3}{2}\rm{OPT}+\+O(1)$.
	\item If $\forall i$, $x_i>\flatfrac{1}{3}$, then we have a smaller problem. Also, there are several machines which only complete one job with size $x_i>\flatfrac{2}{3}$. Let $\+S_1=\{x_i|\flatfrac{2}{3}\le x_i\}$.
	\item Let $\+S_2=\{x_i|\flatfrac{1}{2}<x_i<\flatfrac{2}{3}\}$ and $\+S_3=\{x_i|\flatfrac{1}{3}<x_i<\flatfrac{1}{2}\}$. If $\abs{\+S_2}<\abs{\+S_3}$, then the problem is trivial. Because the number of machines $m=\abs{\+S_1}+\abs{\+S_3}$.
	\item If $\abs{\+S_2}>\abs{\+S_3}$, $m=\abs{\+S_1}+\abs{\+S_3}+\text{minimum number to complete $\+S_2$}$. (Then we have a smaller problem and do this repeatly)
\end{itemize}
While the performance of procedure above depends on the input's distribution. I.e., if $\*x$ centers around $\flatfrac{1}{2}$, we get $\rm{ALG}<2\rm{OPT}-1$ again.

So I refer to \href{https://en.wikipedia.org/wiki/First-fit_bin_packing#cite_note-Ullman713-1}{First-fit bin packing, Wikipedia} to desing Algo.\ref{pbin} whose performance is: $\rm{ALG}\le \flatfrac{5}{3}\rm{OPT}+\+O(1)$. (While I could not prove this proposition.)
\\[5pt]
\begin{algorithm}[htbp]
	\caption{minimize the number of machines approximately}
	\label{pbin}
	\begin{algorithmic}[1]
		\renewcommand{\algorithmicrequire}{\textbf{Input:}}
		\renewcommand{\algorithmicensure}{\textbf{Output:}}
		\renewcommand{\algorithmiccomment}[1]{\hfill\textit{\textcolor{blue}{\##1}}}
		\REQUIRE $n$ jobs with sizes $x_i<C$, the capacity of machines  $C$
		\ENSURE  approximately minimum number of machine: $m$
		\STATE $S_1,S_2,S_3,S_4\gets\emptyset$,  $i\gets 1$
		\WHILE{$i\le n$}
		\IF{$\flatfrac{C}{2}<x_i\le C$}
		\STATE $S_1\gets S_1\cup\{x_i\}$
		\ENDIF
		\IF{$\flatfrac{2C}{5}<x_i\le \flatfrac{C}{2}$}
		\STATE $S_2\gets S_2\cup\{x_i\}$
		\ENDIF
		\IF{$\flatfrac{C}{3}<x_i\le \frac{2C}{5}$}
		\STATE $S_3\gets S_3\cup\{x_i\}$
		\ENDIF
		\IF{$x_i\le \flatfrac{C}{3}$}
		\STATE $S_4\gets S_4\cup\{x_i\}$
		\ENDIF
		\ENDWHILE
		\RETURN $\+F(S_1,C)+\+F(S_2,C)+\+F(S_3,C)+\+F(S_4,C)$
	\end{algorithmic} 
\end{algorithm}

\section*{Problem 5}
About 3 days. Difficulty 4. Talked with Yilin Sun and Wei Jiang.
\end{document}
