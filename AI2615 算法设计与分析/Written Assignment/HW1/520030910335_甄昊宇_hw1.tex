% ! Tex program = xelatex
\documentclass{article}
% 中文
% \usepackage[UTF8]{ctex}

% For more choices
% %! Tex program = xelatex
% \documentclass{article}
%中文
%\usepackage[UTF8]{ctex}
%数学公式
\usepackage{amsmath,amssymb}
%\usepackage{ntheorem}
% \usepackage[framemethod=TikZ]{mdframed}
\usepackage{amsthm}
%边界
\usepackage[letterpaper,top=2cm,bottom=3cm,left=2cm,right=2cm,marginparwidth=1.75cm]{geometry}%table package
%Table
\usepackage{multirow,booktabs}
\usepackage{makecell}
%字体颜色
\usepackage{color}
% \usepackage[dvipsnames]{xcolor}  % 更全的色系
%代码
\usepackage[OT1]{fontenc}
% MATLAB 代码风格
%\usepackage[framed,numbered,autolinebreaks,useliterate]{/Users/anye_zhenhaoyu/Desktop/Latex/mcode}
\usepackage{listings}
\usepackage{algorithm}
\usepackage{algorithmic}
\usepackage{pythonhighlight} % Python
%插图
\usepackage{graphicx}
%改变item格式
\usepackage{enumerate}
%物理
\usepackage{physics}
%extra arrows
\usepackage{extarrows}
% caption(居中指令)
%\usepackage[justification=centering]{caption}
\usepackage{caption}
% htpb
\usepackage{stfloats}
% pdf 拼接
\usepackage{pdfpages}
% 超链接url
\usepackage{url}
\usepackage{tikz}
\usepackage{pgfplots}
\pgfplotsset{compat=newest}
\usepackage[colorlinks=true, allcolors=red]{hyperref}
\usepackage{setspace}
\usepackage{bbm}

% --------------definations-------------- %
\def\*#1{\boldsymbol{#1}}
\def\+#1{\mathcal{#1}} 
\def\-#1{\mathrm{#1}}
\def\rm#1{\mathrm{#1}}
\def\=#1{\mathbb{#1}}
% Domains
\def\RR{\mathbb{R}}
\def\EE{\mathbb{E}}
\def\CC{\mathbb{C}}
\def\NN{\mathbb{N}}
\def\ZZ{\mathbb{Z}}
% Newcommand
\newcommand{\inner}[2]{\langle #1,#2\rangle} 
\newcommand{\numP}{\#\mathbf{P}} 
\renewcommand{\P}{\mathbf{P}}
\newcommand{\Var}[2][]{\mathbf{Var}_{#1}\left[#2\right]}
\newcommand{\E}[2][]{\mathbf{E}_{#1}\left[#2\right]}
\renewcommand{\emptyset}{\varnothing}
\newcommand{\ol}{\overline}
\newcommand{\argmin}{\mathop{\arg\min}}
\newcommand{\argmax}{\mathop{\arg\max}}
\renewcommand{\abs}[1]{\qty|#1|}
\newcommand{\defeq}{\triangleq} % triangle over =
\def\deq{\xlongequal{def}} % 'def' over =
\def\LHS{\text{LHS}}
\def\RHS{\text{RHS}}
\def\angbr#1{\langle#1\rangle} % <x>
\def\set#1{\qty{#1}}

\def\Esolve{\textcolor{blue}{Solve: }}
\def\Eproof{\textcolor{blue}{Proof: }}
\def\case#1{\textcolor{blue}{Case \uppercase\expandafter{\romannumeral#1}: }}

% \newmdtheoremenv{lemma}{Lemma}
% \newmdtheoremenv{theorem}{\textcolor{red}{Theorem}}
% \newmdtheoremenv{defi}{\textcolor{blue}{Definition}}
\newtheorem{lemma}{Lemma}
\newtheorem{thm}{Theorem}
\newtheorem{defi}{Definition}
\newtheorem{prp}{Proposition}
\newtheorem{remark}{Remark}
\newenvironment{md}{\begin{mdframed}}{\end{mdframed}}

\graphicspath{{figures/}}

% \begin{document}
% \title{<++>}
\author{Haoyu Zhen}
% \maketitle
\setlength{\parindent}{0pt}
\setstretch{1.2}
% \end{document}

\usepackage{fancyhdr}
\pagestyle{fancy}
% \fancypagestyle{mainFancy}{
%     \fancyhf{}
%     \renewcommand\headrulewidth{.5pt}       % 页眉横线
%     \renewcommand\footrulewidth{0pt}
%     \fancyhead[OC]{\fzkai{\leftmark}}       % 页眉章标题
%     \fancyhead[EC]{\fzkai{\@title}}         % 页眉文章题目
% 	\lhead{\fzkai{author}}
%     \fancyhead[OR,EL]{\thepage}                 % 页眉编号
%     \fancyfoot[r]{\thumb} % 将拇指放到没有被使用的页眉或页脚处
% }
\fancyhf{}
\fancyfoot[C]{\thepage}
\fancyhead[R]{\slshape{AI2615 Algorithm Design and Analysis}}
\fancyhead[L]{\slshape{Haoyu Zhen}}

% % theorems
\usepackage{thmtools}
\usepackage{thm-restate}
\usepackage[framemethod=TikZ]{mdframed}
\mdfsetup{skipabove=1em,skipbelow=0em, innertopmargin=12pt, innerbottommargin=8pt}

\theoremstyle{definition}

\declaretheoremstyle[headfont=\bfseries\sffamily, bodyfont=\normalfont,
	mdframed={
		nobreak,
		backgroundcolor=brown!14,
		topline=false,
		rightline=false,
		leftline=true,
		bottomline=false,
		linewidth=2pt,
		linecolor=brown!180,
	}
]{thmbrownbox}

\declaretheoremstyle[headfont=\bfseries\sffamily, bodyfont=\normalfont,
	mdframed={
		nobreak,
		backgroundcolor=Blue!4,
		topline=false,
		rightline=false,
		leftline=true,
		bottomline=false,
		linewidth=2pt,
		linecolor=NavyBlue!120,
	}
]{thmbluebox}

\declaretheoremstyle[headfont=\bfseries\sffamily, bodyfont=\normalfont,
	mdframed={
		nobreak,
		backgroundcolor=Green!5,
		topline=false,
		rightline=false,
		leftline=true,
		bottomline=false,
		linewidth=2pt,
		linecolor=OliveGreen!120,
	}
]{thmgreenbox}

\declaretheoremstyle[headfont=\bfseries\sffamily, bodyfont=\normalfont,
	mdframed={
		nobreak,
		topline=false,
		rightline=false,
		leftline=true,
		bottomline=false,
		linewidth=2pt,
		linecolor=OliveGreen!120,
	}
]{thmgreenline}

\declaretheoremstyle[headfont=\bfseries\sffamily, bodyfont=\normalfont,
	mdframed={
		nobreak,
		topline=false,
		rightline=false,
		leftline=true,
		bottomline=false,
		linewidth=2pt,
		linecolor=NavyBlue!70,
	}
]{thmblueline}

\declaretheorem[numberwithin=section, style=thmbrownbox, name={\color{Brown}Definition}]{defi}
\declaretheorem[numberwithin=section, style=thmgreenbox, name={\color{OliveGreen}Law}]{law}
\declaretheorem[numberwithin=section, style=thmbluebox, name={\color{Blue}Corollary}]{cor}
\declaretheorem[numberwithin=section, style=thmgreenline, name={\color{OliveGreen}Property}]{prt}
\declaretheorem[numberwithin=section, style=thmbluebox, name={\color{Blue}Proposition}]{prp}
\declaretheorem[numberwithin=section, style=thmbluebox, name={\color{Blue}Theorem}]{thm}
\declaretheorem[numberwithin=section, style=thmbluebox, name={\color{Blue}Lemma}]{lemma}
\declaretheorem[numberwithin=section, style=thmbrownbox,  name={\color{Brown}Example}]{eg}
\declaretheorem[numberwithin=section, style=thmgreenline, name={\color{OliveGreen}Remark}]{remark}
\declaretheorem[numbered=no,style=thmblueline, name={\color{NavyBlue!70}Proof},qed=$\square$]{prf}
\numberwithin{equation}{section}


% %! Tex program = xelatex
% \documentclass{article}
%中文
%\usepackage[UTF8]{ctex}
%数学公式
\usepackage{amsmath,amssymb}
%\usepackage{ntheorem}
% \usepackage[framemethod=TikZ]{mdframed}
\usepackage{amsthm}
%边界
\usepackage[letterpaper,top=2cm,bottom=3cm,left=2cm,right=2cm,marginparwidth=1.75cm]{geometry}%table package
%Table
\usepackage{multirow,booktabs}
\usepackage{makecell}
%字体颜色
\usepackage{color}
% \usepackage[dvipsnames]{xcolor}  % 更全的色系
%代码
\usepackage[OT1]{fontenc}
% MATLAB 代码风格
%\usepackage[framed,numbered,autolinebreaks,useliterate]{/Users/anye_zhenhaoyu/Desktop/Latex/mcode}
\usepackage{listings}
\usepackage{algorithm}
\usepackage{algorithmic}
\usepackage{pythonhighlight} % Python
%插图
\usepackage{graphicx}
%改变item格式
\usepackage{enumerate}
%物理
\usepackage{physics}
%extra arrows
\usepackage{extarrows}
% caption(居中指令)
%\usepackage[justification=centering]{caption}
\usepackage{caption}
% htpb
\usepackage{stfloats}
% pdf 拼接
\usepackage{pdfpages}
% 超链接url
\usepackage{url}
\usepackage{tikz}
\usepackage{pgfplots}
\pgfplotsset{compat=newest}
\usepackage[colorlinks=true, allcolors=red]{hyperref}
\usepackage{setspace}
\usepackage{bbm}

% --------------definations-------------- %
\def\*#1{\boldsymbol{#1}}
\def\+#1{\mathcal{#1}} 
\def\-#1{\mathrm{#1}}
\def\rm#1{\mathrm{#1}}
\def\=#1{\mathbb{#1}}
% Domains
\def\RR{\mathbb{R}}
\def\EE{\mathbb{E}}
\def\CC{\mathbb{C}}
\def\NN{\mathbb{N}}
\def\ZZ{\mathbb{Z}}
% Newcommand
\newcommand{\inner}[2]{\langle #1,#2\rangle} 
\newcommand{\numP}{\#\mathbf{P}} 
\renewcommand{\P}{\mathbf{P}}
\newcommand{\Var}[2][]{\mathbf{Var}_{#1}\left[#2\right]}
\newcommand{\E}[2][]{\mathbf{E}_{#1}\left[#2\right]}
\renewcommand{\emptyset}{\varnothing}
\newcommand{\ol}{\overline}
\newcommand{\argmin}{\mathop{\arg\min}}
\newcommand{\argmax}{\mathop{\arg\max}}
\renewcommand{\abs}[1]{\qty|#1|}
\newcommand{\defeq}{\triangleq} % triangle over =
\def\deq{\xlongequal{def}} % 'def' over =
\def\LHS{\text{LHS}}
\def\RHS{\text{RHS}}
\def\angbr#1{\langle#1\rangle} % <x>
\def\set#1{\qty{#1}}

\def\Esolve{\textcolor{blue}{Solve: }}
\def\Eproof{\textcolor{blue}{Proof: }}
\def\case#1{\textcolor{blue}{Case \uppercase\expandafter{\romannumeral#1}: }}

% \newmdtheoremenv{lemma}{Lemma}
% \newmdtheoremenv{theorem}{\textcolor{red}{Theorem}}
% \newmdtheoremenv{defi}{\textcolor{blue}{Definition}}
\newtheorem{lemma}{Lemma}
\newtheorem{thm}{Theorem}
\newtheorem{defi}{Definition}
\newtheorem{prp}{Proposition}
\newtheorem{remark}{Remark}
\newenvironment{md}{\begin{mdframed}}{\end{mdframed}}

\graphicspath{{figures/}}

% \begin{document}
% \title{<++>}
\author{Haoyu Zhen}
% \maketitle
\setlength{\parindent}{0pt}
\setstretch{1.2}
% \end{document}

\usepackage{fancyhdr}
\pagestyle{fancy}
% \fancypagestyle{mainFancy}{
%     \fancyhf{}
%     \renewcommand\headrulewidth{.5pt}       % 页眉横线
%     \renewcommand\footrulewidth{0pt}
%     \fancyhead[OC]{\fzkai{\leftmark}}       % 页眉章标题
%     \fancyhead[EC]{\fzkai{\@title}}         % 页眉文章题目
% 	\lhead{\fzkai{author}}
%     \fancyhead[OR,EL]{\thepage}                 % 页眉编号
%     \fancyfoot[r]{\thumb} % 将拇指放到没有被使用的页眉或页脚处
% }
\fancyhf{}
\fancyfoot[C]{\thepage}
\fancyhead[R]{\slshape{AI2615 Algorithm Design and Analysis}}
\fancyhead[L]{\slshape{Haoyu Zhen}}

% % theorems
\usepackage{thmtools}
\usepackage{thm-restate}
\usepackage[framemethod=TikZ]{mdframed}
\mdfsetup{skipabove=1em,skipbelow=0em, innertopmargin=12pt, innerbottommargin=8pt}

\theoremstyle{definition}

\declaretheoremstyle[headfont=\bfseries\sffamily, bodyfont=\normalfont,
	mdframed={
		nobreak,
		backgroundcolor=brown!14,
		topline=false,
		rightline=false,
		leftline=true,
		bottomline=false,
		linewidth=2pt,
		linecolor=brown!180,
	}
]{thmbrownbox}

\declaretheoremstyle[headfont=\bfseries\sffamily, bodyfont=\normalfont,
	mdframed={
		nobreak,
		backgroundcolor=Blue!4,
		topline=false,
		rightline=false,
		leftline=true,
		bottomline=false,
		linewidth=2pt,
		linecolor=NavyBlue!120,
	}
]{thmbluebox}

\declaretheoremstyle[headfont=\bfseries\sffamily, bodyfont=\normalfont,
	mdframed={
		nobreak,
		backgroundcolor=Green!5,
		topline=false,
		rightline=false,
		leftline=true,
		bottomline=false,
		linewidth=2pt,
		linecolor=OliveGreen!120,
	}
]{thmgreenbox}

\declaretheoremstyle[headfont=\bfseries\sffamily, bodyfont=\normalfont,
	mdframed={
		nobreak,
		topline=false,
		rightline=false,
		leftline=true,
		bottomline=false,
		linewidth=2pt,
		linecolor=OliveGreen!120,
	}
]{thmgreenline}

\declaretheoremstyle[headfont=\bfseries\sffamily, bodyfont=\normalfont,
	mdframed={
		nobreak,
		topline=false,
		rightline=false,
		leftline=true,
		bottomline=false,
		linewidth=2pt,
		linecolor=NavyBlue!70,
	}
]{thmblueline}

\declaretheorem[numberwithin=section, style=thmbrownbox, name={\color{Brown}Definition}]{defi}
\declaretheorem[numberwithin=section, style=thmgreenbox, name={\color{OliveGreen}Law}]{law}
\declaretheorem[numberwithin=section, style=thmbluebox, name={\color{Blue}Corollary}]{cor}
\declaretheorem[numberwithin=section, style=thmgreenline, name={\color{OliveGreen}Property}]{prt}
\declaretheorem[numberwithin=section, style=thmbluebox, name={\color{Blue}Proposition}]{prp}
\declaretheorem[numberwithin=section, style=thmbluebox, name={\color{Blue}Theorem}]{thm}
\declaretheorem[numberwithin=section, style=thmbluebox, name={\color{Blue}Lemma}]{lemma}
\declaretheorem[numberwithin=section, style=thmbrownbox,  name={\color{Brown}Example}]{eg}
\declaretheorem[numberwithin=section, style=thmgreenline, name={\color{OliveGreen}Remark}]{remark}
\declaretheorem[numbered=no,style=thmblueline, name={\color{NavyBlue!70}Proof},qed=$\square$]{prf}
\numberwithin{equation}{section}


% On my MAC's Desktop
%! Tex program = xelatex
% \documentclass{article}
%中文
%\usepackage[UTF8]{ctex}
%数学公式
\usepackage{amsmath,amssymb}
%\usepackage{ntheorem}
% \usepackage[framemethod=TikZ]{mdframed}
\usepackage{amsthm}
%边界
\usepackage[letterpaper,top=2cm,bottom=3cm,left=2cm,right=2cm,marginparwidth=1.75cm]{geometry}%table package
%Table
\usepackage{multirow,booktabs}
\usepackage{makecell}
%字体颜色
\usepackage{color}
% \usepackage[dvipsnames]{xcolor}  % 更全的色系
%代码
\usepackage[OT1]{fontenc}
% MATLAB 代码风格
%\usepackage[framed,numbered,autolinebreaks,useliterate]{/Users/anye_zhenhaoyu/Desktop/Latex/mcode}
\usepackage{listings}
\usepackage{algorithm}
\usepackage{algorithmic}
\usepackage{pythonhighlight} % Python
%插图
\usepackage{graphicx}
%改变item格式
\usepackage{enumerate}
%物理
\usepackage{physics}
%extra arrows
\usepackage{extarrows}
% caption(居中指令)
%\usepackage[justification=centering]{caption}
\usepackage{caption}
% htpb
\usepackage{stfloats}
% pdf 拼接
\usepackage{pdfpages}
% 超链接url
\usepackage{url}
\usepackage{tikz}
\usepackage{pgfplots}
\pgfplotsset{compat=newest}
\usepackage[colorlinks=true, allcolors=red]{hyperref}
\usepackage{setspace}
\usepackage{bbm}

% --------------definations-------------- %
\def\*#1{\boldsymbol{#1}}
\def\+#1{\mathcal{#1}} 
\def\-#1{\mathrm{#1}}
\def\rm#1{\mathrm{#1}}
\def\=#1{\mathbb{#1}}
% Domains
\def\RR{\mathbb{R}}
\def\EE{\mathbb{E}}
\def\CC{\mathbb{C}}
\def\NN{\mathbb{N}}
\def\ZZ{\mathbb{Z}}
% Newcommand
\newcommand{\inner}[2]{\langle #1,#2\rangle} 
\newcommand{\numP}{\#\mathbf{P}} 
\renewcommand{\P}{\mathbf{P}}
\newcommand{\Var}[2][]{\mathbf{Var}_{#1}\left[#2\right]}
\newcommand{\E}[2][]{\mathbf{E}_{#1}\left[#2\right]}
\renewcommand{\emptyset}{\varnothing}
\newcommand{\ol}{\overline}
\newcommand{\argmin}{\mathop{\arg\min}}
\newcommand{\argmax}{\mathop{\arg\max}}
\renewcommand{\abs}[1]{\qty|#1|}
\newcommand{\defeq}{\triangleq} % triangle over =
\def\deq{\xlongequal{def}} % 'def' over =
\def\LHS{\text{LHS}}
\def\RHS{\text{RHS}}
\def\angbr#1{\langle#1\rangle} % <x>
\def\set#1{\qty{#1}}

\def\Esolve{\textcolor{blue}{Solve: }}
\def\Eproof{\textcolor{blue}{Proof: }}
\def\case#1{\textcolor{blue}{Case \uppercase\expandafter{\romannumeral#1}: }}

% \newmdtheoremenv{lemma}{Lemma}
% \newmdtheoremenv{theorem}{\textcolor{red}{Theorem}}
% \newmdtheoremenv{defi}{\textcolor{blue}{Definition}}
\newtheorem{lemma}{Lemma}
\newtheorem{thm}{Theorem}
\newtheorem{defi}{Definition}
\newtheorem{prp}{Proposition}
\newtheorem{remark}{Remark}
\newenvironment{md}{\begin{mdframed}}{\end{mdframed}}

\graphicspath{{figures/}}

% \begin{document}
% \title{<++>}
\author{Haoyu Zhen}
% \maketitle
\setlength{\parindent}{0pt}
\setstretch{1.2}
% \end{document}

\usepackage{fancyhdr}
\pagestyle{fancy}
% \fancypagestyle{mainFancy}{
%     \fancyhf{}
%     \renewcommand\headrulewidth{.5pt}       % 页眉横线
%     \renewcommand\footrulewidth{0pt}
%     \fancyhead[OC]{\fzkai{\leftmark}}       % 页眉章标题
%     \fancyhead[EC]{\fzkai{\@title}}         % 页眉文章题目
% 	\lhead{\fzkai{author}}
%     \fancyhead[OR,EL]{\thepage}                 % 页眉编号
%     \fancyfoot[r]{\thumb} % 将拇指放到没有被使用的页眉或页脚处
% }
\fancyhf{}
\fancyfoot[C]{\thepage}
\fancyhead[R]{\slshape{AI2615 Algorithm Design and Analysis}}
\fancyhead[L]{\slshape{Haoyu Zhen}}


\graphicspath{{figures/}}

\begin{document}
% \tableofcontents
\title{Written Assignment 1}
\maketitle
\section*{Problem 1}
Firstly, we have
\[
	\begin{aligned}
		T(n)=a^{\log_bn}\times\+O(1)+
		\+O(n^d\log^wn)+
		a\times\+O\qty(\qty(\frac{n}{b})^d\log^w\qty(\frac{n}{b}))
		+\cdots+
		a^m\times\+O\qty(\qty(\frac{n}{b^m})^d\log^w\qty(\frac{n}{b^m}))
	\end{aligned}
\] where $m=\log_b n$.
More compactly,
\[
	\begin{aligned}
		T(n)
		&=\+O\qty(a^{\log_bn})+
		\sum_{i=0}^m
		\+O\qty(\qty(\frac{a}{b^d})^in^d\log^w\qty(\frac{n}{b^i}))
	\end{aligned}
.\] 
\case{1} If $a=b^d$:
 \[
	 \begin{aligned}
		 T(n)
		 =\+O(n^d)+\sum_{i=0}^m
		 \+O\qty(n^d\log^w\qty(\frac{n}{b^i}))
		 =
		 \sum_{i=0}^m\+O\qty(n^d\log^wn)
		 =
		 m\times\+O(n^d\log^{w}n)
		 =
		 \+O(n^d\log^{w+1}n)
	 \end{aligned}
.\] 
\case{2} If $a<b^d$:
 \[
	 \begin{aligned}
		 T(n)
		 &=
		 \+O(n^{\log_ba})+
		 \sum_{i=0}^m
		 \+O\qty(\qty(\frac{a}{b^d})^in^d\log^wn)
		 \\&=
		 \+O(n^{\log_ba})+
		 \frac{1-\qty(\flatfrac{a}{b^d})^m}
		 {1-\flatfrac{a}{b^d}}
		 \+O\qty(n^d\log^wn)
		 \\[5pt]&=
		 \+O(n^{\log_ba})+\+O(n^d\log^wn)
		 \\&=
		 \+O(n^d\log^wn)
	 \end{aligned}
.\] 
\case{3} If $a>b^d$: $\forall \varepsilon>0$
\[
	\begin{aligned}
		T(n)&=
		\+O(n^{\log_ba})+
		\sum_{i=0}^m\+O\qty(\qty(\frac{a}{b^d})^in^d\qty(\frac{n}{b^i})^\varepsilon)
		\\&=\+O(n^{\log_ba})+
		\frac{1-\qty(\flatfrac{a}{b^{d+\varepsilon}})^m}
		{1-\flatfrac{a}{b^{d+\varepsilon}}}\+O\qty(n^{d+\varepsilon})
		\\[5pt]&=
		\+O(n^{\log_ba})+
		\qty(\flatfrac{a}{b^{d+\varepsilon}})^m\+O\qty(n^{d+\varepsilon})
		\\[5pt]&=\+O(n^{\log_ba+\varepsilon})
		\longrightarrow\+O\qty(n^{\log_ba})
	\end{aligned}
.\]
\\
Finally the generalization of the master theorem holds.
Reference: \href{https://facultystaff.richmond.edu/~dszajda/classes/cs315/Spring_2020/lectures/Divide-and_conquer2020.pdf}{CS315 Richmond University}.
\newpage
\section*{Problem 2}
Design the function $\mathrm{Merge}$ recursivly:
\begin{algorithm}[H]
	\caption{Merge sort with one third dividing approach}
	\begin{algorithmic}
		\renewcommand{\algorithmicrequire}{\textbf{Input:}}
		\renewcommand{\algorithmicensure}{\textbf{Output:}}
		\REQUIRE A sequence $\qty{a_i}$ to be sorted where $i\in\set{1,2, \cdots,L}$.
		\ENSURE $\qty{R_i}$.
		\IF{$L==1$}
		\STATE $R_1=a_1$
		\RETURN		
		\ENDIF
		\IF{$L==2$}
		\STATE $R_1=\min(a_1,a_2)$ and $R_2=\max(a_1,a_2)$
		\RETURN		
		\ENDIF
		\STATE $\qty{b_i}\gets \qty{a_1,a_2, \cdots a_{[\flatfrac{n}{3}]}}$
		\STATE $\qty{c_i}\gets \qty{a_{[\flatfrac{n}{3}]+1},a_{[\flatfrac{n}{3}]+2}, \cdots a_L}$\\[5pt]
		\STATE $\{S_i\}\gets\mathrm{Merge}\qty(\qty{b_i},[\flatfrac{n}{3}])$\\[4pt]
		\STATE $\{T_i\}\gets\mathrm{Merge}(\qty{c_i},n-[\flatfrac{n}{3}])$\\[4pt]
		\STATE $f_1\gets 1$, $f_2\gets 1$ and $i\gets 1$.
		\WHILE{$\qty(f_1\le [\flatfrac{n}{3}]\land f_2\le n-[\flatfrac{n}{3}])$}
		\IF{$f_2>n-[\flatfrac{n}{3}]$ or $b_{f_1}<c_{f_2}$}
		\STATE $R_i\gets b_{f_1}$
		\STATE $f_1\gets f_1+1$
		\ELSE
			\STATE $R_i\gets c_{f_2}$
			\STATE $f_2\gets f_2+1$
		\ENDIF
		\STATE $i\gets i+1$
		\ENDWHILE
		\RETURN
	\end{algorithmic} 
\end{algorithm}
\textbf{The Soundness} of ALG.1: 

Proof by induction. Trivially, when $n=1,2$, satisfied. Suppose that $\mathrm{Merge}(A,n)$ could successfully sort  $A$ for any  $n\le k$. Then the result $R_i$ of $\mathrm{Merge}(A,k+1)$ satisfies that: $\forall i$, assume $R_{i}=S_k$, then  $R_{i+1}=S_{k+1}\mbox{ or }T_k>R_i$ by the ``if'' condition. Thus $R$ is sorted.
\\
\\
\textbf{Complexity Analysis} of ALG.1:

Thus we have
\[
	T(n)=T\qty(\frac{n}{3})+T\qty(\frac{2n}{3})+\+O(n)
.\] 
$\exists B>0$, $\displaystyle \exists C=\qty[\frac{1}{3}\log3+\frac{2}{3}\log(\frac{3}{2})]^{-1}B$:
\[
	\begin{aligned}
		T(n)&\le
		C\frac{n}{3}\log(\frac{n}{3})+C\frac{2n}{3}\log(\frac{2n}{3})+Bn
		\\&\le
		Cn\log(n)-(\frac{1}{3}\log3+\frac{2}{3}\log(\frac{3}{2}))Cn+Bn
		\\[4pt]&\le
		Cn\log n
	\end{aligned}
\] which means that $T(n)=\+O(n\log n)$.

\newpage
\section*{Problem 3}
The solution of (b).
\begin{algorithm}[H]
	\caption{Find the second largest integer}
	\begin{algorithmic}
		\renewcommand{\algorithmicrequire}{\textbf{Input:}}
		\renewcommand{\algorithmicensure}{\textbf{Output:}}
		\REQUIRE A squence $\{A_i\}$ where $i\in\qty{1,2, \cdots,2^k}$
		\ENSURE $R$, the second largest integer in $\qty{A_i}$\\[5pt]
		\STATE $m\gets 1$, $S\gets \qty{A_i}$
		\WHILE{$m\le k$}
		\STATE Partition $S$ into subsets $\{S_i^{m}\}$ with size 2.\\[3pt]
		\STATE For each $S^m_i$, find the max integer $A^{m}_i$. \# Comparation 1\\[3pt]
		\STATE $S\gets\{A^m_i\}$, $m\gets m+1$\\[3pt]
		\ENDWHILE
		\STATE Finally we have the maximum $M\defeq A^k_1$ in $\{A_i\}$.\\[3pt]
		\STATE $T\gets \bigcup S^m_i-\{M\}$ where  $M\in S^m_i$\\[3pt]
		\STATE Use a single step in Bubble Sort to find the maximum $M'$ in T. \# Comparation 2\\[3pt]
		\RETURN $M'$
	\end{algorithmic}
\end{algorithm}
\textbf{The Soundness} of ALG.2: the second largest integer $M'$ must be compared with $M$ in ``while''. If not,  $\exists\ x$ such that $M>x>M'$ which leads to a contradiction.
\\\\
Times to compare: $\abs{T}=k$ and  $\abs{\qty{S^m_i}}=2^{k-m}$. Thus
\[
	\mathrm{NumOfComp}=\abs{T}+\sum_{m=1}^k\abs{\qty{S^m_i}}=2^k+k-1<n+\log n
.\] 

\section*{Problem 4}
\textbf{(a)}
\begin{algorithm}[H]
	\caption{count the number of pairs when $d=1$}
	\label{d=1}
	\begin{algorithmic}
		\renewcommand{\algorithmicrequire}{\textbf{Input:}}
		\renewcommand{\algorithmicensure}{\textbf{Output:}}
		\REQUIRE $A,B$
		\STATE Sum $\gets 0$
		\STATE Sort $A$: $a_1<a_2<\cdots<a_k$
		\STATE Sort $B$:  $b_1<b_2<\cdots<b_l$
		\STATE $f_1\gets 1$,  $f_2\gets 1$ and $f\gets 0$
		\WHILE{$f_1\le k$ and $f_2\le l$}
		\IF{$f_2>l$ or $a_{f_1}<b_{f_2}$}
		\STATE $f_1\gets f_1+1$
		\STATE $\mathrm{Sum}\gets\mathrm{Sum}+f$
		\ELSE
			\STATE $f_2\gets f_2+1$
			\STATE $f\gets f+1$
		\ENDIF
		\ENDWHILE
		\RETURN Sum
	\end{algorithmic}
\end{algorithm}
\textbf{The Soundness} of ALG.3:
Trivial. We just sort $A$ and $B$. And $\={sum}=\sum_i\qty(\sum_j\*1(b_j<a_i))$
\\\\
\textbf{Complexity Analysis:}
\[
	T(n)=\+O(a\log a)+\+O(b\log b)+\+O(n)=\+O(n\log n)
.\] 
\newpage
\textbf{(b)}
Note: we only sort $A,B$ for the first time.
\begin{algorithm}[H]
	\caption{count the number of pairs when $d=2$}
	\begin{algorithmic}
		\renewcommand{\algorithmicrequire}{\textbf{Input:}}
		\renewcommand{\algorithmicensure}{\textbf{Output:}}
		\REQUIRE $A,B\in\RR^2$
		\ENSURE Sum
		\IF{$\abs{A}=0$ or $\abs{B}=0$}
		\RETURN 0
		\ENDIF
		\STATE $\mathrm{Sum}\gets 0$
		\IF {$A,B$ are not sorted}
		\STATE Sort $A$ and $B$ for each dimension ($x$ and $y$) \# $\+O(n\log n)$
		\ENDIF
		\STATE Find the median $m$ of $A\cup B$ for $x$-axis \# $\+O(n)$
		\STATE Patition $A$ into  $A_1$ and $A$ where $A_1\cup A_2=A$ and $\forall a\in A_1,a'\in A_2$, $a_x\le m<a'_x$
		\STATE Similarly, partition $B$ into  $B_1$ and $B_2$
		\STATE \# Do recursively and reduce the dimensionality
		\STATE $\mathrm{Sum1}\gets\mathrm{Count}(A_1,B_1,\mathrm{dimension}=2)$ \# $S(\flatfrac{n}{2})$
		\STATE $\mathrm{Sum2}\gets\mathrm{Count}(A_2,B_2,\mathrm{dimension}=2)$ \# $S(\flatfrac{n}{2})$
		\STATE $\mathrm{Sum3}\gets\mathrm{Count}(A_2,B_1,\mathrm{dimension}=1,\mathrm{axis}=y)$ \# $\+O(n)$ (We have already sorted $A_y$ and  $B_y$.)
		\RETURN $\mathrm{Sum1}+\mathrm{Sum2}+\mathrm{Sum3}$
	\end{algorithmic}
\end{algorithm}
\textbf{The Soundness} of ALG.4: same as \textbf{The Soundness} of ALG.5.
\\\\
\textbf{Complexity Analysis:}
Thus we have $T(n)=S(n)+\+O(n\log n)$ where
\[
	S(n)=S\qty(\frac{n}{2})+\+O(n)\Longrightarrow S(n)=\+O(n\log n)
.\]
Finally, $T(n)=\+O(n\log n)$.
\\[8pt]
\textbf{(c)}
\begin{algorithm}[H]
	\caption{count the number of pairs when $\mathrm{dimension}=d$}
	\begin{algorithmic}
		\renewcommand{\algorithmicrequire}{\textbf{Input:}}
		\renewcommand{\algorithmicensure}{\textbf{Output:}}
		\REQUIRE $A,B\in\RR^2$
		\ENSURE Sum
		\IF{$\abs{A}=0$ or $\abs{B}=0$}
		\RETURN 0
		\ENDIF
		\STATE $\mathrm{Sum}\gets 0$
		\IF {$A,B$ are not sorted}
		\STATE Sort $A$ and $B$ for each dimension \# $\+O(dn\log n)$
		\ENDIF
		\STATE Find the median $m$ of $A\cup B$ for the first axis \# $\+O(n)$
		\STATE Patition $A$ into  $A_1$ and $A$ where $A_1\cup A_2=A$ and $\forall a\in A_1,a'\in A_2$, $a_x\le m<a'_x$
		\STATE Similarly, partition $B$ into  $B_1$ and $B_2$
		\STATE \# Do recursively and reduce the dimensionality
		\STATE $\mathrm{Sum1}\gets\mathrm{Count}(A_1,B_1,\mathrm{dimension}=d,\mathrm{axis}=\{1,2, \cdots,d\})$ \# $S(\flatfrac{n}{2},d)$
		\STATE $\mathrm{Sum2}\gets\mathrm{Count}(A_2,B_2,\mathrm{dimension}=d,\mathrm{axis}=\{1,2, \cdots,d\})$ \# $S(\flatfrac{n}{2},d)$
		\STATE $\mathrm{Sum3}\gets\mathrm{Count}(A_2,B_1,\mathrm{dimension}=d-1,\mathrm{axis}=\{2,3, \cdots,d\})$ \# $S(n,d-1)$ (We have already sorted $A$ and $B$.)
		\RETURN $\mathrm{Sum1}+\mathrm{Sum2}+\mathrm{Sum3}$
	\end{algorithmic}
\end{algorithm}
\newpage
\textbf{The Soundness} of ALG.5:
Obviously, $\forall a\in A_1$, $b\in B_2$, $(a,b)$ is not the pair we want. Then $\forall a\in A_2$, $b\in B_1$, $a_1>b_1$. Then we only need to count $A_2,B_1$ for dimension $d-1$.
\\\\
\textbf{Complexity Analysis:}

Now we have $T(n,d)=\+O\qty(dn\log(n))+S(n,d)$ where
\[
	S(n,d)=2S\qty(\frac{n}{2},d)+S(n,d-1)+\+O(n)
.\] 
\begin{prp}
	\[
		S(n,d)=\+O(n\log^{d-1}n)
	.\] 
\end{prp}
\begin{proof}
	When $d=2$, satisfied. Now we suppose that when  $d\le k$, $S(n,d)=\+O(n\log^{d-1}n)$.

	Thus we have
	\[
		S(n,k+1)=2S\qty(\frac{n}{2},k+1)+\+O(n\log^kn)
	.\]
	By the general master Thm. in problem 1: $S(n,k+1)=\+O(n\log^{k+1}n)$.
\end{proof}
Ultimately,
\[
	T(n)=\+O\qty(n\log^{d-1}n+nd\log(n))
.\] 

\section*{Problem 5}
About a day. Difficulty 3. No collaborators.

\end{document}

